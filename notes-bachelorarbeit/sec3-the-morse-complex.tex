\section{The Morse Complex}

\begin{prop}
    Define $C_k(M, f)$ as the $\Z/2$-Module generated by the critical points of $f$ of
    index $k$ and let $n_X(p, q) = \# \mathcal{L}(p, q) \mod 2$. Then
    \[ \del_X : C_k(M, f) \to C_{k-1}(M, f) \]
    such that if $p$ is a critical point of $f$ of index $k$, we have
    \[ \del_X(p) = \sum_{\Index(p) = \Index(q) + 1} n_X(p, q)q \]
    Then $(C_*(M, f), \del_X)$ is a chain complex.
\end{prop}

To proof that this is a chain complex we first have to examin the so called
\textit{space of broken trajectories}:

\begin{definition}
    The \textit{space of broken trajectories} is
    \[ 
        \Lb(p, q) = \bigcup_{ \{c_1, ..., c_i\} \subseteq \Crit(f )} 
            \Lt (p, c_1) \times \Lt (c_1, c_2) \times ... \times \Lt(c_i, q) 
    \]

    We can define a topology on this space as follows: \\
    For this, let $\lambda = (\lambda_1, ..., \lambda_l) \in \Lb(p, q)$. Then $\lambda$ 
    connects a certain number of critical points via the trajectories $\lambda_i$, where 
    $\lambda_i$ exits a critical point $c_i$ and enters $c_{i+1}$. Now let $U_i^-$ be a 
    neghboehood of the point at which $\lambda_i$ exits the chosen Morse neighborhood 
    around $c_{i-1}$ and $U_i^+$ be a neighborhood of the point at which $\lambda_i$
    enters the Morse-neighborhood of $c_i$. Then $U^-$ is the collection of the $U_i-$
    and $U_+$ the collection of the $U_i^+$. Then we say that a trajectory 
    $\mu = (\mu_1, ..., \mu_k) \in \mathcal{W}(\lambda, U^-, U^+)$, if there exist 
    integers
    \[ 0 < i_0 < ... < i_k = l \]
    such that:
    \begin{itemize}
        \item $\mu_j \in \Lt(c_{i_j}, c_{i_{j+1}})$ for every $j \leq k$
        \item $\mu_j$ exits the chart $\Omega(c_{j+1})$ the interior of the corresponding
            element in $U^-$ and enters the chart $\Omega(c_{j})$ through  interior of 
            the corresponding element in $U^+$.
    \end{itemize}
    The $\mathcal{W}(\lambda, U^-, U^+)$ form a fundamental system of open neighborhoods 
    for a topology on $\Lb(p, q)$. \todo{But do they?}
\end{definition}

It is clear \todo{Is it really?} that the resulting topology coincides with the topology
of $\Lt(p, q)$. 

\begin{remark}
    We have seen earlier that $\Lt(c_i, c_{i+1})$ is only well defined if 
    $c_i \neq c_{i + 1}$, and we know that the index is decreasing along trajectory lines,
    so $\Lt(c_i, c_{i+1}) = \varnothing$ if $\Index \, c_i \leq \Index \, c_{i + 1}$.
    Then $\Lb(p, q)$ is a tuple that gives a "direction" along trajectories
    from $p$ to $q$.
\end{remark}

As suggested by the notation, $\Lb(p, q)$ can be endowed with a topology, such that 
it is the closure of $\Lt(p, q)$, indeed its comactification. 
To show this is the aim of the next couple sections. We will then see that if 
$\Index \, p = \Index \, q + 2$, then $\Lb(p, q)$ is a one dimensional manifold with
boundary, and then $\del \Lb (p, q) = \Lt (p, q)$ is an even number.
