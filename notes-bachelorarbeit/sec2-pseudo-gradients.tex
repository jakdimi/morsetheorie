\section{Pseudo-Gradients}

\begin{definition}[Riemannian Metric, Gradient]
    A \textit{Riemannian metric} $g$ on a manifold $M$ is a choice of scalar products
    $g_p: T_pM \times T_pM \to \R$ for every point $p \in M$, such that for any vectorfields
    $X$ and $Y$, the map
    \[ p \mapsto g_p(X(p), Y(p)) \]
    is smooth. For $x, y \in T_pM$, we write 
    \[ \langle x, y \rangle := g_p(x, y)\]
    and 
    \[ || x || = \sqrt{g_p(x, x)} \]
    If $f: M \to \R$ is a smooth map, then the gradient of $f$ is a vectorfield $\grad f$, such
    that for any vectorfield $X$ the identity
    \[ \langle X, \grad f \rangle = \opd f X \]
    holds.
\end{definition}

\begin{definition}[Pseudo-Gradient]
    Let $f: M \to \R$ be a smooth function. A pseudo-gradient of $f$ is a vectorfield
    $X$ on $M$, such that 
    \begin{enumerate}
        \item $\opd f(p) X(p) \geq 0$, where equality holds if and only if $p$ is a 
            critical point of $f$,
        \item For every critical point of $f$, there exists a Morse chart neighborhood 
            in which $X$ coincides with $-\grad f$
    \end{enumerate}
\end{definition}

\begin{lemma}[Existence of Pseudo Gradients]
    For any smooth function $f: M \to \R$ there exists a pseudo-gradient of $f$.
\end{lemma}

\begin{proof}
    This follows from the fact that every Manifold can be equipped with a Riemannian
    metric, then the gradient is a pseudo-gradient.
\end{proof}

\begin{definition}[Stable and unstable Manifolds]
    Let $p$ be a critical point of a smooth function $f: M \to \R$. Denote by $\varphi_s$ the
    flow of a pseudo-gradient of $f$. Then We define the \textit{stable manifold} to be
    \[ W^s(p) = \left\{ q \in M: \lim_{s \to \infty} \varphi_s = p \right\} \]
    and the \textit{unstable manifold}
    \[ W^u(p) = \left\{ q \in M: \lim_{s \to -\infty} \varphi_s = p \right\} \]
\end{definition}

\begin{prop}
    Let $p$ be a critical point of index $k$ of a smooth function $f: M \to \R$. Then
    $W^s(p)$ is diffeomoric to the open disk of dimension $k$ and $W^u(p)$ is diffeomorphic
    to the open disc of dimension $n - k$.
\end{prop}

\begin{definition}
    Before the proof, we fix some notation and examine the situation in a specific real case:
    We set 
    \begin{align*}
        x_- &= (x_1, ..., x_k) : \R^n \to \R^k \\
        x_+ &= (x_{k+1}, ..., x_n): \R^n \to \R^{n - k}
    \end{align*}
    and then
    \[ Q = - || x_- ||^2 + || x_+ ||^2 \]
    This is a map $\R^n \to \R$, so we can utilize the gradient that we are used to. We get:
    \[ -\grad Q (x_-, x_+) = 2 (x_-, -x_+) \]
    Further for some $\varepsilon, \eta \in \R$ we set
    \[ U(\varepsilon, \eta) = \left\{ x \in \R^n : - \varepsilon < Q(x) < \varepsilon
    \text{ and } || x_- ||^2 || x_+ ||^2 \leq \eta(\varepsilon + \eta) \right\} \]
    We also define 
    \begin{align*}
        \del_{\pm} &= \left\{ x \in U : Q(x) = \pm \varepsilon 
            \text{ and } || x_{\mp} ||^2 \leq \eta \right\} \\
        \del_0 &= \left\{ x \in \del U : || x_- ||^2 || x_+ ||^2 = \eta(\varepsilon + \eta) \right\} 
    \end{align*}
    Then we have
    \[ \del U = \del_+ \cup \del_- \cup \del_0 \]
    We also fix $V_-, V_+ \subseteq \R^n$ to be the subspaces on which $Q$ is negative and
    positive definite respectively.  We get 
    \[ \del U \cap V_{\pm} \subseteq \del_{\pm} U \]
\end{definition}

\begin{proof}
    With the Morse-lemma, we obtain a Morse-neighborhood $U = U(\varepsilon, \eta)$ 
    with local coordinates $h^{-1} = (x_1, ..., x_n)$. We have
    \[ \tilde{f} = f \circ h : U \to \R \text{ with } \tilde{f} = f(p) + Q \]
    The only critical point of $\tilde{f}$ is $0$. We have
    \[ W^s(0) = U \cap V_+ \text{ and } W^u = U \cap V_- \]
    We also get a smooth embedding
    \[ ( h(\del_+ U \cap V_+) \times (-\infty, \infty]) /( h(\del_+ U \cap V_+) \times \{ \infty \}) 
    \to M; (x, s) \mapsto \varphi_s(x) \] 
    onto $W^s(p)$.
    $\del_+ U \cap V_+$ is a sphere of dimension $n - k - 1$, and $h$ is a diffeomorphism,
    so $( h(\del_+ U \cap V_+) \times (-\infty, \infty]) /( h(\del_+ U \cap V_+) \times \{ \infty \})$
    is diffeomorphic to the open disk of dimension $n - k$, and then $W^s(p)$ is as well.
    Similarly $W^u(p)$ is diffeomorphic to the open disk of dimension $k$.
\end{proof}

\begin{prop}
    Assume that $M$ is a compact manifold Let $X$ be a pseudo-gradient vectorfield of some
    smooth function $f: M \to \R$ and $\gamma$ be a trajectory of $X$. Then there
    exist critical points $p$ and $q$ of $f$, such that
    \[ \lim_{t \to \infty} \gamma(t) = p \text{ and } \lim_{t \to -\infty} \gamma(t) = q \]
\end{prop}

\begin{proof}
    We show that $\gamma(t)$ has a limit as $t$ tends to $+ \infty$, and that this limit is
    a critical point $p$ of $f$. This is the case if at some point the trajectory enters 
    $S_+(p) := \del_+ \Omega(p) \cap W^s(p)$. Suppose that this is not true. Then every time 
    the trajectory enters a morse neighborhood, it must also leave it again and never return to
    it, because $f$ is decreasing along $\gamma$. Let $t_0$ be the time that $\gamma$ leaves the
    last of the morse neighborhoods, i.e. the finite union
    \[ \Omega = \bigcup_{q \in \Crit(f)} \Omega(p) \]
    Because $df(x)X(x)$ is zero iff $x$ is a critical point of $f$, and $dfX \leq 0$, there 
    exists an $\varepsilon_0 > 0$, such that 
    \[ \forall x \in V - \Omega, df(x)X(x) \leq - \varepsilon_0 \]
    Then for every $t \geq t_0$, we have 
    \begin{align*}
        f(\gamma(t)) - f(\gamma(t_0)) &= \int_{t_0}^t \derive[(f \circ \gamma)]{u} \opd u \\
        &= \int_{t_0}^t \opd f (\gamma(u))X(\gamma(u))du \\
        &\leq - \varepsilon (t - t_0)
    \end{align*}
    And then 
    \[ \lim_{t \to +\infty} f(\gamma(t)) = - \infty \]
    which is absurd.
\end{proof}

\begin{theorem}[Fist Deformation Lemma]
    %TODO
\end{theorem}

\begin{proof}
    %TODO
\end{proof}

\begin{theorem}[Second Deformation Lemma]
    %TODO
\end{theorem}

\begin{proof}
    %TODO
\end{proof}

\begin{definition}[Transversality]
    Let $U, V \subseteq M$ be submanifolds. Then $U$ and $V$ are said to meet 
    \textit{transversly}, if for all $p \in U \cap V$ we have
    \[ T_pU + T_pV = T_pM \]
    If $U$ and $V$ meet transversly, we write
    \[ U \pitchfork V \]
    A vectorfield $X$ on $M$ is \textit{transversal} to a submanifold $U \in M$ of dimension
    $n - 1$, if for all $p \in U$ we have 
    \[ X(p) \notin T_p(U) \]
    Note that this is similar to the first definition, because then we have
    \[ \langle X(p) \rangle + T_pU = T_pM \]
    for all $p \in U$.
    We write 
    \[ X \pitchfork U \]
\end{definition}

\begin{definition}[Smale Condition]
    A pseudo gradient vectorfield is said to satisfy the \textit{Smale condition}, if
    all its stable and unstable manifolds meet transversly, i.e if for all critical points
    $p$ and $q$ of $f$, we have
    \[ W^u(p) \pitchfork W^s(q) \]
\end{definition}

\begin{prop}
    Let $(f, X)$ be a Smale pair. For critical points $p$ and $q$ of $f$ define
    \[ \mathcal{M}(p, q) = W^s \cap W^u = 
    \left\{ r \in M : \lim_{s \to \infty} \varphi_s(r) = p \text{ and } 
    \lim_{s \to -\infty} \varphi_s(r) = q \right\} \]
\end{prop}

\begin{prop}
    \textbf{Why does the action not need to be proper?}
    If $p \neq q$ are critical points of $f$, then
    $\R$ acts freely on $\mathcal{M}(p, q)$ via 
    \begin{align*}
        g: \R \times \mathcal{M}(p, q) & \to \mathcal{M}(p, q) \\
        (t, p) & \mapsto \varphi_t(p)
    \end{align*}
    We define $\mathcal{L}(p, q) := \mathcal{M}(p, q)/\R$. 
    Consequently $\mathcal{L}(p, q)$ is a manifold with 
    \[ \dim \mathcal{L}(p, q)  = \Index (p) - \Index (q) - 1 \]
\end{prop}

\begin{proof}
    Note that $\R$ with addition is a Lie-group, that acts freely on $\mathcal{L}(p, q)$. 
    This is easy to see:
    For any $x \in \mathcal{M}(p, q)$, the function $t \mapsto f(\varphi_t(x))$ is strictly
    decreasing, so if there are $t, t'$ s.th. $\varphi_t(x) = \varphi_{t'}(x)$, then
    $t = t'$. With the quotient manifold theorem, $\mathcal{L}(p, q)$ is a manifold of dimension
    $\dim \mathcal{M}(p, q) - \dim \R = \Index (p) - \Index (q) - 1 $.
\end{proof}

\begin{remark}
    The most convenient way to consider the quotient is the following. If $a$ is a 
    \textbf{regular} value of $f$ lying between $f(p)$ and $f(q)$, then $\mathcal{M}(p, q)$ 
    is transversal to the level set $f^{-1}(a)$. This level-set has codimension 1 and the 
    vector field $X$ is transversal to it. All trajectories starting at $p$ meet this level
    set at exactly one point, so $\mathcal{L}(p, q)$ can be identified by 
    $\mathcal{M}(p, q) \cap f^{-1}(a)$.

    Hence, if $p$ and $q$ are two distinct critical points and if the gradient used 
    satisfies the Smale-condition, then for $\mathcal{M}(p, q)$ or $\mathcal{L}(p, q)$
    to be non-empty, we must have 
    \[ \Index p \geq \Index q \]
    In other words, the index decreases along gradient lines.
\end{remark}
