\section{Pseudo-Gradients}

\begin{definition}[Riemannian Metric, Gradient]
    A \textit{Riemannian metric} $g$ on a manifold $M$ is a choice of scalar products
    $g_p: T_pM \times T_pM \to \R$ for every point $p \in M$, such that for any vectorfields
    $X$ and $Y$, the map
    \[ p \mapsto g_p(X(p), Y(p)) \]
    is smooth. For $x, y \in T_pM$, we write 
    \[ \langle x, y \rangle := g_p(x, y)\]
    and 
    \[ || x || = \sqrt{g_p(x, x)} \]
    If $f: M \to \R$ is a smooth map, then the gradient of $f$ is a vectorfield $\grad f$, such
    that for any vectorfield $X$ the identity
    \[ \langle X, \grad f \rangle = \opd f X \]
    holds.
\end{definition}

\begin{definition}[Pseudo-Gradient]
    Let $f: M \to \R$ be a smooth function. A pseudo-gradient of $f$ is a vectorfield
    $X$ on $M$, such that 
    \begin{enumerate}
        \item $\opd f(p) X(p) \geq 0$, where equality holds if and only if $p$ is a 
            critical point of $f$,
        \item in a Morse chart neighborhood of a critical point of $f$ the vectorfield
            $X$ coincides with $- \grad f$.
    \end{enumerate}
\end{definition}

\begin{lemma}[Existence of Pseudo Gradients]
    For any smooth function $f: M \to \R$ there exists a pseudo-gradient of $f$.
\end{lemma}

\begin{proof}
    This follows directly from the fact that on any manifold there exists a Riemannian metric.
\end{proof}

\begin{definition}[Stable and unstable Manifolds]
    Let $p$ be a critical point of a smooth function $f: M \to \R$. Denote by $\varphi_s$ the
    flow of a pseudo-gradient of $f$. Then We define the \textit{stable manifold} to be
    \[ W^s(p) = \left\{ q \in M: \lim_{s \to \infty} \varphi_s = p \right\} \]
    and the \textit{unstable manifold}
    \[ W^u(p) = \left\{ q \in M: \lim_{s \to -\infty} \varphi_s = p \right\} \]
\end{definition}

\begin{prop}
    Let $p$ be a critical point of index $k$ of a smooth function $f: M \to \R$. Then
    $W^s(p)$ is diffeomoric to the open disk of dimension $k$ and $W^u(p)$ is diffeomorphic
    to the open disc of dimension $n - k$.
\end{prop}

\begin{proof}
    %TODO
\end{proof}

\begin{prop}
    Assume that $M$ is a compact manifold Let $X$ be a pseudo-gradient vectorfield of some
    smooth function $f: M \to \R$ and $\gamma$ be a trajectory of $X$. Then there
    exist critical points $p$ and $q$ of $f$, such that
    \[ \lim_{t \to \infty} \gamma(t) = p \text{ and } \lim_{t \to -\infty} \gamma(t) = q \]
\end{prop}

\begin{proof}
    %TODO
\end{proof}

\begin{theorem}[Fist Deformation Lemma]
    %TODO
\end{theorem}

\begin{proof}
    %TODO
\end{proof}

\begin{theorem}[Second Deformation Lemma]
    %TODO
\end{theorem}

\begin{proof}
    %TODO
\end{proof}

\begin{definition}[Transversality]
    Let $U, V \subseteq M$ be submanifolds. Then $U$ and $V$ are said to meet 
    \textit{transversly}, if for all $p \in U \cap V$ we have
    \[ T_pU \oplus T_pV = T_pM \]
    If $U$ and $V$ meet transversly, we write
    \[ U \pitchfork V \]
    A vectorfield $X$ on $M$ is \textit{transversal} to a submanifold $U \in M$ of dimension
    $n - 1$, if for all $p \in U$ we have 
    \[ X(p) \notin T_p(U) \]
    Note that this is similar to the first definition, because then we have
    \[ \langle X(p) \rangle \oplus T_pU = T_pM \]
    for all $p \in U$.
    We write 
    \[ X \pitchfork U \]
\end{definition}

\begin{definition}[Smale Condition]
    A pseudo gradient vectorfield is said to satisfy the \textit{Smale condition}, if
    all its stable and unstable manifolds meet transversly, i.e if for all critical points
    $p$ and $q$ of $f$, we have
    \[ W^u(p) \pitchfork W^s(q) \]
\end{definition}