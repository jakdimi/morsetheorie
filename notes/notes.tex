\documentclass[a4paper,11pt]{article}

\usepackage{../general/preamble}

\begin{document}

\section{Non-Degenerate Functions}

\subsection{Introduction}

Let us consider a Torus $M$ tangent to a plane $V$:
\begin{figure}[H]
    \centering
    \includegraphics[width=0.4\linewidth]{resources/Diagram1.png}
    \label{fig:diagram1}
\end{figure}
Let $f: M \rightarrow \R$ be the distance of a point to the plane $V$. 
For a Number $a \in \R $, let $M^a$ be the set of all points $p \in M$, s.th. 
$f(p) \leq a$.
Then the following are true:
\begin{enumerate}
   \item[(1)] If $a < 0 < f(p)$, then $M^a = \varnothing$
   \item[(2)] If $f(p) < a < f(q)$, then $M^a$ is homeomorphic to a 2-cell.
   \item[(3)] If $f(q) < a < f(r)$, then $M^a$ is homeomorphic tp a cylinder.
   \item[(4)] If $f(q) < a < f(r)$, then $M^a$ is homeomorphic to a compact 
   manifold of genus one with a circle as a boundary.
   \item[(5)] If $f(s) < a$, then $M^a = M$.
\end{enumerate}

To describe how $M^a$ changes as it passes through the points $f(p)$, $f(q)$, 
$f(r)$, $f(s)$ it is conveniant to consider
homotopy type rather than homeomorphism type. 
\begin{enumerate}[leftmargin=2cm]
   \item[(1) $\rightarrow$ (2)]: In case (1), $M^a$ has the same homotopy type 
   as a point, so this step is the attaching of a 0-cell. 
   \item[(2) $\rightarrow$ (3)]: Is the operation of attaching a 1-cell. 
   \item[(3) $\rightarrow$ (4)]: Again is the operation of attaching a 1-cell. 
   \item[(4) $\rightarrow$ (5)]: Is the operation of attaching a 2-cell. 
\end{enumerate}

The definition of attaching a k-cell can be given as follows: \\
Let $S^k = \{x \in \R^{k+1} : \lVert x \rVert = 1\}$ be the $k$-sphere \\
and $D^k = \{ x \in \R^k : \lVert x \rVert \leq 1 \}$ be the $k$-disk.

Let $M$ and $N$ be manifolds, then $N$ is created from $M$ by atteching a 
k-cell, if $N$ is of the same homotopy type as a topological space $X$ s.th. 
there exists a pushout square in Top

\begin{figure}[H]
   \centering
   \begin{tikzcd}
      S^{k-1} \arrow[r] \arrow[d] \arrow[dr, phantom, "\ulcorner", very near end] & M \arrow[d] \\
      D^k \arrow[r]                                                               & X
   \end{tikzcd}
\end{figure}

A pushout square in a category $\mathcal{C}$ is a commutative square 

\begin{figure}[H]
   \centering
   \begin{tikzcd}
      A \arrow[r, "f_0"] \arrow[d, "f_1"] & B \arrow[d, "g_0"] \\
      C \arrow[r, "g_0"]                  & D 
   \end{tikzcd}
\end{figure}      
   
s.th if there is another commutative diagram

\begin{figure}[H]
   \centering
   \begin{tikzcd}
      A \arrow[r, "f_0"] \arrow[d, "f_1"] & B \arrow[d, "h_0"] \\
      C \arrow[r, "h_1"]                  & \tilde{D}
   \end{tikzcd}
\end{figure}

Then 

\begin{figure}[H]
   \centering
   \begin{tikzcd}
      A \arrow[r, "f_0"] \arrow[d, "f_1"] & B \arrow[d, "g_0"] \arrow[rdd, "h_0", bend left] & \\
      C \arrow[r, "g_1"] \arrow[drr, "h_1", bend right] & D & \\
        &   & \tilde{D} \arrow[ul, "\exists!h", dashed]
   \end{tikzcd}
\end{figure}

which commutes everywhere.

\subsection{Definitions and Lemmas}

\begin{definition}[critical Point, non-degenerate critical Point]
   \label{def:critical point}

   Let $M$ be a (smooth) manifold and $f: M \rightarrow \R$ be a smooth 
   function. Then $p \in M$ is called a \textit{critical point}, if the 
   tangent map $f_*: TM_p \rightarrow R$ is not zero. \\
   A critical point is called \textit{non-degenerate}, if for some local 
   coordinates $\varphi = (x_1, ..., x_n)$
   the matrix 
   \[ H_p^{\varphi}f := 
   \left(\frac{\partial^2 f}{\partial x_i \partial x_j}(p)\right)_{1 \leq i,j \leq n} \]
   is non-singular, i.e. invertable. \\
   $H_p^{\varphi}f$ is called the Hessian of $f$ at $p$ (wrt. $\varphi$).

\end{definition}

\begin{lemma}[Congruency of Hessians]
   \label{lemma:congruency}   

   Let $M$ be a manifold, $f: M \rightarrow \R$, $p$ a critical point of $f$
   and $\varphi := (x_1, ..., x_n)$ and $\psi := (y_1, ..., y_n)$ local 
   coordinates aroud $p$. Let 
   \[ D_p = \left( \frac{\partial x_i}{\partial y_j}(p) \right)_{1 \leq i, j \leq n}\].
   Then 
   \[ H_p^{\psi}f = D_p^T H_p^{\varphi}f D_p \]

\end{lemma}

\begin{proof}

   Let $M$ be a manifold, $f: M \rightarrow \R$ a smooth function.
   Let $p \in M$ be a critical point of $f$ and $\varphi = (x_1, ..., x_n)$, 
   $\psi = (y_1, ..., y_n)$ local coordinates in a nbhd. around $p$.
   From functoriality of the tangent space, we know that 
   \[ \frac{\partial f}{\partial y_k} = 
   \sum_{i=1}^n \frac{\partial f}{\partial x_i} \cdot \frac{\partial x_i}{\partial y_k} \,t \]
   so 
   \[ \pdderive[f]{y_k}{y_l} (p) 
   = \pderive{y_k} \left( \pderive[f]{y_l} \right) (p) 
   = \pderive{y_k} \left( \sum_{i=1}^n \pderive[f]{x_i} \cdot \pderive[x_i]{y_l} \right) (p) \]
   \[ = \sum_{i=1}^n \pderive{y_k} \left( \pderive[f]{x_i} \right) (p) \cdot \pderive[x_i]{y_l} (p) 
   + \sum_{i=1}^n \pderive[f]{x_i} (p) \cdot \pderive{y_k} \left( \pderive[x_i]{y_l} \right) (p) \]
   \[ = \sum_{i,j = 1}^n \pdderive[f]{x_i}{x_j} (p) \cdot \pderive[x_j]{y_k} (p) \cdot \pderive[x_i]{y_l}(p)
   + \sum_{i=1}^n \pderive[f]{x_i} (p) \cdot \pdderive[x_i]{y_k}{y_l}(p)\]
   
   Because $p$ is a critical point, $\pderive[f]{x_i} (p) = 0$ for all $i$, so then
   \[ \left( H_p^{\psi}f \right)_{k,l} 
   = \sum_{i,j = 1}^n \pdderive[f]{x_i}{x_j} (p) \cdot \pderive[x_j]{y_k}(p) \cdot \pderive[x_i]{y_l} (p) \]
   \[ = \left( D_p^T \cdot H_p^{\varphi}f \cdot D_p \right)_{k,l} \]

\end{proof}

\begin{lemma}[Invariance of non-degeneracy] 
   \label{lemma:non-degeneracy}
   Non-degeneracy does not depend on the chosen local coordinates.
\end{lemma}

\begin{proof}
   Let $f: M \rightarrow \R$ be smooth, $p$ a critical point of $f$ and 
   $\varphi = (x_1, ..., x_n)$, $\psi = (y_1, ..., y_n)$ local coordinates 
   around $p$. Assume that $p$ is non-non degenerate wrt. $\varphi$ Note that 
   $D_p$ from lemma~\ref{lemma:congruency} is invertable. Then
   \[ \text{det}(H_p^{\psi}f) = \text{det}(D_p^T \cdot H_p^{\varphi} \cdot D_p) = 
   \text{det}(D^T) \cdot \text{det}(H_p^{\varphi}) \cdot \text{det}(D_p) \neq 0 \]
\end{proof}

\begin{definition}[Index]
   \label{def:index}

   The \textit{index} of a Matrix $A$ is the number of (not necessarily 
   destinct) negative Eigenvalues of $A$ and is denoted by $\text{index}(A)$. \\
    The \textit{index} of a critical point $p$ of a function 
    $f: M \rightarrow \R$ (wrt. a chart $\varphi$) is the index of the matrix 
    $H_p^{\varphi}f$.
\end{definition}

\begin{lemma}[Invariance of the Index]
   \label{lemma:index}
   The index of a critical point does not depend on the chosen local coordinates.
\end{lemma}

\begin{proof}
   Let $f: M \rightarrow \R$ be a smooth function, $p$ a critical point of $f$
   and $\varphi$, $\psi$ two charts around $p$.
   As seen in lemma~\ref{lemma:congruency}, 
   $H_p^{\varphi}f = D^T \cdot H_p^{\psi}f \cdot D$, i.e.
   $H_p^{\varphi}$ is congruent to $H_p^{\psi}$, because $D$ is invertable. 
   Then by Sylvester's Law, 
   \[ \text{{index}}(H_p^{\varphi}f) = \text{{index}}(H_p^{\varphi}f) \]
\end{proof}

\begin{theorem}[Morse's Lemma]
   \label{theorem:morse lemma}

   Let $M$ be a manifold, $f: M \rightarrow \R$ smooth and $p$ a non-degenerate 
   critical point of $f$ of index $k$. Then there exist local coordinates 
   $\varphi = (x_1, ..., x_n)$, s.th,
   \[ f = f(p) - x_1^2 - ... - x_k^2 + x_{k+1}^2 + ... + x_n^2 \]
\end{theorem}

\begin{proof}
   %TODO
   \textbf{TODO}
   This is the proof of Morse's Lemma <3 . It's true i swear
\end{proof}

\subsection{Homotopy Type in Terms of critical Values}

\begin{definition}[1-parameter group of diffeomorphisms]
   \label{def:1-param group}
   A \textit{1-parameter group of diffeomorphisms} of a manifold $M$ is a 
   smooth map
   \[ \varphi: \R \times M \rightarrow M \text{ ; } (t, p) \mapsto \varphi_t(p) \]
   where $ \varphi_t $ is a diffeomorphism on $M$, and s.th. 
   \[ \varphi_{t+s} = \varphi_t \circ \varphi_s \] \\
   A vector field $X$ is said to generate a 1-parameter group of diffeomorphisms 
   $ \varphi $, if for every smooth real valued function the identity 
   \[ dfX(p) = \lim_{h \to 0} \frac{f(\varphi_h(p)) - f(p)}{h} \]
   holds for all points $p$ in $M$.
\end{definition}

\begin{lemma}[compactly supported vectorfields generate 1-parameter groups]
   \label{lemma:vector fields generate 1-parameter groups}
   A vector field whose support lies in a compact set $K \subseteq M$ generates 
   a unique 1-parameter group of diffeomorphisms on $M$.
\end{lemma}

\begin{proof}
   Define for a path $c: \R \rightarrow M$ the velocity vector 
   \[ \frac{dc}{dt} \in T_{c(t)}M \text{ as } \frac{dc}{dt}(f) = 
   \lim_{h \to 0} \frac{f(c(h + t)) - f(c(t))}{h}\]
   Let Now $X$ be a vector field whose support lies in a compact set $K \in M$. 
   Assume there exists a 1-parameter group of diffeomorphisms $\varphi_t$ that 
   is generated by $X$. Then the path $t \mapsto \varphi_t(p)$ for some fixed 
   $p \in M$ satisfies the differential equation
   \[ \frac{d\varphi_t(p)}{dt} = X(\varphi_t(p)) \]
   with initial condition $\varphi_0(p) = p$. This is true by definition:
   \[ \frac{d\varphi_t(p)}{dt}(f) 
   = \lim_{h \to 0} \frac{f(\varphi_{h + t}(p)) - f(\varphi_t(p))}{h} 
   = \lim_{h \to 0} \frac{f(\varphi_{h}(\varphi_t(p))) - f(\varphi_t(p))}{h} 
   = X(\varphi_t(p))(f) \]
   So to prove the lemma, one needs to show that such a map exists for all $p$ 
   that depends soomthly on $p$. \\
   Let $\psi = (x_1, ..., x_n)$ be local coordinates of some open neighborhood 
   $U$ of some point $p$. Then in local coordinates we get
   \[ X = X_1 \cdot \pderive{x_1} + ... + X_n \cdot \pderive{x_n} \]
   and 
   \[ \frac{d\varphi_t(p)}{dt} 
   = \frac{d\varphi_t^1(p)}{dt} \cdot \pderive{x_1} + ... + \frac{d\varphi_t^n(p)}{dt} \cdot \pderive{x_n}\]
   , so for $u_i = \varphi_t^i(p)$ we get the differential equation 
   \[ \frac{du_i}{dt} = X_i(u_1, ..., u_n) \]
   Because $X$ is compactly supported, $X_i$ is bounded, so with
   Picard-Lindelöf there exists $\varepsilon_i > 0$ s.th. the differential equation
   has a unique smooth solutionon on the interval $[-\varepsilon_i, \varepsilon_i]$. 
   This is true in every dimension. \\
   So for each point on $M$, there exists a neighborhood $U$ and a number 
   $\varepsilon > 0$, such that the differential equation
   \[ \frac{d\varphi_t(p)}{dt} = X(\varphi_t(p)) \]
   with initial condition 
   \[ \varphi_0(p) = p \]
   has a unique solution for $t \in [-\varepsilon, \varepsilon]$ for all 
   $p \in U$, which (apperently) is smooth in $(-\varepsilon, \varepsilon)$. \\
   Since $K$ is compact, it can be covered by a finite number of such 
   neighborhoods $U$. Let $\varepsilon_0 > 0$ be the smallest of the 
   corresponding numbers $\varepsilon$. For $p \notin K$, set $\varphi_t(p) = p$
   for all $t \in \R$, then we get a uniquesolution $\varphi_t(p)$ for all 
   $p \in M$ that is smooth in both vairables. Furthermore, we can see that 
   $\varphi_{t+s}(p) = \varphi_t \circ \varphi_s (p)$ , provided that $t+s$, $t$
   and $s \in (-\varepsilon_0, \varepsilon_0)$. \\ 
   We now need to define $\varphi_t(p)$ for $t \geq \varepsilon_0$. Any Number 
   can be expressed as $t = k \cdot \frac{\varepsilon_0}{2} + r$, where 
   $0 \leq r < \varepsilon_0/2$. Then set 
   \[ \varphi_t = 
   \varphi_{\varepsilon_0/2} \circ \varphi_{\varepsilon_0/2} \circ ... \circ \varphi_r \]
   , where $\varphi_{\varepsilon_0}$ is iterated $k$ times. For 
   $t \leq \varepsilon_0$, replace $\varphi_{\varepsilon_0/2}$ with 
   $\varphi_{-\varepsilon_0/2}$. 
   Apperently it is not difficult to verify that this is well defined, smooth,
   and satisfies the condition \[\varphi_{t+s} = \varphi_t + \varphi_s\].
\end{proof}

\begin{definition}[Morse Function]
   \label{def:morse function}

   Let $M$ be a manifold, $f:M \rightarrow \R$ a smooth map. $f$ is called a 
   \textit{Morse Function}, if all critical values of $f$ are non-degenerate.
\end{definition}

\begin{definition}[Riemannian Metric]
   \label{def:riemannian metric}
   A \textit{riemannian metric} $g$ is a smoothly chosen scalar product 
   $g_p: T_pM \times T_pM \rightarrow \R$ for every point $p \in M$, s.th. for 
   any vector fields $ X, Y: M \rightarrow TM $ the map 
   $ p \mapsto g_p(X(p), Y(p)) $ is smooth. \\
   A different definition can be given follows: \\
   A \textit{riemannian metric} $g$ on a smooth manifold $M$ is a smooth map 
   $g: M \rightarrow T^*M \otimes T^*M$; $p \mapsto g_p$,
   s.th. $g_p$ is a scalar product $T_pM \times T_pM \rightarrow \R$. \\
   A manifold together with a riemannian metric is called a 
   \textit{riemannian manifold}. \\
   We write:
   \[ g_p(x,y) \: =: \: <x, y>_g \: =: \: <x, y> \]
\end{definition}

\begin{definition}[gradient]
   \label{def:gradient}
   Let $(M, g^{TM})$ be a riemannian manifold. The \textit{gradient} of a smooth 
   map $f: M \rightarrow \R$ is the vector field which is characterized by the 
   identity 
   \[ <X, \text{grad} f> = X(f) \text{, where } X(f) = dfX \]
\end{definition}

\begin{definition}[smooth deformation retract]
   \label{def:deformation retract}
   Let $M$ be a smooth manifold. A smooth map $r: M \times \R \rightarrow M$ is 
   a \textit{smooth deformation retraction} onto a subspace $A$, if for every 
   \[ r(\cdot,0) = \text{id}_M \text{ , } f(M, 1) \subseteq A \text{ , } f(\cdot, 1) = \text{id}_A\]
   $A$ is called a \textit{smooth deformation retract} of $M$, if there exists a
   smooth deformation retraction from $M$ onto $A$.
\end{definition}

\begin{theorem}[First deformation Lemma, Milnor]
   \label{theorem:1st deformation lemma}
   Let $M$ be a manifold, $f: M \rightarrow \R$ smooth. Let $a < b \in \R$, 
   s.th. $f^{-1}[a, b]$ is compact and contains no critical points of $f$. Then 
   $M^a$ is diffeomorphic to $M^b$. \\ 
   Furthermore, $M^a$ is a deformation retract of $M^b$, s.th. the 
   inclusion $M^a \rightarrow M^b$ is a homotopy equivalence.
\end{theorem}

\begin{proof}
   Let $\rho: M \rightarrow \R$ be a smooth function where 
   \[ \rho(p) = 1/<\text{grad} f, \text{grad} f> \]
   for all $p \in f^{-1}[a, b]$ and which vanishes outside of a compact 
   neighborhood of $f^{-1}[a, b]$, i.e. which is compactly supported.
   Note $\rho$ is well defined inside $f^{-1}[a, b]$, because there are no 
   critical points in $f^{-1}[a, b]$. \\ 
   Then the vector field $X$ which is defined by
   \[ X(p) = \rho(p) \cdot \text{grad} f (p) \]
   is compactly supported as well, i.e satisfies the conditions of 
   lemma~\ref{lemma:vector fields generate 1-parameter groups}, hence X generates 
   a unique 1-parameter group of diffeomorphisms on $M$
   \[ \varphi_t : M \rightarrow M \]. 
   For fixed $p \in M$, consider the function 
   $ t \mapsto f(\varphi_t(p)) $. If $\varphi_t(p)$ lies in $f^{-1}[a, b]$, then
   \[ \frac{d}{dt}f (\varphi_t(p)) = <d\varphi_t(p), \text{grad}f> = <X, \text{grad}f> = + 1 \].
   Thus the correnspondence 
   \[ t \rightarrow f(\varphi_t(p)) \]
   is linear with derivative $+1$, if $p \in \varphi_t^{-1}(f^{-1}[a, b])$. 
   Note that $f \circ \varphi_t$ is "strictly increasing" in the following sense:
   \[ f(p) > f(q) \Leftrightarrow f(\varphi_t(p)) > f(\varphi_t(p)) \]
   , as long as $p$, $q$, $\varphi_t(p)$ and $\varphi_t(q)$ are in the interior
   of the support of $X$. Also the map $t \mapsto f(\varphi_t(p))$ is strictly
   increasing, as long as $p$ and $\varphi_t(p)$ are in the interior of the
   support of $X$. This is true because the paths $t \mapsto \varphi_t(p)$ move
   parallel to the gradient of $f$ and have velocity $>0$ inside the interior 
   of the support of $X$. \\
   Then the set $f^{-1}(a)$ is diffeomorphically mapped onto $f^{-1}(b)$ by $\varphi_{b-a}$: \\
   Let $p \in f^{-1}(a)$. then for $t \in [0, b-a]$, The condition 
   $\varphi_t(p) \in f^{-1}([a, b])$ from above is satisfied, because
   $t \mapsto \varphi_t(p)$ is strictly increasing in the above interval, so 
   \[ f(\varphi_t(p)) = f(\varphi_0(p)) + t = f(p) + t 
   = a + t \Rightarrow f(\varphi_{b-a}(p)) = b \]
   Similarly, if $p \in f^{-1}(b)$, then 
   $f((\varphi_{b-a})^{-1}(p)) = f(\varphi_{a-b}(p)) = a$. 
   Now let $p \in f^{-1}(-\infty, a)$ wlog. let p be in the interior of the 
   support of $X$, otherwise $\varphi_{b-a}(p) = p \in M^b$. \\
   If we now show that for $p \in f^{-1}(-\infty, a)$, $\varphi_{b-a}(p) \in M^b$,
   then by the same argument as before, we use the inverse $\varphi_{b-a}$ and
   this then implies the first assertion of the theorem. For this, choose such a
   $p$. Assume $\varphi_{b-a}(p) \notin M^b$. Then 
   $f(\varphi_{b-a}(p)) > b = f(\varphi_{b-a}(q))$ for some $q \in f^{-1}(a)$.
   But then $f(p) > f(q) \Rightarrow p \notin M^a$. \\
   Now we still have to show that $M^a$ is a deformation retract of $M^b$, and 
   that the inclusion $M^a \rightarrow M^b$ is a homotopy equivalence. \\
   Take $r: M^b \times \R \rightarrow M^b$, 
   \[ 
      r(p, t) = \begin{cases}
         p & \text{if } f(p) \leq a \\
         \varphi_{t(a - f(p))}(p) & \text{if } a \leq f(p) \leq b
      \end{cases}
   \]
   Then $r(\cdot, 0)$ is the identity and $r(\cdot, 1)$ restricted to $M^a$ is
   the identity and $r(1, p) \in M^a$ for all $p \in M^b$. 
\end{proof}

\begin{remark}
   Note that for $M^a$ and $M^b$ to be $C^r$ diffeomorphic, $f$ needs only to be
   $C^{r+1}$.
\end{remark}

\begin{corollary}[Hirsch]
   Let $f: M \rightarrow [a, b]$ be a $C^{r+1}$-map on a compact manifold with 
   boundary, $1 \leq r \leq \omega$. Suppose $f$ has no critical values and 
   $f(\partial M) = \{a, b\}$. Then there is a $C^r$ diffeomorphism such that the 
   following diagram commutes:
   \begin{figure}[H]
      \centering
      \begin{tikzcd}
         f^{-1}(a) \times [a, b] \arrow[rr, "F"] \arrow[dr] & & M \arrow[dl, "f"] \\
         & \left[ a, b \right] &
      \end{tikzcd}
   \end{figure}
\end{corollary}

\begin{proof}
   Take $F(p,t) = r(p, t)$, where $r$ is the deformation retraction from the 
   proof above.
\end{proof}

\begin{corollary}[Hirsch]
   Let $M$ be a compact manifold with boundary, such that $\partial M = A \cup B$
   and $A$ and $B$ are disjoint. If there exists a $C^2$ map $f: M \rightarrow [0,1]$
   with no critical points, such that $f(A) = 0$ and $f(B) = 1$, then $M$ is 
   $C^1$-diffeomorphic to the cylinders $A \times [0, 1]$ and $B \times [0, 1]$.
\end{corollary}
    
\begin{theorem}[Second deformation Lemma, Milnor]
   \label{theorem:2nd deformation lemma}
   Let $M$ be a manifold, $f: M \rightarrow \R$ smooth and $p$ be a 
   non-degenerate critical point of $f$ of index $k$. Let $c := f(p)$ and 
   $\varepsilon > 0$, s.th. $f^{-1}[c-\varepsilon, c+\varepsilon]$ is compact and 
   contains no critical points of $f$. Then $M^{c+\varepsilon}$ is obtained from 
   $M^{c-\varepsilon}$ by attaching a $k$-cell.
\end{theorem}

\begin{proof}
   %TODO
   \textbf{TODO}
   I'm not so sure if this is true but i guess
\end{proof}

\subsection{Examples}

\subsection{The Morse Inequalities}

\begin{lemma}[Weak Morse Inequalities]
   \label{lemma:weak morse inequalities}
   Let $f: M \rightarrow \R$ be a smooth map from a compact smooth manifold. Let 
   $C_{\lambda}$ be the number of critical points of $f$ of degree $\lambda$.
   \begin{enumerate}
      \item $R_{\lambda} (M) \leq C_{\lambda}$
      \item $\sum_{\lambda} (-1)^{\lambda} \cdot R_{\lambda}(M) = 
         \sum_{\lambda} (-1)^{\lambda} \cdot C_{\lambda} $
   \end{enumerate}
\end{lemma}

\begin{proof}
   %TODO
   \textbf{TODO}
\end{proof}

\begin{lemma}[Morse Inequalities]
   \label{lemma:strung morse inequalities}
   Something Something
   \begin{enumerate}
      \item $ S_{\lambda}(M) \leq \sum_{i = 1}^k S_{\lambda}(M^{a_i}, M^{a_{i-1}}) 
         = C_{\lambda} - C_{\lambda - 1} + ... \pm C_0 $
      \item $ R_{\lambda}(M) - R_{\lambda - 1}(M) + ... \pm R_0(M) \leq C_{\lambda} - C_{\lambda - 1} ... \pm C_0 $
   \end{enumerate}
\end{lemma}

\begin{proof}
   %TODO
   \textbf{TODO}
\end{proof}

%\printbibliography[title={Whole bibliography}]

\end{document}
