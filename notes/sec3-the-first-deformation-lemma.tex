\section{The first Deformation Lemma}

\begin{definition}[1-parameter group of diffeomorphisms]
   \label{def:1-param group}
   A \textit{1-parameter group of diffeomorphisms} of a manifold $M$ is a 
   smooth map
   \[ \varphi: \R \times M \rightarrow M \text{ ; } (t, p) \mapsto \varphi_t(p) \]
   where $ \varphi_t $ is a diffeomorphism on $M$, and s.th. 
   \[ \varphi_{t+s} = \varphi_t \circ \varphi_s \] \\
   A vector field $X$ is said to generate a 1-parameter group of diffeomorphisms 
   $ \varphi $, if for every smooth real valued function the identity 
   \[ dfX(p) = \lim_{h \to 0} \frac{f(\varphi_h(p)) - f(p)}{h} \]
   holds for all points $p$ in $M$.
\end{definition}

\begin{lemma}[compactly supported vectorfields generate 1-parameter groups]
   \label{lemma:vector fields generate 1-parameter groups}
   A vector field whose support lies in a compact set $K \subseteq M$ generates 
   a unique 1-parameter group of diffeomorphisms on $M$.
\end{lemma}

\begin{proof}
   Define for a path $c: \R \rightarrow M$ the velocity vector 
   \[ \frac{dc}{dt} \in T_{c(t)}M \text{ as } \frac{dc}{dt}(f) = 
   \lim_{h \to 0} \frac{f(c(h + t)) - f(c(t))}{h}\]
   Let Now $X$ be a vector field whose support lies in a compact set $K \in M$. 
   Assume there exists a 1-parameter group of diffeomorphisms $\varphi_t$ that 
   is generated by $X$. Then the path $t \mapsto \varphi_t(p)$ for some fixed 
   $p \in M$ satisfies the differential equation
   \[ \frac{d\varphi_t(p)}{dt} = X(\varphi_t(p)) \]
   with initial condition $\varphi_0(p) = p$. This is true by definition:
   \[ \frac{d\varphi_t(p)}{dt}(f) 
   = \lim_{h \to 0} \frac{f(\varphi_{h + t}(p)) - f(\varphi_t(p))}{h} 
   = \lim_{h \to 0} \frac{f(\varphi_{h}(\varphi_t(p))) - f(\varphi_t(p))}{h} 
   = X(\varphi_t(p))(f) \]
   So to prove the lemma, one needs to show that such a map exists for all $p$ 
   that depends soomthly on $p$. \\
   Let $\psi = (x_1, ..., x_n)$ be local coordinates of some open neighborhood 
   $U$ of some point $p$. Then in local coordinates we get
   \[ X = X_1 \cdot \pderive{x_1} + ... + X_n \cdot \pderive{x_n} \]
   and 
   \[ \frac{d\varphi_t(p)}{dt} 
   = \frac{d\varphi_t^1(p)}{dt} \cdot \pderive{x_1} + ... + \frac{d\varphi_t^n(p)}{dt} \cdot \pderive{x_n}\]
   , so for $u_i = \varphi_t^i(p)$ we get the differential equation 
   \[ \frac{du_i}{dt} = X_i(u_1, ..., u_n) \]
   Because $X$ is compactly supported, $X_i$ is bounded, so with
   Picard-Lindelöf there exists $\varepsilon_i > 0$ s.th. the differential equation
   has a unique smooth solutionon on the interval $[-\varepsilon_i, \varepsilon_i]$. 
   This is true in every dimension. \\
   So for each point on $M$, there exists a neighborhood $U$ and a number 
   $\varepsilon > 0$, such that the differential equation
   \[ \frac{d\varphi_t(p)}{dt} = X(\varphi_t(p)) \]
   with initial condition 
   \[ \varphi_0(p) = p \]
   has a unique solution for $t \in [-\varepsilon, \varepsilon]$ for all 
   $p \in U$, which (apperently) is smooth in $(-\varepsilon, \varepsilon)$. \\
   Since $K$ is compact, it can be covered by a finite number of such 
   neighborhoods $U$. Let $\varepsilon_0 > 0$ be the smallest of the 
   corresponding numbers $\varepsilon$. For $p \notin K$, set $\varphi_t(p) = p$
   for all $t \in \R$, then we get a uniquesolution $\varphi_t(p)$ for all 
   $p \in M$ that is smooth in both vairables. Furthermore, we can see that 
   $\varphi_{t+s}(p) = \varphi_t \circ \varphi_s (p)$ , provided that $t+s$, $t$
   and $s \in (-\varepsilon_0, \varepsilon_0)$. \\ 
   We now need to define $\varphi_t(p)$ for $t \geq \varepsilon_0$. Any Number 
   can be expressed as $t = k \cdot \frac{\varepsilon_0}{2} + r$, where 
   $0 \leq r < \varepsilon_0/2$. Then set 
   \[ \varphi_t = 
   \varphi_{\varepsilon_0/2} \circ \varphi_{\varepsilon_0/2} \circ ... \circ \varphi_r \]
   , where $\varphi_{\varepsilon_0}$ is iterated $k$ times. For 
   $t \leq \varepsilon_0$, replace $\varphi_{\varepsilon_0/2}$ with 
   $\varphi_{-\varepsilon_0/2}$. 
   Apperently it is not difficult to verify that this is well defined, smooth,
   and satisfies the condition \[\varphi_{t+s} = \varphi_t + \varphi_s\].
\end{proof}

\begin{definition}[Riemannian Metric]
   \label{def:riemannian metric}
   A \textit{riemannian metric} $g$ is a smoothly chosen scalar product 
   $g_p: T_pM \times T_pM \rightarrow \R$ for every point $p \in M$, s.th. for 
   any vector fields $ X, Y: M \rightarrow TM $ the map 
   $ p \mapsto g_p(X(p), Y(p)) $ is smooth. \\
   A different definition can be given follows: \\
   A \textit{riemannian metric} $g$ on a smooth manifold $M$ is a smooth map 
   $g: M \rightarrow T^*M \otimes T^*M$; $p \mapsto g_p$,
   s.th. $g_p$ is a scalar product $T_pM \times T_pM \rightarrow \R$. \\
   A manifold together with a riemannian metric is called a 
   \textit{riemannian manifold}. \\
   We write:
   \[ g_p(x,y) \: =: \: <x, y>_g \: =: \: <x, y> \]
\end{definition}

\begin{definition}[gradient]
   \label{def:gradient}
   Let $(M, g^{TM})$ be a riemannian manifold. The \textit{gradient} of a smooth 
   map $f: M \rightarrow \R$ is the vector field which is characterized by the 
   identity 
   \[ <X, \grad f> = X(f) \text{, where } X(f) = \opd fX \]
\end{definition}

\begin{definition}[deformation retract]
   \label{def:deformation retract}
   Let $X$ be a topological space. A continuous map $r: X \times [0, 1] \rightarrow X$ is 
   a \textit{deformation retraction} onto a subspace $A$, if for every 
   \[ r(\cdot,0) = \id_X \text{ , } r(X, 1) \subseteq A \text{ , } r(\cdot, 1)|_A = \id_A\]
   $A \subseteq X$ is called a \textit{deformation retract} of $M$, if there exists a
   deformation retraction from $X$ onto $A$.
   If $X = M$ is a smooth manifold, and there exists a smooth deformation
   retraction from $M$ onto $A$, $A$ is called a \textit{smooth deformation 
   retract}.
\end{definition}

\begin{lemma}
   If $A$ is a deformation retract of a space $X$, then the inclusion 
   $A \rightarrow X$ is a homotopy equivalence.
\end{lemma}

\begin{proof}
   Let $r$ be the deformation retraction from $X$ onto $A$. Let 
   $\iota: A \rightarrow X$ be the inclusion. Let 
   $f: X \rightarrow A$; with $f(x) = r(x, 1)$. This is well defined since 
   $r(\cdot, 1)$ maps into $A$. Then $f \circ \iota (x) = f(x) = x$. Also, 
   $\iota \circ f(x) = f(x) = r(x, 1)$, so $r$ is a homotopy between 
   $f \circ \iota$ and $\id_X$, so $A$ and $X$ have the same homotopy type.
\end{proof}

\begin{remark}
   The following is also true: \\
   Let $A$ and $B$ be subspaces of $X$. Then
   $A$ and $B$ have the same homotopy type, if and only if they both are 
   deformation retracts of some other subspace $C$. \\
   "$\Leftarrow$" follows directly from the lemma above, while "$\Rightarrow$"
   is relatively hard to prove and requires some more topological theory.
\end{remark}

\begin{theorem}[First deformation Lemma, Milnor]
   \label{theorem:1st deformation lemma}
   Let $M$ be a manifold, $f: M \rightarrow \R$ smooth. Let $a < b \in \R$, 
   s.th. $f^{-1}[a, b]$ is compact and contains no critical points of $f$. Then 
   $M^a$ is diffeomorphic to $M^b$. \\ 
   Furthermore, $M^a$ is a deformation retract of $M^b$, s.th. the 
   inclusion $M^a \rightarrow M^b$ is a homotopy equivalence.
\end{theorem}

\begin{proof}
   Let $\rho: M \rightarrow \R$ be a smooth function where 
   \[ \rho(p) = 1/<\grad f, \grad f> \]
   for all $p \in f^{-1}[a, b]$ and which vanishes outside of a compact 
   neighborhood of $f^{-1}[a, b]$, i.e. which is compactly supported.
   Note $\rho$ is well defined inside $f^{-1}[a, b]$, because there are no 
   critical points in $f^{-1}[a, b]$. \\ 
   Then the vector field $X$ which is defined by
   \[ X(p) = \rho(p) \cdot \grad f (p) \]
   is compactly supported as well, i.e satisfies the conditions of 
   lemma~\ref{lemma:vector fields generate 1-parameter groups}, hence X generates 
   a unique 1-parameter group of diffeomorphisms on $M$
   \[ \varphi_t : M \rightarrow M \]. 
   For fixed $p \in M$, consider the function 
   $ t \mapsto f(\varphi_t(p)) $. If $\varphi_t(p)$ lies in $f^{-1}[a, b]$, then
   \[ \frac{d}{dt}f (\varphi_t(p)) = <d\varphi_t(p), \grad f> = <X, \grad f> = + 1 \].
   Thus the correnspondence 
   \[ t \rightarrow f(\varphi_t(p)) \]
   is linear with derivative $+1$, if $p \in \varphi_t^{-1}(f^{-1}[a, b])$. 
   Note that $f \circ \varphi_t$ is "strictly increasing" in the following sense:
   \[ f(p) > f(q) \Leftrightarrow f(\varphi_t(p)) > f(\varphi_t(p)) \]
   , as long as $p$, $q$, $\varphi_t(p)$ and $\varphi_t(q)$ are in the interior
   of the support of $X$. Also the map $t \mapsto f(\varphi_t(p))$ is strictly
   increasing, as long as $p$ and $\varphi_t(p)$ are in the interior of the
   support of $X$. This is true because the paths $t \mapsto \varphi_t(p)$ move
   parallel to the gradient of $f$ and have velocity $>0$ inside the interior 
   of the support of $X$. \\
   Then the set $f^{-1}(a)$ is diffeomorphically mapped onto $f^{-1}(b)$ by $\varphi_{b-a}$: \\
   Let $p \in f^{-1}(a)$. then for $t \in [0, b-a]$, The condition 
   $\varphi_t(p) \in f^{-1}([a, b])$ from above is satisfied, because
   $t \mapsto \varphi_t(p)$ is strictly increasing in the above interval, so 
   \[ f(\varphi_t(p)) = f(\varphi_0(p)) + t = f(p) + t 
   = a + t \Rightarrow f(\varphi_{b-a}(p)) = b \]
   Similarly, if $p \in f^{-1}(b)$, then 
   $f((\varphi_{b-a})^{-1}(p)) = f(\varphi_{a-b}(p)) = a$. 
   Now let $p \in f^{-1}(-\infty, a)$ wlog. let p be in the interior of the 
   support of $X$, otherwise $\varphi_{b-a}(p) = p \in M^b$. \\
   If we now show that for $p \in f^{-1}(-\infty, a)$, $\varphi_{b-a}(p) \in M^b$,
   then by the same argument as before, we use the inverse $\varphi_{b-a}$ and
   this then implies the first assertion of the theorem. For this, choose such a
   $p$. Assume $\varphi_{b-a}(p) \notin M^b$. Then 
   $f(\varphi_{b-a}(p)) > b = f(\varphi_{b-a}(q))$ for some $q \in f^{-1}(a)$.
   But then $f(p) > f(q) \Rightarrow p \notin M^a$. \\
   Now we still have to show that $M^a$ is a deformation retract of $M^b$, and 
   that the inclusion $M^a \rightarrow M^b$ is a homotopy equivalence. \\
   Take $r: M^b \times \R \rightarrow M^b$, 
   \[ 
      r(p, t) = \begin{cases}
         p & \text{if } f(p) \leq a \\
         \varphi_{t(a - f(p))}(p) & \text{if } a \leq f(p) \leq b
      \end{cases}
   \]
   Then $r(\cdot, 0)$ is the identity and $r(\cdot, 1)$ restricted to $M^a$ is
   the identity and $r(1, p) \in M^a$ for all $p \in M^b$. 
\end{proof}

\begin{remark}
   Note that for $M^a$ and $M^b$ to be $C^r$ diffeomorphic, $f$ needs only to be
   $C^{r+1}$. Also we have proved that the level sets between $a$ and $b$ are
   all diffeomorphic.
\end{remark}

\begin{corollary}[Hirsch]
   Let $f: M \rightarrow [a, b]$ be a $C^{r+1}$-map on a compact manifold with 
   boundary, $1 \leq r \leq \omega$. Suppose $f$ has no critical values and 
   $f(\del M) = \{a, b\}$. Then there is a $C^r$ diffeomorphism such that the 
   following diagram commutes:
   \begin{figure}[H]
      \centering
      \begin{tikzcd}
         f^{-1}(a) \times [a, b] \arrow[rr, "F"] \arrow[dr] & & M \arrow[dl, "f"] \\
         & \left[ a, b \right] &
      \end{tikzcd}
   \end{figure}
\end{corollary}

\begin{proof}
   Take $F(p,t) = r(p, t)$, where $r$ is the deformation retraction from the 
   proof above.
\end{proof}

\begin{corollary}[Hirsch]
   Let $M$ be a compact manifold with boundary, such that $\del M = A \cup B$
   and $A$ and $B$ are disjoint. If there exists a $C^2$ map $f: M \rightarrow [0,1]$
   with no critical points, such that $f(A) = 0$ and $f(B) = 1$, then $M$ is 
   $C^1$-diffeomorphic to the cylinders $A \times [0, 1]$ and $B \times [0, 1]$.
\end{corollary}
