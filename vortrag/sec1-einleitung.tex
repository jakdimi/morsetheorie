\section{Einführung}

Wiederholung von letzter Woche:

\begin{definition}
    Sei $f: M \to \R$ eine glatte Abbildung, $\dim(M) = n$, $p$ ein kritischer 
    Punkt von $f$. Sei $\varphi = (x_1, ..., x_n)$ ein lokales Koordinatensystem 
    in einer Umgebung von $p$. Dann ist der \textit{Index} von $p$ die Anzahl 
    der negativen Eigenwerte von der Matrix
    \[ H_p^{\varphi}f := \left( \pdderive[f]{x_i}{x_j} \right)_{1 \leq i, j \leq n} \]
    $p$ heißt \textit{nicht degneriert}, falls  $H_p^{\varphi}f$ invertierbar ist.
\end{definition}

\begin{theorem}[Das Morse-Lemma]
    Sei $M$ eine $n$-dimensionale glatte Mannigfaltigkeit, $f: M \to \R$ glatt 
    und $p$ ein nicht degenerierter kritischer Punkt von $f$ mit Index $k$. Dann 
    existerit ein lokales Koordinatensystem $(x_1, ..., x_n)$ in einer Umgebung
    $U$ von $p$, sodass
    \[ f = f(p) - x_1^2 - ... - x_k^2 + x_{k + 1}^2 + ... + x_n^2 \]
    und 
    \[ x_1(p) = ... = x_n (p) = 0 \]
\end{theorem}
\sectiondone

Sei $M$ eine glatte Mannigfaltigkeit, $f: M \to \R$ eine glatte Abbildung, 
$a \in \R$. Dann ist $M^a = f^{-1}(- \infty, a]$ eine \textit{Subniveaumenge} 
von $f$. Eine \textit{Morse-Funktion} ist eine glatte Funktion $f: M \to \R$, 
sodass alle kritischen Punkte von $f$ nicht degeneriert sind.
Das Ziel des Vortrags ist es, die Topologie der Subnieveuamengen von 
Morse-Funktionen zu verstehen. Wir Untersuchen die Situation anhand des Torus. 
Man stellt sich Morse-Funktionen einfachheitshalber als "Höhenfunktionen" vor.

Um die vorherschende Situation besser zu Untersuchen benötigen wir eine weitere
Definition, die im Weitesten eine Umbenennung bekannter Objekte ist:

\begin{definition}[Eine $k$-Zelle anbringen]
    Es sei $X$ ein Topologischer Raum. Seien
    \[ e^k = D^k = \{(x_1, ..., x_k) \in \R^k: x_1^2 + ... + x_k^2 \leq 1\} \]
    \[ \varphi: \del e^k \rightarrow X \text{ stetig } \]
    \[ X \cup_{\varphi} e^k = (X \amalg e^k) / \sim, \text{ wobei } \]
    \[ \del e^k \ni x \sim y \in X \Leftrightarrow \varphi(x) = y \]
    Dann heißt $e^k$ $k$-Zelle, $\varphi$ Anheftungsabbildung 
    und $X \cup_{\varphi} e^k$ heißt $X$ mit einer $k$-Zelle 
    angebracht.
    Per Definition ist 
    \[ \varnothing \cup e^0 := e^0 = \ast \text{ und falls } X \neq \varnothing : X \cup e^0 := X \]
\end{definition}

\textbf{ZEICHNUNG 1-Zelle anbringen}

Ein Paar Beispiele: 
\begin{itemize}
    \item Eine $1$-Zelle lässt uns Zusammenhangskomponenten mit einer "Schnur"
            verbinden, oder ist das Anheften einer "Schlaufe".
            \begin{figure}[H]
                \centering
                \includegraphics[width=0.45\linewidth]{resources/Me-Diagram1-attaching-a-1-cell.jpeg}
                \label{fig:me-diagram1}
            \end{figure}
    \item Eine $2$-Zelle anzubringen kann sein wie das anbringen einer Blase oder
        das stopfen eines Loches durch eine "Membran".
\end{itemize}

Eine $k$-Zelle kann also unteranderem zwei Sachen: "Löcher", die von der $k$-ten 
Homologie "gemessen" werden zu erschaffen oder "Löcher", die von der $(k-1)$-ten 
Homologie "gemessen" werden zu "stopfen". Wenn wir also unsere Mannigfaltigkeit 
in Bezug auf solche Zellen bringen können, dann macht es uns das leichter die 
Topologie (und Homologie) der Mannigfaltigkeit besser zu verstehen. Wenn wir
sogar die gesamte Mannigfaltigkeit durch iteratives anbringen von $k$-Zellen
an den leeren Raum "erschaffen" können, dann können wir einigen Aussagen über 
die Homologie des Raumes machen. 


\begin{figure}[H]
    \centering
    \includegraphics[width=0.6\linewidth]{resources/Me-Diagram2-torus-plane.png}
    \label{fig:me-diagram2}
    \caption{Torus auf der Ebene}
\end{figure}

Es sei $f$ die Abbildung vom Torus nach $\R$, die jedem Punkt den minimalen
Abstand zu der eingezeichneten Ebene zuordnet. 
Die kritischen Punkte dieser Höhenfunktion sind $p$, $q$, $r$ und $s$.

\begin{figure}[H]
    \centering
    \includegraphics[width=0.8\linewidth]{resources/Me-Diagram3-torus-example.jpeg}
    \label{fig:me-diagram3}
    \caption{Niveaumengen der Höhenfunktion des Torus}
\end{figure}

\begin{itemize}
    \item Für $a < f(p)$ ist $M^a = \varnothing$.
    \item Für $f(p) \leq a < f(p)$ ist $M^a$ homotopieäquivalent zum Punkt.
    \item Für $f(p) \leq a < f(q)$ ist $M^a$ homotopieäquivalent zum Kreis,
        an dem ein "Henkel" befestigt wurde.
    \item Für $f(q) \leq a < f(r)$ ist $M^a$ Homotopieäquivalent zum Zyliner,
        an dem ein "Henkel" befestigt wurde.
    \item Für $f(s) \leq$ ist $M^a$ der Torus selbst.
\end{itemize}

Wir bemerken: 
\begin{itemize}
    \item Gibt es im Intervall $[a, b]$ keine kritischen Werte, so sind $M^a$ und
        $M^b$ diffeomorph.
    \item Gibt es in $f^{-1}[a, b]$ genau einen kritischen Punkt, dann hat $M^b$ 
        den Homotopie-Typ von $M^a$ mit einer $k$-Zelle angebracht, wobei 
        $k \in \N_0$
\end{itemize}

Um diese zwei Aussagen zu präzisieren, brauchen wir einige Definitionen:

\begin{definition}[Deformationsretrakt]
    Sei $X$ ein topologischer Raum und $A \subseteq X$ ein Unterraum von $X$.
    Eine stetige Abbildung $r: X \times [0, 1] \rightarrow X$ heißt 
    \textit{Deformationsretraktion} auf $A$, falls gelten:
    \begin{align*}
        & r(\cdot, 0) = \id_X \\
        & r(X, 1) \subseteq A \\
        & r(\cdot, 1)|_A = \id_A \\
    \end{align*}
    $A$ heißt \textit{Deformationsretrakt}, falls eine Deformationsretraktion
    von $X$ auf $A$ existiert.
\end{definition}

Deformationsretrakte haben einige schöne Eigenschaften. Falls $A$ ein
Deformationsretrakt von $X$ ist, so ist die Inklusion $A \rightarrow X$ eine
Homotopieäquivalenz. Da Homotopieäquivalenz transitiv ist, hat ein weiterer
Deformationsretrakt $B$ von $X$ deshalb denselben Homotopietypen wie $A$. \\
Es ist eine Tatsache, dass zwei Unterräume $A$ und $B$ genau dan denselben 
Homotopietypen haben, wenn sie beide Deformationsretrakte eines gemeinsamen 
Oberraumes sind. Um dies einzusehen benötigt man weitere topologische 
Theorie.