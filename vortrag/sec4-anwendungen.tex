\section{Anwendungen}

\begin{theorem}
    \label{theorem:Sn}
    Sei $M$ eine $n$-dimensionale Mannigfaltigkeit, sodass eine glatte Funktion
    $f: M \to \R$ existiert, die genau zwei kritische Punkte besitzt, die beide
    nicht degeneriert sind. Dann ist $M$ homeomorph zu $S^n$.
\end{theorem}

\begin{proof}
    Seien $p$ und $q$ die kritischen Werte mit $f(p) = a < b = f(q)$. Mit dem 
    Morse Lemma existiert dann $\varepsilon > 0$, sodass 
    $f^{-1}[a, a + \varepsilon]$ und $f^{-1}[b - \varepsilon, b]$ diffeomorph
    zu einer $n$-Zelle $e^n$ sind. 
    Mit dem ersten Deformationslemma~\ref{theorem:erstes deformationslemma} ist
    $f^{-1}[a - \varepsilon, b - \varepsilon]$ deffeomorph zu 
    $\del e^n \times [0, 1]$. Dann ist $M$ der Zylinder $\del e^n \times [0, 1]$ 
    mit $e^n$ an beiden seiten des Zylinders angebracht. Sei 
    $\operatorname{pr}_1: S^n \to \R$ die Projektion auf die erste Koodinate von
    $S^n \subset \R^{n + 1}$. Dann hat $\operatorname{pr}_1$ genau zwei kritische 
    Punkte, beide sind nicht degeneriert, also gilt $M \cong S^n$.
\end{proof}

\begin{theorem}
    \label{theorem:CW-komplex}
    Sei $M$ eine Mannigfaltigkeit und $f: M \to \R$ eine glatte Funktion mit 
    ausschließlich nicht degenerierten kritischen Punkten $p_1, ..., p_m$. 
    Dann besitzt $M$ den Homotopie-Typen von einem CW-Komplex mit einer $k$-Zelle 
    für jeden kritischen Punkt mit Index $k$.
\end{theorem}

\begin{proof}
    Der Beweis ist recht involviert, aber im wesentlichen folgt die Aussage aus
    dem zweiten Deformationslemma~\ref{theorem:zweites deformationslemma}.
\end{proof}

\begin{theorem}[Morse Ungleichungen]
    Sei $M$ eine kompakte Mannigfaltigkeit, $f: M \to \R$ eine glatte Funktion
    mit ausschließlich nicht degenerierten kritischen Punkten. Sei $C_k$ die
    Anzahl der kritischen Punkte mit Index $k$. Sei $b_k(M)$ die $k$-te 
    Betti-Zahl von $M$ und $\chi(M)$ die Euler-Charakteristik von $M$. 
    Dann gelten:
    \begin{enumerate}
        \item $ b_k(M) \leq C_k $ 
        \item $ \chi(M) = \sum_k (-1)^k \cdot C_k $
        \item $ b_k(M) - b_{k - 1}(M) + ... \pm b_0(M) 
            \leq C_k - C_{k-1} + ... \pm C_0 $
    \end{enumerate}
\end{theorem}

\begin{proof}
    Auch folgt die Behauptung wesentlich aus dem zweiten 
    Deformationslemma~\ref{theorem:zweites deformationslemma}. 
\end{proof}
