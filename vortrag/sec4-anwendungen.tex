\section{Anwendungen}

\begin{theorem}
    \label{theorem:Sn}
    Sei $M$ eine $n$-dimensionale Mannigfaltigkeit, sodass eine glatte Funktion
    $f: M \to \R$ existiert, die genau zwei kritische Punkte besitzt, die beide
    nicht degeneriert sind. Dann ist $M$ homeomorph zu $S^n$.
\end{theorem}

\begin{proof}
    Seien $p$ und $q$ die kritischen Werte mit $f(p) = a < b = f(q)$. Mit dem 
    Morse Lemma existiert dann $\varepsilon > 0$, sodass 
    $f^{-1}[a, a + \varepsilon]$ und $f^{-1}[b - \varepsilon, b]$ diffeomorph
    zu einer $n$-Zelle $e^n$ sind. 
    Mit dem ersten Deformationslemma~\ref{theorem:erstes deformationslemma} ist
    $f^{-1}[a - \varepsilon, b - \varepsilon]$ deffeomorph zu 
    $\del e^n \times [0, 1]$. Dann ist $M$ der Zylinder $\del e^n \times [0, 1]$ 
    mit $e^n$ an beiden seiten des Zylinders angebracht. Sei 
    $\operatorname{pr}_1: S^n \to \R$ die Projektion auf die erste Koodinate von
    $S^n \subset \R^{n + 1}$. Dann hat $\operatorname{pr}_1$ genau zwei kritische 
    Punkte, beide sind nicht degeneriert, also gilt $M \cong S^n$.
\end{proof}

\begin{theorem}
    \label{theorem:CW-komplex}
    Sei $M$ eine Mannigfaltigkeit und $f: M \to \R$ eine glatte Funktion mit 
    ausschließlich nicht degenerierten kritischen Punkten $p_1, ..., p_m$. 
    Dann besitzt $M$ den Homotopie-Typen von einem CW-Komplex mit einer $k$-Zelle 
    für jeden kritischen Punkt mit Index $k$.
\end{theorem}

\begin{proof}
    Der Beweis ist recht involviert, aber im wesentlichen folgt die Aussage aus
    dem zweiten Deformationslemma~\ref{theorem:zweites deformationslemma}.
\end{proof}

\begin{theorem}[Morse Ungleichungen]
    Sei $M$ eine kompakte Mannigfaltigkeit, $f: M \to \R$ eine glatte Funktion
    mit ausschließlich nicht degenerierten kritischen Punkten. Sei $C_k$ die
    Anzahl der kritischen Punkte mit Index $k$. Sei $\betti_k(M)$ die $k$-te 
    Betti-Zahl von $M$ und $\chi(M)$ die Euler-Charakteristik von $M$. 
    Dann gelten:
    \begin{enumerate}
        \item $ \betti_k(M) \leq C_k $ 
        \item $ \chi(M) = \sum_k (-1)^k \cdot C_k $
        \item $ \betti_k(M) - b_{k - 1}(M) + ... \pm \betti_0(M) 
            \leq C_k - C_{k-1} + ... \pm C_0 $
    \end{enumerate}
\end{theorem}

\begin{proof}
    Auch hierfolgt die Behauptung wesentlich aus dem zweiten 
    Deformationslemma~\ref{theorem:zweites deformationslemma}:
    Wir definieren $\betti_k$ und $\chi$ für ein Raumpaar $(X, Y)$:
    \[ \betti_k(X, Y) = \dim (\Hom_k (X, Y)) \]
    und 
    \[ \chi(X, Y) = \sum_k (-1)^k \betti_k(X, Y) \]
    Man kann zeigen, dass $\betti_k$ subadditiv und $\chi$ addutiv sind, also
    dass für $Z \subseteq Y \subseteq X$ gilt
    \[ \betti_k(X, Z) \leq \betti_k(X, Y) + \betti_k(Y, Z) \]
    und 
    \[ \chi(X, Z) = \chi(X, Y) + \chi(Y, Z) \]
    Außerdem zeigt man, dass für $X_0 \subseteq ... \subseteq X_n$ gilt
    \[ S(X_n, X_0) \leq \sum_i S(X_i, X_{i-1}) \]
    falls $S$ subadditiv, und dass falls $S$ additiv ist sogar Gleichheit gilt.
    
    ObdA hat $f$ nur isolierte kritische Punkte. Seien dann 
    $a_0, ..., a_m \in \R$ so gewählt, dass $M^{a_i}$ genau $i$ kritische Punkte
    enthalten und $M^{a_m} = M$. Dann gilt schon $M^0 = \varnothing$. Sei nun $k_i$
    der Index des kritischen Punktes in $M^{a_i} - M^{a_{i-1}}$. Dann gilt:
    \begin{align*}
        \Hom_k(M^{a_i}, M^{a_{i-1}}) 
            & = \Hom_k(M^{a_{i-1}} \cup e^{k_i}, M^{a_{i-1}})
            & \text{ (wegen des zweiten Deformationslemmas) } \\
        & = \Hom_k(e^{k_i}, \del e^{k_i})
            & \text{ (wegen der Ausschneidungseigenschaft) } \\
        & = \Hom_k(S^{k_i}, \ast)
    \end{align*}

    Also haben wir
    \[ 
        H_k(M^{a_i}, M^{a_{i-1}}) = \begin{cases}
            \R &\text{ falls } k = k_i \\
            0 & \text{ sonst }
        \end{cases}
    \]
    Da $\betti_k$ subadditiv ist gilt 
    \begin{align*}
        \betti_k(M) 
           & \leq \sum_i \betti_k(M^{a_i}, M^{a_{i-1}}) \\
           & = \sum_i \dim(\Hom_k(M^{a_i}, M^{a_{i-1}})) \\
           & = C_k
    \end{align*}
    
    Und da $\chi$ additiv ist gilt
    \begin{align*}
        \chi(M) 
           & = \sum_i \chi (M^{a_i}, M^{a_{i-1}}) \\
           & = \sum_i \sum_k (-1)^k \dim \Hom_k (M^{a_i}, M^{a_{i-1}}) \\
           & = \sum_k (-1)^k C_k
    \end{align*}

    3. folgt aus der Tatsache, dass $S_k$ subadditiv ist, wobei 
    \[ 
      S_k(X, Y) 
      = \betti_k(X, Y) - \betti_{k - 1}(X, Y) + ... \pm \betti_{0} 
    \]

    Dann gilt wieder
    \begin{align*} 
        S_k (M) 
            & \leq \sum_{i = 1}^m (-1)^i S_k(M^{a_i}, M^{a_{i-1}}) \\
            & = \sum_{i = 1}^m \betti_k(M^{a_i}, M^{a_{i-1}}) 
            - \betti_{k - 1}(M^{a_i}, M^{a_{i-1}})
            + ... \pm \betti_0(M^{a_i}, M^{a_{i-1}})
    \end{align*}
    aber gerade haben wir schon gesehen, dass gilt
    \[ 
        \betti_k(M^{a_i}, M^{a^{i-1}}) = 
            \begin{cases}
                1 & \text{ if } k = i \\
                0 & \text{ else }
            \end{cases}
    \]
    Also gilt
    \[ 
        \sum_{i = 1}^m \betti_k(M^{a_i}, M^{a_{i-1}}) 
            - \betti_{k - 1}(M^{a_i}, M^{a_{i-1}})
            + ... \pm \betti_0(M^{a_i}, M^{a_{i-1}})
        = C_k - C_{k - 1} + ... \pm C_0
    \]

\end{proof}
