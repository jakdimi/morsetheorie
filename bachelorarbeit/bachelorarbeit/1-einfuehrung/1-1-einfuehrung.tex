\section{Einführung}

\subsection*{Notation}

Folgende Notationen werden benutzt:

\begin{itemize}
    \item $\isom$ für eine Vektorraum-Isomorphie oder eine Homeomorphie.
    \item $\diffeo$ für einen Diffeomorphismus.
    \item Die Tangentialräume werden immer als die Menge der Derivationen betrachtet.
    \item Um Subskripte in Subskripten weitestgehend zu vermeiden, ist ein Vektorfeld eine 
        glatte Abbildung $X \colon M \to TM$ mit $p \mapsto X(p) \in T_pM$ (anstatt $X_p$).
    \item ist $f \colon M \to \N$ eine glatte Abbiildung, dann ist die Tangentialabbildung
        von $f$ im Punkt $p$ die Abbildung $\opd f (p) \colon T_pM \to T_{f(p)}N$.
\end{itemize}

\subsection*{Vorraussetzungen}

Wir setzen vorraus, dass der Leser sich mit Mannigfaltigkeiten auskennt. Außerdem sollten 
grundlegende Begriffe der Topologie wie \textit{Homotopie}, \textit{Homotopietyp} ... \todo{...}
bekannt sein. Weiterhin sollten Grundlagen der homologischen Algebra bekannt sein, also 
Definitionen von \textit{Kettenkomplex} und \textit{Homologie}.

\subsection*{Geschichte}

Die Morse-Theorie ist zu Ehren des amerikanischen Mathematikers \textit{Marston Morse} benannt, der
grundlegende Ergebnisse zum Beispiel wie das \textit{Morse-Lemma}~\ref{satz: morse-lemma} in 
\cite{morse} schon im Jahre 1929 bewies. (Marston Morse nannte benutzte den Begriff 
\glqq Critical Point theory\grqq{})In den späten 60er Jahren untersuchte dann 
bewies.
