\section{Nicht-Degeneriertheit und Index}

Dieser Abschnitt folgt dem ersten Abschnitt aus \cite{milnor}.

\begin{definition}[Kritischer Punkt]
    \label{def: kritischer Punkt}
    Sei $M$ eine glatte Mannigfaltigkeit und $f \colon M \to \R$ eine glatte Abbildung. 
    Ein \textit{kritischer Punkt} von $f$ ist ein Punkt $p \in M$, sodass $\opd f (p) = 0$.
\end{definition}

\begin{remark}
    Allgemeiner lassen sich kritische Punkte von glatten Abbildungen \\ $f \colon M \to N$ 
    definieren, siehe im Anhang Definition~\ref{anh.def: kritischer punkt}.
\end{remark}

Wir würden gerne eine Hessische Bilinearform für die Tangentialräume der Mannigfaltigkeit
definieren, allerdings ist dies ein nicht ganz einfaches Unterfangen. Wir werden am Ende
einen Begriff erhalten, der mit dem der gewohnten Hessischen Bilinearform im $\R^n$
übereinstimmt, allerdings nur für kritische Werte definiert ist.

\begin{definition}[Lie-Klammer]
    \label{def: lie-klammer}
    Es seien $X$ und $Y$ Vektorfelder auf einer glatten Mannigfaltigkeit $M$. Die 
    \textit{Lie-Klammer} ist die Abbildung 
    \begin{align*} 
        [\cdot, \cdot] \colon \VFs(M) \times \VFs(M) & \to \VFs(M) \\
        (X, Y) & \mapsto [X, Y] := XY - YX
    \end{align*}
    Wobei 
    \[ (XY - YX) (p) (f) = X(p)(Y(\cdot)(f)) - Y(p)(X(\cdot)(f)) \]
\end{definition}

\begin{remark}
    Es ist leicht nachzurechnen, dass die Lie-Klammer tatsächlich eine \\ Lie-Klammer ist,
    also dass sie folgende Eigenschaften erfüllt:
    \begin{itemize}
        \item $[\cdot, \cdot]$ ist $\R$-bilinear.
        \item $[X, Y] = -[Y, X]$
        \item $[X, [Y, Z]] + [Z, [X, Y]] + [Y, [Z, X]] = 0$
    \end{itemize}
\end{remark}

\begin{prop}
    \label{prop: lie-klammer ist null}
    Es sei $f \colon M \to \R$ glatt, $p$ ein kritischer Punkt von $f$, 
    $X, Y \in \VFs (M)$. Dann gilt:
    \[ [X, Y] (p) f = 0 \]
\end{prop}

\begin{proof}
    Es seien $(x_1, ..., x_n)$ lokale Koordinaten um $p$. Wir können ohne beschränkung
    der Allgemeinheit annehmen, dass $X = g_X \cdot \pderive{x_i}$ und 
    $Y = g_Y \cdot \pderive{x_j}$ für $g_X, g_Y \in C^{\infty} (M)$. Dann gilt:
    \begin{align*}
        \left[g_X \cdot \pderive{x_i}, g_Y \cdot \pderive{x_j}\right] (p) (f) = & 
            g_X (p) \cdot \pderive{x_i} (p) \left(g_Y \cdot \pderive[f]{x_j} \right) -
            g_Y (p) \cdot \pderive{x_j} (p) \left(g_X \cdot \pderive[f]{x_i} \right) \\
        = & g_X (p) \cdot \left( \pderive[g_Y]{x_i}(p) \cdot \pderive[f]{x_j}(p) + 
                g_Y(p) \cdot \pdderive[f]{x_j}{x_i}(p) \right) \\ 
        & - g_Y (p) \cdot \left( \pderive[g_X]{x_j}(p) \cdot \pderive[f]{x_i}(p) + 
            g_X (p) \cdot \pdderive[f]{x_i}{x_j}(p) \right) \\
        = & \; 0
    \end{align*}
    Der letzte Ausdruck ist Null wegen des Satzes von Schwarz und da $p$ ein kritischer
    Punkt von $f$ ist, also gilt $\pderive[f]{x_i}(p) = 0$.
\end{proof}

\begin{definition}[Hessische Bilinearform]
    Es sei $f \colon M \to \R$ eine glatte Abbildung, $p$ ein kritischer Punkt von $f$.
    Es seien $x, y \in T_pM$. Wähle $X, Y \in \VFs (M)$, sodass $X(p) = x$ und 
    $Y(p) = y$. Definiere nun
    \[ \opd^2 f (x, y) (p) = X(p)(Y(\cdot)f). \]
    $\opd^2 f (\cdot, \cdot) (p)$ heißt \textit{Hessische Bilinearform}. 
\end{definition}

\begin{prop}
    \label{prop: hessische ist sym bilinearform}
    $\opd^2 f (\cdot, \cdot) (p)$ hängt nicht von den gewählten Vektorfeldern $X$ und $Y$ ab 
    und ist für alle kritischen Punkte eine symmetrische Bilinearform.
\end{prop}

\begin{proof}
    Bilinearität folgt direkt aus der Definition.
    Da $p$ ein kritischer Punkt ist gilt 
    \[ \opd^2 f (x, y) (p) - \opd^2 f (y, x) (p) = [X, Y] (p) (f) = 0, \]
    die Zuordnung ist also symmetrisch. Außerdem gilt
    \[ XY f (p) = X(p) (Y(\cdot) f) = x(Y(\cdot) f), \]
    also hängt die Form nicht von $X$ ab, und wegen der Symmetrie auch nicht von $Y$.
\end{proof}

\begin{definition}[nicht-degeneriert, Index]
    \label{def: nicht-degeneriert u index}
    Es sei $f \colon M \to \R$ eine glatte Abbildung, $p$ ein kritischer Punkt von
    $f$. Wir nennen $p$ \textit{nicht degeneriert}, falls die Bilinearform 
    $\opd^2 f (\cdot, \cdot) (p)$ nicht ausgeartet ist. Der \textit{Index} eines
    nicht degenerierten kritischen Punktes ist die maximale Dimension eines
    Untervektorraumes, auf dem $\opd^2 f (\cdot, \cdot) (p)$ negativ definit ist.
\end{definition}

\begin{remark}
    Nicht-Degeneriertheit und Index lassen sich auch über lokale Koordinaten definieren,
    aber nachzurechnen, dass diese Begriffe wohldefiniert sind ist recht aufwändig.
    Trotzdem wollen wir diese Sichtweise nicht vorenthaltern:

    Es seien $\phi = (x_1, ..., x_n)$ lokale Koordinaten um den kritischen Punkt $p$. 
    Dann ist $\mathcal{B} = \left(\pderive{x_1}, ..., \pderive{x_n}\right)$ eine Basis des
    Vektorraums $T_pM$. Wir bekommen
    \[ 
        \opd^2 f \left( \pderive{x_i}, \pderive{x_j} \right) (p) 
        = \pderive{x_i}(p) \left( \pderive[f]{x_j} \right) 
        = \pdderive[f]{x_j}{x_i} (p).
    \]
    Dann ist $p$ nicht degeneriert genau dann wenn die Matrix
    \[ H^\phi_p(f) = \left( \pdderive[f]{x_i}{x_j} \right)_{1 \leq i, j \leq n} \]
    invertierbar ist. Der Index von $p$ ist dann die Anzahl der negativen Eigenwerte
    von $H^\phi_p(f)$. Der Index und die nicht-degeneriertheit hängen offensichtlich
    nicht von den gewählten Koordinaten ab, aber die Matrix $H_p^{\phi}(f)$ schon.
\end{remark}

Die Hessische Bilinearform lässt sich auch mithilfe von \textit{Zusammenhängen} für 
alle Punkte von $M$ definieren.

\begin{remark}
    Die beiden Begriffe Index und nicht-Degeneriertheit sind zentral in der Morse-Theorie 
    und werden uns über die gesamte Arbeit begleiten. Auch der nachfolgende Satz wird in 
    fast jedem Beweis genutzt:
\end{remark}

\begin{theorem}[Morse-Lemma]
    \label{satz: morse-lemma}
    Es sei $p$ ein nicht degenerierter kritischer Punkt mit Index $k$ einer glatten 
    Funktion $f \colon M \to \R$. Dann existieren lokale koordinaten 
    $\phi = (x_1, ..., x_n)$, sodass in einer Umgebung $U$ von $p$ gilt:
    \[ f = f(p) - x_1^2 - ... - x_k^2 + x_{k + 1}^2 + ... + x_n^2 \]
    und 
    \[ \varphi (p) = 0. \]
    $(U, \phi)$ heißt \textit{Morse-Karte}, und $U$ \textit{Morse-Umgebung}.
\end{theorem}

Der hier geführte Beweis für das Morse-Lemma ist in \cite{hirsch} zu finden. 
Bevor wir das Morse Lemma beweisen, benötigen wir eine Aussage aus der Linearen Algebra:

\begin{lemma}
    \label{lemma: lina lemma}
    Es sei $A = \diag(a_1, ..., a_n)$ eine diagonale $n \times n$ Matrix mit 
    Diagonaleinträgen $\pm 1$. Dann gibt es eine Umgebung $N$ von $A$ im Vektorraum der 
    symmetrischen $n \times n$ Matrizen und eine glatte Abbildung 
    $P \colon N \to GL_n(\R)$, sodass $P(A) = E_n$ und falls $P(B) = Q$, dann gilt 
    $Q^TBQ = A$.
\end{lemma}

\begin{proof}
    Betrachte zuerst den Fall $n = 1$:

    Dann ist $A = (\pm 1)$. Wähle $N = (0, 2)$ oder $N = (-2, 0)$, $P(B) := 1/\sqrt{|B|}$

    Nun $n - 1 \rightsquigarrow n$:

    Es sei $B$ eine symmetrische $n \times n$ Matrix, die nah genug an $A$ ist, sodass
    $b_{11} \neq 0$ und das selbe Vorzeichen hat wie $a_1$. Betrachte die Matrix 
    \[ T = \begin{pmatrix}
        \frac{1}{\sqrt{|b_{11}|}} & 
            - \frac{1}{\sqrt{|b_{11}|}} \cdot \frac{b_{12}}{b_{11}} & 
            - \frac{1}{\sqrt{|b_{11}|}} \cdot \frac{b_{13}}{b_{11}} & \cdots & 
            - \frac{1}{\sqrt{|b_{11}|}} \cdot \frac{b_{1n}}{b_{11}} \\
        0 & 1 & 0 & \cdots & 0 \\
        0 & 0 & 1 & \cdots & 0 \\
        \vdots & \vdots & \vdots & & \vdots \\
        0 & 0 & 0 & \cdots & 1 
    \end{pmatrix} \]
    Man rechnet nach, dass 
    \[ T^T B T = \begin{pmatrix}
        a_1 & 0 & \cdots & 0 \\
        0 & & & \\
        \vdots & & B_1 & \\
        0 & & &
    \end{pmatrix}. \]
    Die Diagonalmatrix $\diag(a_2, ..., a_n)$ ist invertierbar, und da die 
    Determinante stetig ist, ist falls $B$ nah genug an $A$ ist die symmetrische Matrix 
    $B_1$ auch invertierbar. Bemerke dass sowohl $T$ als auch $B_1$ glatte Abbildungen
    definieren. Laut Induiktionsannahme existiert eine Matrix 
    $Q_1 \in GL_n(\R)$ die glatt von $B_1$ abhängt, sodass $Q_1^T B_1 Q_1 = A_1$.
    Definiere nun $P(B) = Q$ durch $Q = TS$, wobei
    \[ S = \begin{pmatrix}
        1 & 0 & \cdots & 0 \\
        0 & & & \\
        \vdots & & Q_1 & \\
        0 & & &
    \end{pmatrix}. \]
    Dann gilt $Q^TBQ = S^T (T^T B T) S = A$.
\end{proof}

\begin{proof}[Beweis von Satz~\ref{satz: morse-lemma}]
    Es sei $U$ eine Karten Umgebung von $p$. Dann können wir ohne Beschränkung der 
    Allgemeinheit annehmen, dass $f \colon \R^n \to \R$, $p = 0$ und $f(0) = 0$.
    Außerdem können wir mithilfe eines Korrdinatenwechsels annehmen, dass 
    \[ A = H_0(f) \]
    eine Diagonalmatrix mit ausschließlich Diagonaleinträgen $\pm 1$ hat, denn da $p$
    nicht degeneriert ist ist $A$ invertierbar. 
    \begin{claim*}
        Es existiert eine glatte Abbildung $x \mapsto B_x$ von $M$ in die symmetrischen
        $n \times n$ Matrizen, sodass für $B_x = (b_{ij}(x))_{ij}$ gilt 
        \[ f(x) = \sum_{i, j = 1}^n b_{ij}(x) x_i x_j , \]
        und sodass $B_0 = A$. 
    \end{claim*}
    \begin{smallproof}
        Da $f(0) = 0$ bekommen wir mit dem Fundamentalsatz der 
        Differenzial - und Integralrechnung: 
        \begin{align*}
            f(x) = & f(x) - f(0) \\
            = & \int_0^1 \derive[f (tx)]{t} \opd t \\
            = & \int_0^1 \sum_{i = 1}^n \pderive[f]{x_i}(tx) x_i \opd t \\
            = & \sum_{i = 1}^n \left( \int_0^1 \pderive[f]{x_i}(tx) \opd t \right) x_i
        \end{align*}
        Da $p = 0$ ein kritischer Punkt ist, gilt $\pderive[f]{x_i}(0) = 0$ für alle 
        $i$. Mit dem selben Argument sehen wir dann, dass
        \[ \pderive[f]{x_i}(tx) = 
        \sum_{j = 1}^n \left( \int_0^1 \pdderive[f]{x_i}{x_j}(stx) \opd s \right) x_j . \]
        Dann gilt 
        \[ f(x) = 
            \sum_{i, j = 1}^n 
            \left( \int_0^1 \int_0^1 \pdderive[f]{x_i}{x_j} \opd s \opd t \right) 
            x_i x_j 
        . \]
        Setze also 
        \[ b_{ij}(x) = \int_0^1 \int_0^1 \pdderive[f]{x_i}{x_j} \opd s \opd t . \]
        Dann gilt schon $B_0  = A$, und die Abbilfungen $b_{ij}$ sind glatt, also 
        auch $x \mapsto B_ x$.
    \end{smallproof}
    Wir dürfen nun das vorherige Lemma~\ref{lemma: lina lemma} anwenden:

    Sei $P \colon N \to GL_n(\R)$ eine Abbildung wie in~\ref{lemma: lina lemma}.
    Setze $P(B_x) := Q_x$ Definiere nun eine glatte Abbildung $\phi \colon U \to \R^n$
    durch $\phi (x) = Q_x^{-1}x$ in einer Umgebung von $0$. Wir rechnen nach, dass 
    $\opd \phi (0) \colon \R^n \to \R^n$ die Identität ist:

    Schreibe $Q_x^{-1} = (q_{ij}(x))_{ij}$. Dann
    \[ \phi(x) = \left( 
        \sum_{k = 1}^n q_{1k}(x) x_k, \cdots , \sum_{k = 1}^n q_{nk}(x) x_k
        \right)
    \]
    Also 
    \begin{align*}
        \pderive[\phi_i]{x_j} (x) 
            = & \pderive{x_j} \left( \sum_{k = 1}^n q_{ik} (x) x_k \right) \\
        = & \sum_{k = 1}^n \left( 
            \pderive[q_{ik}]{x_j}(x) x_k + q_{ik}(x) \delta_{ki}
        \right)
    , \end{align*}
    Wobei $\delta_{ki}$ das Kronecker Delta ist. Setzen wir also $0$  in $\phi$ ein
    bekommen wir
    \[ \pderive[\phi_i]{x_j}(0) = q_{ij}(0). \]
    Das Differential von $\phi$ in $0$ ist also gegeben durch
    \[ Q_0^{-1} = P(B_0)^{-1} = P(A)^{-1} = E_n . \]
    Das differential an der Stelle $0$ ist also invertierbar, und dann können wir mit dem 
    Satz über die Umkehrfunktion annehmen, dass $U$ klein genug ist, sodass $\phi$ 
    eingeschräkt aufs Bild ein Diffeomorphismus ist. 
    Dann ist $\phi$ eine Karte um $0$. Setze $(y_1, ..., y_n) := \phi$, dann gilt 
    \begin{align*}
        f(x) = & x^T B_x x \\
        = & (Q_x \phi(x))^T B_x (Q_x \phi(x)) \\
        = & \phi(x)^T (Q_x^T B_x Q_x) \phi(x) \\
        = & \phi(x)^T A \phi(x) \\
        = & \sum_{i = 1}^n a_{ii} y_i(x)^2
    . \end{align*}
    Das entspricht genau der gewünschten Form.
\end{proof}

\begin{corollary}
    Nicht-degenerierte kritische Punkte sind isoliert.
\end{corollary}
