\section{Die stabile- und instabile Mannigfaltigkeit und die Smale-Bedingung}

\begin{definition}[Stabile- und instabile Mannigfaltigkeit]
    \label{def: stabile und instabile mannigfaltigkeit}
    Es sei $f \colon M \to \R$ eine Morse-Funktion, $p$ ein kritischer Punkt von $f$ und $X$ ein
    Pseudo-Gradientenfeld von $f$. Die stabile Mannigfaltigkeit von $p$ ist die Menge
    \[ \stab (p) = \left\{ q \in M : \lim_{t \to + \infty} \phi_t(q) = p \right\} \]
    und die instabile Mannigfltigkeit ist
    \[ \unst (p) = \left\{ q \in M : \lim_{t \to - \infty} \phi_t(q) = p \right\} . \]
    (\say{s} wie \say{stable}, \say{u} wie \say{unstable})
\end{definition}

Bevor wir die stabile- und instabile Mannigfaltigkeit eines kritischen Punktes weiter untersuchen, 
fixieren wir ein Paar Notationen zu Morse Umgebungen. Für den Rest dieser Arbeit ist diese 
Vorstellung von Morse-Umgebungen wichtig.

\begin{definition}[Notationen zu Morse Umgebungen]
    \label{def: notation morse umgebung}
    Zuerst untersuchen wir eine quadratische Formin $\R^n$, die die Form hat wie Funktionen in Morse
    Umgebungen, also $Q \colon \R^n \to \R$ mit
    \[ Q(x_1, \dots, x_n) = - x_1^2 - \dots - x_k^2 + x_{k + 1}^2 + \dots + x_n^2 \]
    für ein $1 \leq k \leq n$.
    Mit $x_- := (x_1, \dots x_k)$ und $x_+ := (x_{k + 1} \dots x_n)$ gilt dann
    \[ Q = - \| x_- \|^2 + \| x_+ \|^2 . \]
    Der Gradient von $Q$ ist mit dem Standardskalarprodukt auf $\R^n$
    \[ \grad Q (x_-, x_+) = 2(x_-, x_+) . \]
    Es seien $\eps, \eta > 0$. Dann definiere die Menge
    \[ U(\eps, \eta) := \left\{ x \in \R^n : - \eps \leq Q(x) \leq \eps 
    \text{ und } \| x_- \|^2 \| x_+ \|^2 \leq \eta(\eps + \eta) \right\} := U . \]
    Wir definieren außerdem
    \begin{align*}
        \del_{\pm} U := & \left\{ x \in U: Q(x) = \pm \eps \text{ und } \|x_{\mp} \|^2 \leq \eta \right\} 
            \text{ und} \\
        \del_0 U := & \left\{ x \in \del U: \| x_- \|^2 \| x_+ \|^2 = \eta(\eps + \eta) \right\} .
    \end{align*}
    Dann setzt sich der Rand von $U$ aus diesen drei Teilen zusammen, also
    \[ \del U = \del_+ U \cup \del_- U \cup \del_0 U . \]
    $\del_0 U$ ist parallel zu den Trajektorien des negativen Gradienten von $Q$. $\del_+ U$ und 
    $\del_- U$ sind orthogonal zu den Trajektorien des negativen Gradienten von $Q$, wobei die 
    Trajektorien in $\del_+ U$ in die Menge $U$ eintreten und sie in $\del_- U$ wieder verlassen. 
    Wir setzen nun $V_- = \langle e_1, \dots, e_k \rangle$ und 
    $V_+ = \langle e_1, \dots, e_n \rangle \subseteq \R^n$. $V_+$ ist der größte Vektorraum, 
    auf dem $\opd^2 Q (\cdot, \cdot) (0)$ positiv definit ist und $V_-$ der größte Vektorraum, auf dem 
    $\opd^2 Q (0) (\cdot, \cdot) (0)$ negativ definit ist. 
    Es gilt 
    \[ \del U \cap V_{\pm} \subseteq \del_{\pm} U . \]
    $0$ ist der einzige kritische Punkt von $Q$ und ist offensichtlich nicht degeneriert. Damit ist $Q$
    eine Morse Funktion und es gilt 
    $\stab (0) = V_+$ und $\unst (0) = V_-$.

    Ist nun $f \colon M \to \R$ eine Morse Funktion, $p$ ein kritischer Punkt von $f$ und $(V, \psi)$
    eine Morse Umgebung von $p$, dann gilt $f \circ \psi^{-1} = Q + f(p)$. Sind $\eps$ und $\eta$
    klein genug, dann ist $U \subset \psi(V)$. Wir nennen 
    $\Omega (p, \eps, \eta) := \Omega (p) := \psi^{-1}(U)$, 
    $\del_{\pm} \Omega (p) := \psi^{-1}(\del_{\pm}U)$ und $\del_0 \Omega (p) := \psi^{-1} (\del_0 U)$.
    Dann ist 
    \[ \psi(\stab (p) \cap \Omega (p)) = V^+ \cap U \] 
    und 
    \[ \psi(\unst (p) \cap \Omega (p)) = V^- \cap U . \]
    Wir können uns also mit dieser Notation die stabile- und instabile Mannigfaltigkeit (wenigstens 
    in einer Umgebung von $p$) sehr gut vorstellen.

    Hier einige Beispiele für diese Morse Umgebungen:

    \begin{tikzpicture}
        % Include the image in a node
        \node [
            above right,
            inner sep=0] (image) at (0,0) {\includegraphics[width=\textwidth]{../resources/morse-umgebungen.jpeg}};
            
        % Create scope with normalized axes
        \begin{scope}[
        x={($0.02*(image.south east)$)},
        y={($0.05*(image.north west)$)}]

        \node at (1, 10.7) {$ \del_+ U $};
        \node at (16.9, 10) {$ \del_- U $};
        \node[text=LightBlue] at (29, 10) {$ \del_0 U $};
        \node at (32, 10) {$ \del_+ U $};
        \node at (48.2, 10.3) {$ \del_- U $};
        \node[text=LightBlue] at (48.2, 15.5) {$ \del_0 U $};
        \node at (46, 16.5) {$ \del_+ U $};
            
        \end{scope}
    \end{tikzpicture}

    Links eine Morse Umgebung um ein Minimum, also einen kritischen Punkt mit Index $0$ einer 
    Morse-Funktion aus einer $2$-dimenisionalen Mannigfaltigkeit, in der Mitte eine Morse Umgebung 
    eines Sattelpunktes, also eines kritischen Punktes mit Index $2$, und rechts eine Morse-Umgebung 
    eines kritischen Punktes mit Index $1$ einer Morse Funktion aus einer $3$-dimensionalen 
    Mannigfaltigkeit.

    Links ist die stabile Mannigfaltigkeit einfach eine Ungebung des Kreises, in der Mitte ist die 
    stabile Mannigfaltigkeit die vertikale Achse und die instabile die horizontale, und rechts
    ist die stabile Mannigfaltigkeit die vertikale Achse und die instabile ist eine Umgebung 
    eines Kreises auf der horizontalen Ebene.
    
    $\eps$ und $\eta$ werden häufig in der Notation ausgelassen.
\end{definition}

Mit dieser Vorstellungen von Morse-Funktinoen können wir die folgende Aussage beweisen.

\begin{prop}
    Ist $f \colon M \to \R$ eine Morse Funktion und $p$ ein kritischer Punkt von $f$, dann sind
    $\stab (p)$ und $\unst (p)$ Mannigfaltigkeiten mit 
    \[ \dim \unst (p) = n - \dim \stab (p) = \Index (p) \]
\end{prop}

\begin{proof}
    Es sei $(\psi, V)$ eine Morse Karte um $p$ und $\Omega(p) \subseteq V$ in einer Form wie 
    in ~\ref{def: notation morse umgebung}. Es sei außerdem $\phi$ der Fluss eines 
    Pseudo-Gradientenfeldes von $f$. Dann ist 
    \[ \Phi \colon \del_+ \Omega (p) \cap \stab (p) \times \R \to M ; \; \Phi (q, t) = \phi_t(q) \]
    eine Einbettung und es gilt 
    \[ \stab (p) = \Ima \Phi \cup \psi^{-1}(U \cap V_+) . \]
    Tatsächlich ist 
    \[ \stab (p) - \Ima \Phi = \{ p \} , \]
    denn $\lim_{t \to \infty} \phi_t(q) = p$ für $q \in \del_+ \Omega (p) \cap \stab (p)$. 
    Außerdem ist 
    \[ \del_+ U \cap V_+ = \{ x \in \R^n: \| x_+ \|^2 = \eps \} \diffeo S^{n - k - 1} , \] 
    denn für alle $x \in V_+$ gilt sowieso schon $x_- = 0$. Also ist $\stab(p)$ diffeomorph zum Raum 
    $S^{n - k - 1} \times (-\infty, \infty]/\sim$, in dem alle Punkte in $\infty$ zusammengeklebt 
    werden. Dieser Quotient ist wiederum diffeomorph zur offenen Kreisscheibe mit Dimension $n - k$.
    Genauso zeigt man, dass $\unst (p)$ diffeomorph zur offenen Kreisscheibe mit Dimension $k$ ist.
\end{proof}

\begin{prop}
    \label{prop: trajektorien enden in kritischen punkten}
    Es sei $f \colon M \to \R$ eine Morse-Funktion und $X$ ein Pseudo-\\Gradientenfeld von $f$. 
    Sei außerdem $M$ kompakt. Ist dann $\phi$ der Fluss von $X$, dann existieren für jeden Punkt 
    $p \in M$ kritische Punkte $q$ und $r$ von $f$, sodass
    \[ \lim_{t \to + \infty} \phi_t(p) = q \;\;\; 
    \text{ und } \;\;\; \lim_{t \to -\infty} \phi_t(p) = r \]
\end{prop}

\begin{proof}
    Wir zeigen die erste Aussage. Seien für jeden kritischen Punkt $p$ $(U_p, \psi_p)$ Morse Karten.
    Es ist $\lim_{t \to + \infty} \phi_t(q) = p$, genau dann wenn
    der Fluss $\phi_{\bullet}(q)$ den Punkt $q$ irgendwann in die Umgebung 
    $\del_+ \Omega (p) \cap \stab(p)$ transportiert. Angenommen $\phi_{\bullet}(q)$ transportiert
    $q$ nie zu einem kritischen Punkt. Jedes mal wenn $\phi_{\bullet}(q)$ also ins Innere einer
    Morse-Umgebung $U_q$ gerät, muss diese Umgebung auch wieder verlassen werden. Da 
    $f \circ \phi_{\bullet}(q)$ monoton ist, kann nachdem $\phi_{\bullet}(q)$ die Morse-Umgebung 
    $\Omega (p)$ verlassen hat, nie wieder zu dieser zurückgekehrt werden.
    Sei also 
    \[ \Omega = \bigcup_{p \in \Crit(f)} \Omega (p) \]
    und $t_0$ der Zeitpunkt an dem $\phi_{\bullet} (q)$ die Umgebung $\Omega$ das letzte mal verlässt.
    Da $M - \Omega$ keine kritischen Punkte von $f$ enthält existiert ein $\eps_0 > 0$, sodass für 
    alle $x \in M - U$ gilt 
    \[ \opd f (x) ((X(x))) \leq - \eps_ . \]
    Wir rechnen also: Für jedes $t \geq t_0$ gilt
    \begin{align*}
        f(\varphi_t(q) - f(\varphi_{t_0}(q))) = & 
            \int_{t_0}^t \derive[f \circ \phi_{\bullet}(q)]{s} (s) \opd s \\
        = & \int_{t_0}^t \opd f (\phi_s(q)) (X(\phi_s(q))) \opd s \\
        \leq & - \eps_0 (t - t_0) . 
    \end{align*}
    Also für $t \to + \infty$ gilt $f(\phi_t(p)) \to - \infty$. Das kann aber nicht sein, denn da 
    $M$ kompakt ist und $\R$ Hausdorff, ist $f$ eigentlich, also ist $\Ima f$ kompakt. 
    \todo{stimmt das?} Also kann $\phi_{\bullet}(p)$ 
    nicht alle $U_q$ verlassen. aber dann ist 
    \[ \lim_{t \to + \infty} \phi_t(q) = p \]
    für einen kritischen Punkt $p$.
    Genauso zeigt man, dass $\lim_{t \to - \infty} \phi_t(p) = r$ für einen kritischen Punkt $r$.
\end{proof}

\begin{definition}[Smale-Bedingung]
    \label{def: smale-bedingung}
    Es sei $M$ eine Mannigfaltigkeit und $U$ und $V$ Untermannigfaltigkeiten von $M$. Wir sagen 
    $U$ und $V$ sind \textit{transversal} und schreiben $U \pitchfork V$, falls für alle Punkte 
    $p \in U \cap V$ gilt $T_pU + T_pV = T_pM$. Ein Vektorfeld $X \in \VFs (M)$ heißt 
    \textit{transversal} zur Untermannigfaltigkeit $U$, falls für alle $p$ in $U$ gilt 
    $\langle X(p) \rangle + T_pU = T_pM$.

    Sei nun $f \colon M \to \R$ eine Morse Funktion und $X$ ein Pseudo-Gradientenfeld von $f$. Dann 
    sagen wir, dass $X$ die \textit{Smale-Bedingung} erfüllt, falls für alle kritischen Punkte 
    $p$ und $q$ von $f$ gilt 
    \[ \stab (p) \pitchfork \unst (q) . \]
    Ein Paar $(f, X)$ aus einer Morse-Funktion $f$ und einem Pseudo-Gradientenfeld $X$, das die 
    Smale-Bedingung erfüllt, nennt man \textit{Morse-Smale Paar}.
\end{definition}

\begin{prop}
    \label{prop: schnitt von transversalen untermannigfaltigkeiten}
    Sind $U_1$ und $U_2$ Untermannigfaltigkeiten von einer $n$-\\
    dimensionalen Mannigfaltigkeit $M$ mit 
    Dimensionen $d_1$ und $d_2$, sodass 
    \[ U_1 \pitchfork U_2 , \]
    dann ist $U_1 \cap U_2$ eine Untermannigfaltigkeit von $M$ mit Dimension $d_1 + d_2 - n$.
\end{prop}

\begin{proof}
    Es sei $p \in U_1 \cap U_2$. Da $U_1$ und $U_2$ Untermannigfaltigkeiten sind existieren 
    Karten $(\phi_1, V_1)$ und $(\phi_2, V_2)$ von $M$, sodass 
    \[ \phi_1 = (\phi_1', \phi_1'') \colon V_1 \to 
        \Omega_1 \times \Omega_1' \subseteq \R^{d_1} \times \R^{n - d_1} \]
    mit $\phi_1(V_1 \cap U_1) = \Omega_1 \times \{ 0 \}$ und 
    \[ \phi_2 = (\phi_2', \phi_2'')\colon V_2 \to 
        \Omega_2 \times \Omega_2' \subseteq \R^{d_2} \times \R^{n - d_2} \]
    mit $\phi_2(V_2 \cap U_2) = \Omega_2 \times \{ 0 \}$. Definiere 
    \[ \phi = (\phi_1'', \phi_2'') \colon 
        M \supseteq V_1 \cap V_2 \to \R^{n - d_1} \times \R^{n - d_2}. \]
    Dann ist 
    $(\phi)^{-1}(0) = (U_1 \cap V_1) \cap (U_2 \cap V_2) = (U_1 \cap U_2) \cap (V_1 \cap V_2) := V$.
    Wir wollen nun den Satz über reguläre Werte anwenden, aber dafür müssen wir zeigen, dass 
    $\opd \phi (p)$ surjektiv ist. Bemerke, dass 
    $\opd \phi (p) = \opd (\phi''_1, \phi''_2) (p) = (\opd \phi''_1 (p), \opd \phi''_2 (p))$.
    Es sei $v \in T_p \R^{n - d_1}$ und $w \in T_p\R^{n - d_2}$. Da $\opd \phi''_1 (p)$ und 
    $\opd \phi''_2 (p)$ surjektiv sind, und da $U_1$ und $U_2$ transversal sind, 
    existieren $v_1' + v_2', w_1' + w_2' \in T_pU_1 + T_pU_2$, sodass 
    $\opd \phi''_1 (p) (v_1 + v_2) = \opd \phi''_1 (p) (v_2) = v$ und 
    $\opd \phi''_2 (p) (w_1 + w_2) = \opd \phi''_2 (p) (w_1) = w$.
    Die ersten Gleichheiten gelten, da $T_p U_1$ der Kern von $\opd \phi''_1 (p)$ und 
    $T_p U_2$ der Kern von $\opd \phi''_2 (p)$ sind. Dann gilt
    \[ \opd \phi (p) (w_1' + v_2')  = 
        ( \opd \phi''_1 (p) (v_2'), \opd \phi''_2 (p) (w_1')) = (v, w) . \]
    Wir können also den Satz über reguläre Werte anwenden, dann ist $V$ eine Untermannigfaltigkeit
    mit Dimension $n - ((n - d_1) + (n - d_2)) = d_1 + d_2 - n$. Dann ist auch $U_1 \cap U_2$
    eine Untermannigfaltigkeit von Dimension $d_1 + d_2 - n$.
\end{proof}

Dann folgt direkt:

\begin{prop}
    Es sei $f \colon M \to \R$ eine Morse Funktion und $p$ und $q$ kritische Punkte von $f$ mit 
    Index $k_1$ und $k_2$ respektive. Falls $X$ die Smale-Bedingung erfüllt ist
    \[ \mathcal{M} (p, q) := \unst (p) \cap \stab (q) = 
        \left\{ r \in M : \lim_{t \to - \infty} \phi_t(p) \text{ und } 
        \lim_{t \to + \infty} \phi_t(q) \right\} \]
    ist eine Mannigfaltigkeit mit Dimension $k_1 - k_2$.
\end{prop}

Der Raum $\mathcal{M} (p, q)$ beinhaltet alle Punkte, die auf Trajektorien zwischen den kritischen
Punkten $p$ und $q$ liegen. 

\todo{Zeichnung}

\begin{prop}
    \label{prop: wohldefiniertheit von Lt}
    $\R$ wirkt via $(p, t) \mapsto \phi_t(p)$ auf $\mathcal{M}(p, q)$. Sind $p$ und $q$ kritische 
    Punkte mit $p \neq q$ und Indizes $k_1$ und $k_2$, dann ist 
    \[ \Lt (p, q) = \mathcal{M} (p, q) / \R \]
    eine $k_1 - k_2 - 1$-dimensionale Mannigfaltigkeit und es gilt 
    \[ \Lt (p, q) \diffeo \mathcal{M} (p, q) \cap f^{-1}(c) \]
    für einen regulären Wert $c$ zwischen $f(p)$ und f(q)
    (Wir können ohne Beschränkung der Allgemeinheit annehmen, dass $f(p) \neq f(q)$).
\end{prop}

\begin{proof}
    Die Gruppenwirkung ist frei, denn in $\mathcal{M} (p, q)$
    sind keine kritischen Punkte, da $p \neq q$. Es sei $x$ in $\mathcal{M} (p, q)$. 
    Ist nun $t \neq 0$, dann gilt da $f \circ \phi_{\bullet}(x)$ streng monoton ist 
    $f(\phi_t(x)) \neq f(\phi_0(x)) = f(x)$, also $\phi_t(x) \neq x$.

    Es sei nun $c$ ein regulärer Wert zwischen $f(p)$ und $f(q)$. $f^{-1}(c)$ ist eine 
    $n - 1$-dimensionale Mannigfaltigkeit, und für das Pseudo-Gradientenfeld $X$ gilt 
    $X \pitchfork f^{-1}(c)$, und da für jeden Punkt $a$ gilt $X(a) \in T_a \mathcal{M} (p, q)$
    ist $f^{-1}(c) \cap \mathcal{M} (p, q)$ eine $k_1 - k_2 - 1$-dimensionale Mannigfaltigkeit.
    Sei 
    \[ \iota \colon \mathcal{M} (p, q) \cap f^{-1}(c) \to \mathcal{M}(p, q)\] 
    die Inklusion und 
    \[ p \colon \mathcal{M} (p, q) \to \Lt (p, q) \] 
    die Quotientenabbildung. $p \circ \iota$ ist bijektiv und stetig, und ist 
    $U \subseteq \mathcal{M} (p, q)$ offen, dann ist $p \circ \iota (U)$ offen in $\Lb (p, q)$. 
    Also ist $p \circ \iota$ ein Homeomorphismus und $\Lt (p, q)$ ist Hausdorff, also ist $\Lt (p, q)$
    eine Mannigfaltigkeit, und es gilt sogar
    \[ \Lt (p, q) \diffeo \mathcal{M} (p, q) \cap f^{-1}(c) \]
\end{proof}

Der Raum $\Lt (p, q)$ enthält für jede Trajektorie, die zwischen den kritischen Punkten $p$ und $q$ 
verläuft einen Repräsentanten. Später wird $\Lt (p, q)$ benutzt, um den Morse-Komplex zu definieren.
Die Smale-Bedingung ist also für unsere Zwecke wichtig.

\begin{example}
    Wir haben vorher schon zwei Beispiele für Morse Funktionen auf dem Torus gesehen. 
    Die Höhenfunktion ist erfüllt die Smale-Bedingung nicht - tatsächlich führen die Trajektorien 
    vom oberen Sattelpunkt direkt zum unteren Sattelpunkt.

    Die Funktion mit $f(x, y) = - \cos(x) - \cos(y)$ ist eine Morse Funktion und erfüllt auch die 
    Smale-Bedingung.
\end{example}

Wir gewinnen auch eine wichtige Erkenntnis: 

\begin{corollary}
    Der Index von kritischen Punkten verringert sich entlang von Trajektorien. Denn falls 
    $\Index (p) \leq \Index (q)$, dann ist die Dimension von $\Lt (p, q)$
    kleiner $0$, also ist dann $\Lt (p, q) = \varnothing$.
\end{corollary}

Um den Morse Komplex für jede (kompakte) Mannigfaltigkeit definieren zu können, muss noch gezeigt
werden, dass auf jeder Mannigfaltigkeit tatsächlich ein Morse-Smale Paar existiert. 
Sogar noch stärker ist die folgende Aussage:

\begin{theorem}[Satz von Smale-Kupta]
    Es sei $M$ eine Mannigfaltigkeit mit Rand und $f$ eine Morse-Funktion. Es sei $\Omega$ die 
    Vereinigung von Morse-Umgebungen von allen kritischen Punkten. Sei $X$ ein Pseudo-Gradientenfeld 
    von $f$. Dann existiert ein Pseudo-Gradientenfeld $X'$ von $f$, das die Smale-Bedingung erfüllt, 
    das innerhalb von $\Omega$ gleich $X$ ist und für das gilt:

    Für jedes $\eps > 0$, jeden Atlas $(\phi_i, U_i)_{i \in I}$ von $M$ und alle $i \in I$ existiert 
    für jede Kompakte Teilmenge $K_i \subseteq U_i$ ein Vektorfeld $X'$, sodass 
    \[ \| \opd \phi_i^{-1} (\cdot) (X') - \opd \phi_i^{-1} (\cdot) (X) \| < \eps . \]
\end{theorem}

\begin{proof}
    Der Satz wurde von S. Smale in \cite{smale1} bewiesen.
\end{proof}