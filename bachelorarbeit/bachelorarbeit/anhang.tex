\chapter*{Anhang}

\begin{definition}[Mannigfaltigkeit \cite{ludwig}]
    \label{def: mannigfaltigkeit}
    Es sei $M$ ein topologischer Raum. 
    
    Eine \textit{Karte} von $M$ ist ein Tupel
    $(U, \varphi)$, wobei $U \subseteq M$ offen und $\varphi \colon U \to U' \in \R^n$ 
    ein Homeomorphismus ist.

    $\mathcal{A} = \left\{ (U_{\alpha}, \varphi_{\alpha}) \right\}_{\alpha \in I}$ ist ein 
    \textit{$n$-dimensionaler Atlas} von $M$ falls
    \begin{enumerate}
        \item $(U_{\alpha}, \varphi_{\alpha})$ ist eine Karte für jedes $\alpha \in I$
        \item $M = \bigcup_{\alpha \in I U_{\alpha}}$
    \end{enumerate}
    Ein Atlas ist $C^k$ für $k \in \{ \N_0 \cup \{ \infty, \omega \} \}$, falls für
    alle $\alpha, \beta \in I$ der \textit{Koordinatenwechsel} 
    \[ 
        \varphi_{\alpha \beta} := 
        \varphi_{\alpha} \circ \varphi_{\beta}^{-1} \colon 
        \varphi_{\beta} (U_{\alpha} \cap U_{\beta}) \to \varphi_{\alpha} (U_{\alpha} \cap U_{\beta})
    \]
    $C^k$ ist.
    
    Eine Karte $(U, \varphi)$ heißt \textit{$C^k$ kompatibel} mit einem $C^k$ Atlas 
    $\mathcal{A} = \left\{ (U_{\alpha}, \varphi_{\alpha}) \right\}_{\alpha \in I}$,
    falls für alle $\alpha \in I$ die Koordinatenwechsel $\varphi \circ \varphi_{\alpha}^{-1}$
    und $\varphi_{\alpha} \circ \varphi^{-1}$ $C^k$ sind.

    Eine \textit{$n$-dimensionale $C^k$ Mannigfaltigkeit} ist ein topologischer Raum $M$ 
    zusammen mit einem maximalen $C^k$ Atlas $\mathcal{A}$, sodass $M$ ein Hausdorff-Raum und
    zweitabzählbar ist. Maximal bedeutet hier, dass es keine mit $\mathcal{A}$ $C^k$ 
    kompatiblen Karten gibt, die nicht in $\mathcal{A}$ enthalten sind.

    Eine Mannigfaltigkeit heißt \textit{glatt} falls $k = \infty$.

    Für einen Punkt $p \in M$ und eine Karte $(\varphi, U)$ mit $p \in U$ 
    heißen $\varphi = (x_1, ..., x_n)$ \textit{lokale Koordinaten} um $p$.
\end{definition}

\begin{remark}
    Wenn der Atlas einer Mannigfaltigkeit angegeben wird, dann nie als maximaler Atlas.
    Es reicht ein Atlas, alle anderen Karten sind dann schon impliziert.
\end{remark}

\begin{definition}[Differenzierbarkeit]
    Sind $M, N$ $C^k$ Mannigfaltigkeiten, 
    $\mathcal{A} = (\varphi_{\alpha}, U_{\alpha})_{\alpha \in I}$ ein Atlas von $M$, 
    $\mathcal{B} = (\varphi_{\beta}, U_{\beta})_{\beta \in J}$ ein Atlas von $N$,
    dann heißt eine Abbildung $C^k$ oder \textit{$k$-mal differenzierbar}, falls für alle 
    $\alpha \in I$ und $\beta \in J$ die Abbildung
    \[ \psi_{\beta} \circ f \circ \varphi_{\alpha}^{-1} \colon \R^n \to \R^m \]
    $C^k$ ist.
\end{definition}

\begin{definition}[Tangentialraum]
    Der Tangentialraum einer $C^k$ Mannigfaltigkeit $M$ an einem Punkt $p \in M$ ist
    \[ T_pM := \left\{ X_p \colon C^{k} \to \R : X_p 
        \text{ ist eine Derivation von $M$ an dem Punkt $p$} \right\} \]
    Wobei $X_p: C^{k} \to \R$ ein \textit{Derivation} ist, falls folgende Bedingungen erfüllt
    sind:
    \begin{itemize}
        \item $X_p$ ist linear
        \item Für $X_p$ gilt die Leibnitz-Regel, also
            \[ X_p (f \cdot g) = X_p (f) \cdot g + f \cdot X_p (g) \]
    \end{itemize}
    Dann ist $T_pM$ ein Untervektorraum von $C^k(C^k(M))$.

    Für eine $C^k$ Abbildung $f \colon M \to N$  und einen Punkt $p \in M$ ist dann 
    \begin{align*}
        \opd f (p) \colon T_pM & \to T_{f(p)}N \\
        X_p & \mapsto f_* X_p
    \end{align*}
    wobei $f_*X_p$ definiert ist durch
    \[ f_*X_p (g) = X_p (g \circ f) \]
\end{definition}

\begin{remark}
    \begin{align*}
        T \colon \mathbf{Man}_{*} & \to \mathbf{Vect}_{\R} \\
        (M, p) & \mapsto T_pM \\
        f & \mapsto \opd f (p)
    \end{align*}
    Ist ein Funktor.
\end{remark}

\begin{remark}
    Es sei $M$ eine $C^k$ Mannigfaltigkeit, $k \geq 1$, $p \in M$ und $\varphi = (x_1, ..., x_n)$
    lokale Koordinaten um $p$. Definiere
    \begin{align*} 
        \pderive{x_1}(p): C^k(M) & \to \R \\
        \pderive[f]{x_i} (p) & = \pderive[\varphi \circ f]{x_i} (\varphi(p))
    \end{align*}
    Dann ist $\left( \pderive{x_i} \right)_{1 \leq i \leq n}$ eine Basis von $T_pM$.

    Für eine glatte Abbildung $f \colon M \to N$, einem Punkt $p \in M$ und lokale Koordinaten 
    $(x_1, ..., x_n)$ um $p$ und $(y_1, ..., y_m)$ um $f(p)$ bekommen wir in einer Umgebung 
    von $p$ wohldefinierte Abbildungen $f_i = y_i \circ f$. Dann ist das differential 
    $\opd f (p)$ von $f$ gegeben durch die Matrix
    \[ D_p(f) = \left( \pderive[f_i]{x_j} \right)_{i,j} . \]
\end{remark}

