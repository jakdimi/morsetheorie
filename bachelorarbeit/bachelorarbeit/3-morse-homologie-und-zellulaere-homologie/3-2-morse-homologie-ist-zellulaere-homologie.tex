\section{Morse-Homologie ist zelluläre Homologie}

Das Ziel dieses Abschnittes ist es, mithilfe der im 2. Kapitel erarbeiteten Mitteln eine 
CW-Zerlegung von einer kompakten Mannigfaltigkeit $M$ zu finden.

Wir kennen schon eine disjunkte Zerlegung von $M$ in offene Kreisscheiben, nämlich
\[ \mathcal{E} = \{ \unst (p): p \in \Crit (f) \} . \]
Tatsächlich ist jede instabile Mannigfaltigkeit eine offene Kreisscheibe und jeder Punkt in 
$ p \in M$ wird von $\phi_{\bullet}(p)$ für $t \to - \infty$ genau auf einen kritischen Punkt 
transportiert (siehe Prop.~\ref{prop: trajektorien enden in kritischen punkten}).

\begin{prop}
    \label{prop: cw-zerlegung}
    Die Zerlegung $\mathcal{E} = \{ \unst (p): p \in \Crit (f) \}$ ist eine CW-Zerlegung.
\end{prop}

Bevor wir diese Proposition beweisen können, müssen wir noch ein wenig arbeiten.
Wir definieren für einen kritischen Punkt $p$:
\[ \clunst (p) = 
    \unst (p) \cup \left( \bigcup_{q \in \Crit (f)} \Lb (p, q) \times \unst (q) \right) . \]

$\Lb (p, q) \times \unst (q)$ ist nur nicht leer, wenn $\Index (p) > \Index (q)$.
Für alle $x \in \unst (p) - \{ p \}$ gibt es einen kritischen Punkt $q$ und eine Trajektorie 
$\lambda_x \in \Lt (p, c)$,m sodass $x \in \lambda_x$. wir können also jedes 
$x \in \unst (p) - \{ p \}$ künstlich zu einem Tupel $(\lambda_x, x)$ machen. 

Wie im letzten Kapitel geben wir für $\clunst (p)$ eine Topologie:

\begin{definition}[Topologie von $\clunst (p)$]
    Wir definieren eine Basis der Topologie von $\clunst (p)$. Offene Mengen in $\unst (p)$ sind 
    auch in $\clunst (p)$ offen. Für $(\lambda, x) \in \Lt (p, c_1) \times \Lt (c_{k - 1}, c)$,
    eine Umgebung $U^0$ von $x$ in $M$ und $U^+$ und $U^-$ wie vorher 
    in~\ref{def: topologie gebrochener trajektorien} die Vereinigungen offener Umgebungen der Ein-
    bzw. Austrittspunkte der jeweiligen Trajektorien in $\del_+ \Omega (c_i)$ bzw. 
    $\del_- \Omega (c_i)$ definieren wir die restlichen Elemente der Baisis
    $\mathcal{U}(\lambda, x, U^0, U^-, U^+)$ wie folgt: \\
    Falls $(\mu, x') \in \Lb(c, c_i) \times \unst (c_i)$ (mit der 
    Notation von vorher können dies also auch Punkte in $\stab(p) - \{ p \}$ sein), dann ist 
    \[ (\mu, x') \in \mathcal{U}(x, \lambda, U^0, U^-, U^+) , \]
    falls gelten:
    \begin{enumerate}
        \item $x' \in U^0$.
        \item Die (einfache) Trajektorie, die $c_i$ mit $x'$ verbindet, tritt in $\Omega (c_j)$ durch 
            $U^+_j$ für alle $j > i$ und verlässt $\Omega (c_j)$ durch $U^-_j$ für alle $j \geq i$.
            Eine solche Trajektorie existiert, da $x' \in \unst (c_i)$.
        \item $\mu \in \mathcal{U}(\tilde{\lambda}, \tilde{U}^-, \tilde{U}^+)$, wobei 
            $\tilde{\lambda} = (\lambda_1, \dots, \lambda_i)$ und 
            $\tilde{U}^{\pm} = \bigcup_{j = 1}^i U^{\pm}_j$
    \end{enumerate}
    Die offenen Mengen in $\clunst (p)$ sind dann die Mengen, die sich als Vereinigung der 
    Elemente der Basis schreiben lassen. 
\end{definition}

% \begin{remark}
%     Hier handelt es sich tatsächlich um eine Basis der Topologie, denn die oben definierete Basis
%     ist sogar unter Schnitten abgeschlossen.
% \end{remark}

\begin{example}
    \todo{}
\end{example}

\begin{prop}
    $\clunst (p)$ ist homeomorph zur abgeschlossenen Kreisscheibe $B^{\Index (p)}$, und 
    $\unst (p)$ ist das Innere von $\clunst (p)$.
\end{prop}

\begin{bigproof}
    Für Elemete $(\lambda, x) \in \clunst (p)$ gilt sobieso schon, dass $f(x) \leq f(p)$. Definiere
    \[ \clunst (p, \alpha) = 
        \{ (\lambda, x) \in \clunst (p) - \{ p \} : f(x) \geq \alpha \}  \cup \{ p \} \]
    und 
    \[ \unst (p, \alpha) = \{ x \in \unst (p) : f(x) \geq \alpha \} . \]
    Für $\alpha = f(c) - \eps$ und $\eps$ klein genug gilt 
    \[ \clunst (p, \alpha) = \unst (p) \cap \Omega (p, \eps, \eta) . \]
    Für ein belibiges $\eta$. $\Omega (p, \eps, \eta)$ ist wie in der Notation zu 
    Morse-Umgebungen~\ref{def: notation morse umgebung}. Dies ist via einer Morse-Karte homeomorph 
    zu $V^- \cap U(\eps, \eta)$, also zur abgeschlossenen $\Index (p)$-dimensionalen Kreisscheibe.
    $\Index (p)$-dimensionalen abgeschlossenen Kreisscheibe. Da $M$ kompakt ist besitzt $f$ ein 
    Minimum, und falls gilt $\alpha < \min (f)$, dann gilt offenbar
    \[ \clunst (p, \alpha) = \clunst (p) . \]
    Wir wollen also zeigen, dass für $\alpha' < \alpha$ gilt
    \[ \clunst (p, \alpha') \isom \clunst (p, \alpha) . \]
    Die Menge $\clunst (p, \alpha)$ erinnert uns an die Subniveaumengen aus 
    Abschnitt~\ref{sec: topologische eigenschaften anhand kritischer punkte}. Das erste 
    Deformationslemma leifert eine ähnliche Aussage, und wir können die folgende Behauptung 
    beweisen:

    \begin{claim}
        Wenn sich im Intervall $[\alpha', \alpha]$ keine kritischen Werte von $f$
        befinden, dann sind $\clunst (p, \alpha')$ und $\clunst (p, \alpha)$ homeomorph.
    \end{claim}

    \begin{smallproof}
        Unter Anwendung des ersten Deformationslemmas auf $-f$ bekommen wir einen Homeomorphismus
        \[ \phi \colon M^{\alpha'} \longto M^{\alpha} . \]
        Die Subniveaumengen sind die Subniveaumengen von $-f$. Dann ist auch 
        \begin{align*}
            \chi \colon \clunst (p, \alpha') \longto & \clunst (p, \alpha) \\
            (\lambda, x) \longmapsto & (\lambda, \phi(x))
        \end{align*}
        ein Homeomorphismus.
    \end{smallproof}

    Sehr viel schwieriger ist es zu beweisen, dass sich $\clunst (p, \alpha)$ selbst wenn $\alpha$
    einen kritischen Wert überquert nicht verändert. Wir nehmen an, dass es für jeden kritischen Wert
    $c$ genau einen kritischen Punkt $p$ gibt, sodass $f (p) = c$. 

    \begin{claim}
        
    \end{claim}
\end{bigproof}