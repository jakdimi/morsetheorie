\section{Morse-Homologie ist zelluläre Homologie}

Das Ziel dieses Abschnittes ist es, mithilfe der im 2. Kapitel erarbeiteten Mitteln eine 
CW-Zerlegung von einer kompakten Mannigfaltigkeit $M$ zu finden.

Wir kennen schon eine disjunkte Zerlegung von $M$ in offene Kreisscheiben, nämlich
\[ \mathcal{E} = \{ \unst (p): p \in \Crit (f) \} . \]
Tatsächlich ist jede instabile Mannigfaltigkeit eine offene Kreisscheibe und jeder Punkt in 
$ p \in M$ wird von $\phi_{\bullet}(p)$ für $t \to - \infty$ genau auf einen kritischen Punkt 
transportiert (siehe Prop.~\ref{prop: trajektorien enden in kritischen punkten}).

\begin{prop}
    \label{prop: cw-zerlegung}
    Die Zerlegung $\mathcal{E} = \{ \unst (p): p \in \Crit (f) \}$ ist eine CW-Zerlegung.
\end{prop}

Bevor wir diese Proposition beweisen können, müssen wir noch ein wenig arbeiten.
Wir definieren für einen kritischen Punkt $p$:
\[ \clunst (p) = 
    \unst (p) \cup \left( \bigcup_{q \in \Crit (f)} \Lb (p, q) \times \unst (q) \right) . \]

$\Lb (p, q) \times \unst (q)$ ist nur nicht leer, wenn $\Index (p) > \Index (q)$.
Für alle $x \in \unst (p) - \{ p \}$ gibt es einen kritischen Punkt $q$ und eine Trajektorie 
$\lambda_x \in \Lt (p, q)$, sodass $x \in \lambda_x$. wir können also jedes 
$x \in \unst (p) - \{ p \}$ künstlich zu einem Tupel $(\lambda_x, x)$ machen. Wir werden im Folgenden
nicht zwischen $x$ und $(\lambda_x, x)$ in $\unst (p)$ unterscheiden.

Wie im letzten Kapitel geben wir für $\clunst (p)$ eine Topologie:

\begin{definition}[Topologie von $\clunst (p)$]
    Wir definieren eine Basis der Topologie von $\clunst (p)$. Offene Mengen in $\unst (p)$ sind 
    auch in $\clunst (p)$ offen. Für $(\lambda, x) \in \Lt (p, c_1) \times \Lt (c_{k - 1}, c)$,
    eine Umgebung $U^0$ von $x$ in $M$ und $U^+$ und $U^-$ wie vorher 
    in~\ref{def: topologie gebrochener trajektorien} die Vereinigungen offener Umgebungen der Ein-
    bzw. Austrittspunkte der jeweiligen Trajektorien in $\del_+ \Omega (c_i)$ bzw. 
    $\del_- \Omega (c_i)$ definieren wir die restlichen Elemente der Baisis
    $\mathcal{U}(\lambda, x, U^0, U^-, U^+)$ wie folgt: \\
    Falls $(\mu, x') \in \Lb(p, c_i) \times \unst (c_i)$ oder $(\mu, x') \in W^u (p) \cap \stab (c)$, 
    dann ist \[ (\mu, x') \in \mathcal{U}(x, \lambda, U^0, U^-, U^+) , \]
    falls gelten:
    \begin{enumerate}
        \item $x' \in U^0$.
        \item Die (einfache) Trajektorie, die $c_i$ mit $x'$ verbindet, tritt in $\Omega (c_j)$ durch 
            $U^+_j$ für alle $j > i$ und verlässt $\Omega (c_j)$ durch $U^-_j$ für alle $j \geq i$.
            Eine solche Trajektorie existiert, da $x' \in \unst (c_i)$.
        \item $\mu \in \mathcal{U}(\tilde{\lambda}, \tilde{U}^-, \tilde{U}^+)$, wobei 
            $\tilde{\lambda} = (\lambda_1, \dots, \lambda_i)$ und 
            $\tilde{U}^{\pm} = \bigcup_{j = 1}^i U^{\pm}_j$
    \end{enumerate}
    Für $(\mu, x) \in \unst(p) \cap \stab (c)$ sind die zweite und die dritte Bedingung äquivalent.
    Die offenen Mengen in $\clunst (p)$ sind dann die Mengen, die sich als Vereinigung der 
    Elemente der Basis schreiben lassen. 
\end{definition}

\begin{example}
    \todo{}
\end{example}

\begin{prop}
    \label{prop: abschluss von instabilen mannigfaltigkeiten}
    Ist $p$ ein kritischer Punkt von $f$, $k = \Index (p)$, dann ist $\clunst (p)$ homeomorph 
    zur abgeschlossenen Kreisscheibe $B^k$, und $\unst (p)$ ist das Innere von $\clunst (p)$.
\end{prop}

\begin{remark}
    Mit dieser Proposition wird auch Proposition~\ref{prop: cw-zerlegung} bewiesen:
    Die ersten beiden Bedingungen für CW-Zerlegungen sind schon erfüllt. Da wir annehmen, dass 
    $M$ kompakt ist, sind auch die letzten beiden Bedingungen erfüllt. Die dritte Bedingung folgt
    dann sofort aus der letzten Proposition~\ref{prop: abschluss von instabilen mannigfaltigkeiten}.
\end{remark}

\begin{bigproof}
    Für Elemete $(\lambda, x) \in \clunst (p)$ gilt sobieso schon, dass $f(x) \leq f(p)$. Definiere
    \[ \clunst (p, \alpha) = 
        \{ (\lambda, x) \in \clunst (p) - \{ p \} : f(x) \geq \alpha \}  \cup \{ p \} \]
    und 
    \[ \unst (p, \alpha) = \{ x \in \unst (p) : f(x) \geq \alpha \} . \]
    Für $\alpha = f(c) - \eps$ und $\eps$ klein genug gilt 
    \[ \clunst (p, \alpha) = \unst (p) \cap \Omega (p, \eps, \eta) . \]
    Für ein belibiges $\eta$. $\Omega (p, \eps, \eta)$ ist wie in der Notation zu 
    Morse-Umgebungen~\ref{def: notation morse umgebung}. Dies ist via einer Morse-Karte homeomorph 
    zu $V^- \cap U(\eps, \eta)$, also zur abgeschlossenen $\Index (p)$-dimensionalen Kreisscheibe.
    Da $M$ kompakt ist besitzt $f$ ein Minimum, und falls gilt $\alpha < \min (f)$, dann gilt 
    offenbar
    \[ \clunst (p, \alpha) = \clunst (p) . \]
    Wir wollen also zeigen, dass für $\alpha' < \alpha$ gilt
    \[ \clunst (p, \alpha') \isom \clunst (p, \alpha) . \]
    Die Menge $\clunst (p, \alpha)$ erinnert uns an die Subniveaumengen aus 
    Abschnitt~\ref{sec: topologische eigenschaften anhand kritischer punkte}. Das erste 
    Deformationslemma leifert eine ähnliche Aussage, und wir können die folgende Behauptung 
    beweisen:

    \begin{claim}
        Wenn sich im Intervall $[\alpha', \alpha]$ keine kritischen Werte von $f$
        befinden, dann sind $\clunst (p, \alpha')$ und $\clunst (p, \alpha)$ homeomorph.
    \end{claim}

    \begin{smallproof}
        Unter Anwendung des ersten Deformationslemmas auf $-f$ bekommen wir einen Homeomorphismus
        \[ \phi \colon M^{\alpha'} \longto M^{\alpha} . \]
        Die Subniveaumengen sind die Subniveaumengen von $-f$. Dann ist auch 
        \begin{align*}
            \chi \colon \clunst (p, \alpha') \longto & \clunst (p, \alpha) \\
            (\lambda, x) \longmapsto & (\lambda, \phi(x))
        \end{align*}
        ein Homeomorphismus.
    \end{smallproof}

    Sehr viel schwieriger ist es zu beweisen, dass sich $\clunst (p, \alpha)$ selbst wenn $\alpha$
    einen kritischen Wert überquert nicht verändert. Wir nehmen an, dass es für jeden kritischen Wert
    $c$ genau einen kritischen Punkt $p$ gibt, sodass $f (p) = c$. 

    \begin{claim}
        Ist $\alpha = f(p)$, $q$ ein weitererkritischer Punkt von $f$ mit 
        $\Index (p) = \Index (q) + 1$ und $\eps > 0$ klein genug, dann sind 
        $\clunst (p, \alpha + \eps)$ und $\clunst (p, \alpha - \eps)$ homeomorph.
    \end{claim}

\end{bigproof}

\begin{theorem}[Morse-Homologie ist zelluläre Homologie]
    \label{satz: morse-homologie ist zellulaere homologie}
    Der zelluläre Kettenkomplex \\ $(K_{\ast}, \del)$, der durch die CW-Zerlegung in instabilen 
    Mannigfaltigkeiten durch Proposition~\ref{prop: cw-zerlegung} gegeben ist, ist isomorph zum 
    Morse-Komplex $(C_{\ast}(M, (f, X)), \del_X)$, also es existieren lineare Isomorphismen
    \[ F \colon C_k(M, (f, X)) \longto K_k \]
    für jedes $k \in \N$, sodass das folgende Diagramm kommutiert:
    \[ \begin{tikzcd}[column sep = 5ex]
        \cdots \arrow[r, "\del_X"] & 
            C_{k + 1}(M, (f, X)) \arrow[r, "\del_X"] \arrow[d, "F"] & 
            C_k(M, (f, X)) \arrow[r, "\del_X"] \arrow[d, "F"] &
            C_{k - 1}(M, (f, X)) \arrow[r, "\del_X"] \arrow[d, "F"] & \cdots \\
        \cdots \arrow[r, "\del"] & 
            K_{k + 1} \arrow[r, shorten <= 2em, shorten >= 2em, "\del"] & 
            K_k \arrow[r, shorten <= 2em, shorten >= 2em, "\del"] &
            K_{k - 1} \arrow[r, "\del"] & \cdots
    \end{tikzcd} \] 
\end{theorem}

\begin{proof}
    Es sei $F \colon C_k (M, (f, X)) \to K_k$ die lineare Abbildung, die den kritischen Punkt $p$
    auf die Zelle $\unst (p)$ schickt. Dann bildet $F$ Erzeuger auf Erzeuger ab, ist also 
    offensichtlich ein linearer Isomorphismus. Wir wollen zeigen, dass 
    $F \circ \del_X = \del \circ F$. Wir zeigen sogar, dass für kritische Punkte $p$ und $q$ mit 
    Index $k$ und $k - 1$ die Zahl $N (\unst (p), \unst (q))$, also der Grad der Abbildung 
    $\psi_{\unst (q)} \circ \phi_{\unst (p)}$ modulo $2$ gleich der Zahl $n_X(p, q)$, also 
    der Anzahl der Trajektorien von $p$ nach $q$ modulo $2$ ist. 

    Da $\Lt (p, q)$ $0$-dimensional ist, gilt $\Lb (p, q) = \Lt (p, q)$ und dann befinden sich im 
    Rand von $\clunst (p)$, also 
    \[ \bigcup_{c \in \Crit (f)} \Lb (p, c) \times \unst (c) \isom S^{k - 1} , \] 
    genau $\# \Lt (p, q)$ disjunkte Kopien von $\unst (q) \isom U^{k - 1}$. Jede dieser Kopien 
    wird mit der Anbringungsabbildung $\phi_{\unst (p)}$ via der Inklusion auf die Zelle $\unst (q)$
    geschickt. Wenn wir nun $M$ mit der Abbildung $\psi_{\unst (q)}$ kollabieren, dann ist 
    $\psi_{\unst (q)} (M)$ die $1$-Punkt-Kompaktifizierung von $\unst (q)$ und 
    \[ \phi_{\unst (p)} \circ \psi_{\unst (q)} |_{ \{ \lambda \} \times \unst (q) } \]
    Ist die Inklusion von $\unst (q)$ in ihre $1$-Punkt kompaktifizierung. Wir wissen, dass 
    $\unst (q) \isom B^{k - 1}$, und dass die Kopien $\{ \ell \} \times \unst (q)$ in 
    $\del \clunst (p)$ disjunkt sind, wir haben also die folgende Situation:
    Eine Abbildung $\Phi \colon S^{k - 1} \to S^{k - 1}$, endlich viele disjunkte Kopien 
    $(\ell, B^{k - 1}) \subseteq S^{k - 1}$, sodass ein Punkt $\ast$ in $S^{k - 1}$ existiert, sodass
    \[ \Phi \colon (\ell, B^{k - 1}) \to S^{k - 1} - \{ \ast \} \]
    für jedes $\ell$ ein Homeomorphismus ist und sodass
    \[ \Phi \left( S^{k - 1} - \bigcup_{\ell} \left( \ell, B^{k - 1} \right) \right) = \{ \ast \} . \]
    Es sei $m$ die Anzahl der kopien von $B^{k - 1}$ in $S^{k - 1}$
    Wir benutzen die lokale Grad Formel (siehe zum Beispiel \cite{hatcher}): 
    
    Wähle einen von $\ast$ verschiedenen Punkt $y$ in $S^{k - 1}$. Dann ist 
    $\Phi^{-1}(y) = \{ x_{\ell} \}_{\ell}$ mit $x_{\ell} \in (\ell, B^k)$. 
    $\Phi \colon (\ell, B^k) \to S^{k - 1} - \{ \ast \}$ ist ein Homeomorphismus, also ist 
    \[ \Phi_{\ast} \colon H_{k - 1} ((\ell, B^{k - 1}), (\ell, B^{k - 1}) - x_{\ell}) \to 
        H_{k - 1}(S^{k - 1} - \{ \ast \}, S^{k - 1} - \{ \ast, y \}) \]
    ein Isomorphismus, dann sind alle lokalen Grade $1$ oder $-1$. Dann ist 
    \[ \deg \Phi = \sum_{\ell} (\pm_{\ell} 1) = m \mod 2 \]
    Damit ist die Aussage gezeigt.
\end{proof}

Zum Beispiel bei Hatcher \cite{hatcher} kann man nachlesen, wie für einen toipologischen Raum die 
singuläre Homologie $H_{\ast} (X)$ definiert wird. Insbesondere hängt diese nur vom topologischen 
$X$ Raum ab. Hatcher zeigt auch: 

\begin{theorem}
    \label{satz: zellulaere Homologie ist singuläre Homologie}
    Für jedes $k \in \N$ gilt 
    \[ H_k (X) \isom HC_k (X, \mathcal{E}) . \]
\end{theorem}

Die zelluläre Homologie ist also nicht von der gewählten CW-Zerlegung abhägig und wir können schreiben
\[ HC_{\ast} (X) := HC_{\ast} (X, \mathcal{E}) \]

Da aber für eine kompakte Mannigfaltigkeit $M$ und ein Morse-Smale Paar $(f, X)$ gilt 
\[ HM_k(M, (f, X)) \isom HC_k(X) . \]
Es folgt direkt:

\begin{theorem}
    Die Morse Homologie ist nicht vom gewählten Morse-Smale Paar abhängig, und wir können schreiben
    \[ HM_{\ast} (M) := HM_{\ast}(M, (f, X)) \]
\end{theorem}
