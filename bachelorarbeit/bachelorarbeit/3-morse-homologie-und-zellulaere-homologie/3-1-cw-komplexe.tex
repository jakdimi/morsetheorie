\section{CW-Komplexe}

Die folgenden Definitionen sind in \cite{dold} zu finden.

\begin{definition}[CW-Zerlegung und CW-Komplexe]
    \label{def: cw-komplex}
    Es sei $X$ ein topologischer Raum. Eine offene Überdeckung $\mathcal{E}$ von $X$ heißt 
    $CW-Zerlegung$, wenn gelten:
    \begin{enumerate}
        \item Die Elemente in $\mathcal{E}$ sind paarweise disjunkt.
        \item Alle $e \in \mathcal{E}$ sind homeomorph zu offenen Kreisscheiben. \\
            Wir nennen $X^n$ die Vereinigung aller $e \in \mathcal{E}$, die Homeomorph zu einer
            $k$-dimensionalen offenen Kreisscheibe mit $k \leq n$ sind.
        \item Für jedes Element $e \in \mathcal{E}$ existiert eine stetige Abbildung
            $\Phi_e \colon (B^n, S^{n - 1}) \to (X^n \cup e, X^n)$, sodass 
            $\Phi_e \colon B^n - S^n \to e$ ein Homeomorphismus ist. $B^n$ ist die abgeschlossene 
            $n$-dimiensionale Kresscheibe.
        \item Der Abschluss $\overline{e}$ enthält nur endlich viele Elemente aus $\mathcal{E}$.
        \item Eine Teilmenge $U$ ist offen in $X$ genau dann, wenn für alle $e \in \mathcal{E}$ die
            Menge $U \cap \overline{e}$ offen in $\overline{e}$ ist. 
    \end{enumerate}
    
    Ein topologischer Raum $X$ zusammen mit einer CW-Zerlegung $\mathcal{E}$ heißt 
    \textit{CW-Komplex}.
    Wir nennen die $k$-dimensionalen $e \in \mathcal{E}$ $k$-Zellen, die Abbildung $\Phi_e$
    heißt charakteristische Abbildung von $e$, die Abbildung 
    $\phi_e := \Phi_e|_{S^{n - 1}} \colon S^{n - 1} \to X^{n - 1}$ heißt Anheftungsabbildung.
    $X^n$ heißt $n$-Skelett von $X$. Existiert ein $n$, sodass $X^n = X$, dann ist $n$ die 
    Dimension des CW-Komplexes. Existiert kein solchens $n$, dann ist die Dimiension $\infty$.
\end{definition}

\begin{remark}
    Ist $X$ kompakt mit einer CW-Zerlegung $\mathcal{E}$, dann ist $\mathcal{E}$ endlich dimensional, 
    und die Bedingungen 4. und 5. sind sowieso erfüllt.
\end{remark}

\begin{definition}[Zelluläre Homologie]
    \label{def: zellulaere homologie}
    Bemerke, dass $X^n / X^{n - 1}$ das Wedge-Produkt vieler $D^n / S^{n-1} = S^n$'s ist. 
    Wir bekommen durch die Anheftungsabbildung für jede $k$-Zelle $e$ eine Abbildung in den 
    Quotienten
    \[ \phi_e \colon S^{k-1} \longto X^{k - 1} \longto X^{k - 1} / X^{k - 2} . \]
    ist $d$ eine $k - 1$-Zelle, dann bekommen wir außerdem eine Abbildung 
    \[ \psi_d \colon X^{k - 1} / X^{k - 2} \longto S^{k - 1} , \]
    die alle Sphären in dem Quotienten außer die, die aus $\Ima (\Phi_d)$ entstanden ist
    auf den Klebepunkt abbildet.

    Sei dann $N(e, d)$ der Grad$\mod 2$ der Abbildung
    \[ \psi_d \circ \phi_c \colon S^{k - 1} \longto S^{k - 1} . \]
    Dann ist das $k$-te Glied $K_k$ des \textit{zellulären Komplexes} $(K_{\ast}, \del)$ das 
    $\F_2$-Modul, das von den $k$-Zellen in $\mathcal{E}$ erzeugt wird, und das Differential ist für 
    eine $k$-Zelle $e$ gegeben durch
    \[ \del (e) = \sum_{d \in K_{k  - 1}} N(e, d) \cdot d \] 
\end{definition}

\begin{remark}
    Man zeigt, dass $\del \circ \del = 0$, also dass $(K_{\ast}, \del)$ ein Kettenkomplex ist. 
    Siehe \cite{dold}. Außerdem ist die Homologie des zellulären Kettenkomplexes isomorph zur 
    singulären Homologie, und damit unabhängig von der gewählten CW-Zerlegung (\cite{hatcher}). 
    Wie schon vorher angedeutet, gibt es nicht für jeden topologischen Raum eine CW-Zerlegung,
    aber es ist tatsächlich gar nicht leicht einen topologischen Raum zu konstruieren, der 
    kein CW-Komplex ist.
\end{remark}

\begin{example}
    \begin{itemize}
        \item Die Sphären $S^n$ sind CW-Komplexe, denn mit $\R^n \hookrightarrow \R^{n + 1}$,
            haben wir $S^n \cap \R^n = S^{n - 1}$ und $S^n - \R^n$ ist die disjunkte Vereinigung
            zweier offener Kreisschebien der Dimension $k$. So findet man induktiv eine CW-Zerlegung
            von $S^n$ mit zwei $k$-Zellen für jedes $k$ zwischen $0$ und $n$.
            \todo{Zeichnung}
        \item der Torus ist (zum Beispiel) auf folgende Art ein CW-Komplex:
            \todo{Zeichnung}
            Wir haben also eine $0$-Zelle, zwei $1$-Zellen und eine $2$-Zelle.
    \end{itemize}
\end{example}
