\chapter{Morse-Funktionen und Pseudo-Gradienten}

Das Ziel dieses Kapitels ist es, Morse-Funktionen und Pseudo-Gradienten zu
definieren und ihre \todo{allgegenwer- tigkeit ist nicht so ein schönes Wort}
\textit{allgegenwertigkeit} zu zeigen. Ein weiteres wichtiges Ergebnis ist
das \textit{Morse-Lemma}.

\section{Morse-Funktionen}

\begin{definition}[Morse-Funktion]
    \label{def: morse-funktion}
    Eine Morse-Funktion auf einer glatten Mannigfaltigkeit $M$ ist eine glatte Funktion
    $f \colon M \to \R$ deren kritische Punkte alle nicht degeneriert sind.
\end{definition}

\begin{prop}
    Es sei $M \subseteq \R^n$ eine Untermannigfaltikgeit. Für fast jeden Punkt in $\R^n$ ist
    die Funktion
    \begin{align*}
        f_p \colon M & \longrightarrow \R \\
        x & \longmapsto || x - p ||^2
    \end{align*}
    eine Morse-Funktion.
\end{prop}

\begin{proof}
    Offensichtlich ist $f_p$ glatt. $x \in M$ ist genau dann ein kritischer Punkt von $f_p$, wenn
    $T_xM \perp (x - p)$, denn das differential von $f_p$ erweitert auf $\R^n$ ist
    \[ \opd f_p (x) = 2 (x - p). \]
    Also gilt
    \[ \opd f_p (x) (v) = \langle 2 (x - p), v \rangle . \]
    $x \in M$ ist folglich genau dann ein kritischer Punkt von $f_p$, wenn $T_xM$ orthogonal 
    zu $(x - p)$ ist.

    Bemerke, dass für eine Abbildung $f \colon \R^n \to \R$ mit $f = \langle \phi_1, \phi_2 \rangle$,
    $\phi_1, \phi_2 \colon \R^n \to \R^n$ und eine Derivation $X_p$ gilt 
    \[ X_p (f) = \langle X_p(\phi_1), \phi_2 \rangle + \langle \phi_1, X_p(\phi_2) \rangle.  \]
    Sei nun $x \in M$. Dann existiert eine Umgebung $U \subseteq \R^n$ von $x$, eine Umgebung 
    $\Omega \subseteq \R^d$ von $0$ und eine Immersion 
    \[ h \colon \Omega \longto \R^n , \]
    die ein Diffeomorphismus $h \colon \Omega \to U \cap M$ ist.
    Schreibe
    \[ h(u_1, ..., u_n) = x(u_1, ..., u_n). \]
    Dann bekommen wir die partiellen Ableitungen
    \[ 
        \pderive[f_p]{u_i} = \sum_{k = 1}^n \pderive[f_p]{x_k} \cdot \pderive[x_k]{u_i} 
        = \langle 2(x - p), \pderive[x]{u_i} \rangle 
    \]
    und 
    \[ 
        \pdderive[f_p]{u_i}{u_j} = 
            2 \left( \left\langle \pderive[x]{u_j}, \pderive[x]{u_i} \right\rangle + 
            \left\langle x - p , \pdderive[x]{u_i}{u_j} \right\rangle \right) 
    . \]
    Betrachte nun das Normalrmalenbündel
    \[ NM = \{ (x, v) \in M \times \R^n : v \perp T_xM \} \]
    von M. 
    (Wir haben $T_xM \subseteq T_x\R^n \approx \R^n$ via der Basis 
    $\left( \left. \pderive{x_i} \right|_x \right)_i$. )
    \begin{claim*}
        Der Punkt $p = x + v \in \R^n$ ist kritischer Wert von der Abbildung 
        \[ E \colon NM \longto \R^n ; (x, v) \longmapsto x + v \]
        genau dann, wenn die Matrix
        \[
            H_{(x, v)}^{h^{-1}} (f_p)
            = \left( \pdderive[f_p]{u_i}{u_j} \right)_{ij}
            = 2 \left( \left\langle \pderive[x]{u_j}, \pderive[x]{u_i} \right\rangle - 
            \left\langle v , \pdderive[x]{u_i}{u_j} \right\rangle \right)_{ij}
        \]
        nicht invertierbar ist.
    \end{claim*}

    Bemerke: Mit dem \textit{Satz von Sard} gilt dann, dass die Menge der kritischen Werte von 
    $E$ eine Nullmenge ist. Aber die Menge der kritischen Werte von $E$ sind schon alle möglichen
    Punkte $p$, für die $f_p$ nicht degenerierte kritische Punkte besitzt. Also folgt mit der 
    Behauptung schon die Aussage des Satzes.

    \begin{smallproof}
        Das orthogonale Komplement von $T_xM$ in $\R^n$ hat Dimension $n - d$. Es sei also 
        $(v_1(x), ..., v_{n-d}(x))$ eine Basis von $(T_xM)^{\perp}$. Dann ist 
        \[ (u_1, ..., u_d, t_1, ..., t_{n - d}) \longmapsto 
            \left(x(u_1, ..., u_n), \sum_{k = 1}^{n - d} t_k \cdot v_k(u_1, ..., u_d)\right) \]
        eine lokale Parametrisierung von $NM$ als Untermannigfaltigkeit von $M \times \R^n$. Dann 
        bekommen wir in lokalen Koordinaten:
        \[ E (u_1, ..., u_d, t_1, ..., t_{n - d}) = x(u_1, ..., x_d) 
        + \sum_{k = 1}^{n - d} t_k \cdot v_k (u_1, ..., u_d) , \]
        also haben wir partielle Ableitungen
        \[ \pderive[e]{u_i} = \pderive[x]{u_i} + \sum_{k = 1}^{n - d} t_k \pderive[v_k]{u_i} \]
        und 
        \[ \pderive[E]{t_j} = v_j \]
        Nun ein kleines Ergebnis aus der Linearen Algebra:

        sind $v_1, ..., v_n, u_1, ..., u_n \in \R^n$ und $u_1, ..., u_n$ linear unabhängig, 
        dann ist
        \[ (v_1 \; ... \; v_n)^T \cdot (u_1 \; ... \; u_n) = (\langle v_i, u_j \rangle)_{ij} , \]
        Also 
        \[ \rank (v_1 ... v_n) = \rank (\langle v_i, u_j \rangle)_{ij} . \]

        Die Vektoren $ \pderive[x]{u_1}, ..., \pderive[x]{u_d}, v_1, ..., v_{n - d}$ sind linear
        unabhängig. Außerdem ist $\pderive[x]{u_l}$ orthogonal zu $v_k$, also hat die Matrix mit 
        Einträgen die Skalarprodukte dieser linear unabhängigen Vektoren mit den obigen partiellen
        Ableitungen vob $E$ die Form 
        \[
            \begin{pmatrix}
                \left( \left\langle \pderive[x]{u_i}, \pderive[x]{u_j} \right\rangle + 
                    \sum_{k = 1}^{n - d} t_k 
                    \left\langle \pderive[v_k]{u_i} , \pderive[x]{u_j} \right\rangle \right)_{ij} &
                \left( \sum_{k = 1}^{n - d} 
                    \left\langle \pderive[v_k]{u_i}, v_j \right\rangle \right)_{ij} \\
                0 & E_{n - d}
            \end{pmatrix}
        \]
        Diese Matrix hat Rang $< n$ genau dann, wenn 
        \[ \rank \left( \left\langle \pderive[x]{u_i}, \pderive[x]{u_j} \right\rangle + 
            \sum_{k = 1}^{n - d} t_k 
            \left\langle \pderive[v_k]{u_i} , \pderive[x]{u_j} \right\rangle \right)_{ij} < d 
        , \]
        Aber da $v_k$ und $\pderive[x]{u_j}$ orthogonal aufeinander stehen gilt 
        \[ 
            0 = \pderive{u_i} \left\langle v_k, \pderive[x]{u_j} \right\rangle
            = \left\langle \pderive[v_k]{u_j}, \pderive[x]{u_i} \right\rangle 
            + \left\langle v_k, \pdderive[x]{u_i}{u_j} \right\rangle
        \]
        Also 
        \begin{align*}
            \left\langle \pderive[x]{u_i}, \pderive[x]{u_j} \right\rangle + 
                    \sum_{k = 1}^{n - d} t_k 
                    \left\langle \pderive[v_k]{u_i} , \pderive[x]{u_j} \right\rangle
            = & \left\langle \pderive[x]{u_i}, \pderive[x]{u_j} \right\rangle - 
            \sum_{k = 1}^{n - d} t_k 
            \left\langle v_k , \pdderive[x]{u_i}{u_j} \right\rangle \\
            = & \left\langle \pderive[x]{u_i}, \pderive[x]{u_j} \right\rangle - 
            \left\langle v , \pdderive[x]{u_i}{u_j} \right\rangle
        \end{align*}
        Es folgt die Behauptung.
    \end{smallproof}
\end{proof}

\section{Topologische Eigenschaften anhand von Morse-Funktionen}

\section{Pseudo-Gradienten}
