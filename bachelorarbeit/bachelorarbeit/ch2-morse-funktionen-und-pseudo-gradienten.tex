\chapter{Morse-Funktionen und Pseudo-Gradienten}

Das Ziel dieses Kapitels ist es, Morse-Funktionen und Pseudo-Gradienten zu
definieren und ihre \todo{allgegenwer- tigkeit ist nicht so ein schönes Wort}
\textit{allgegenwertigkeit} zu zeigen. Ein weiteres wichtiges Ergebnis ist
das \textit{Morse-Lemma}.

\section{Morse-Funktionen}

\begin{definition}[Morse-Funktion]
    \label{def: morse-funktion}
    Eine Morse-Funktion auf einer glatten Mannigfaltigkeit $M$ ist eine glatte Funktion
    $f \colon M \to \R$ deren kritische Punkte alle nicht degeneriert sind.
\end{definition}

\begin{prop}
    Für jede Mannigfaltigkeit 
\end{prop}
