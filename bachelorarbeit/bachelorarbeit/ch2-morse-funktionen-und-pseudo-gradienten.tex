\chapter{Morse-Funktionen und Pseudo-Gradienten}

Das Ziel dieses Kapitels ist es, Morse-Funktionen und Pseudo-Gradienten zu
definieren und ihre \todo{allgegen- wertigkeit ist nicht so ein schönes Wort}
\textit{allgegenwertigkeit} zu zeigen. Ein weiteres wichtiges Ergebnis ist
das \textit{Morse-Lemma}.

\section{Morse-Funktionen}

In diesem Abschnitt untersuchen wir \textit{Morse-Funktionen}:

\begin{definition}[Morse-Funktion]
    \label{satz: morse-funktion}
    Eine \textit{Morse-Funktion} auf einer glatten Mannigfaltigkeit $M$ ist eine glatte Funktion
    $f \colon M \to \R$, deren kritische Punkte alle nicht degeneriert sind.
\end{definition}

Insbesondere zeigen wir, dass Morse Funktionen nichts besonderes sind. Dafür zeigen wir, dass für 
eine Untermannigfaltigkeit $M \subseteq \R^n$ und einen Punkt $p \in \R^n$ die Abbildung
$x \mapsto \| x - p \|^2$ nur für $p$, die so gennanten \textit{Brennpunkte}
sind, keine Morse Funktion ist.

\begin{definition}[Normalenbündel]
    \label{def: normalenbuendel}
    Es sei $M \subseteq \R^n$ eine Untermannigfaltigkeit von $\R^n$. Das Normalenbündel ist die 
    Menge
    \[ NM = \{ (x, v) \in M \times \R^n : v \perp T_xM \} . \]
    Wir betrachten hier $T_xM \subseteq T_x\R^n \isom \R^n$ via der Basis 
    $\left( \del / \del x_i \right)$, wobei $x_i$ die Achsen des $\R^n$ sind.
\end{definition}

\begin{prop}
    \label{prop: NM ist untermannigfaltigkeit}
    Das Normalenbündel $NM$ ist eine $n$-dimensionale Untermannigfaltigkeit von $M \times \R^n$.
\end{prop}

\begin{proof}
    Es sei $x \in M$. Dann existiert eine Umgebung $U \subseteq \R^n$ von $x$, eine Umgebung 
    $\Omega \subseteq \R^d$ von $0$ und eine Immersion 
    \begin{align*}
        h \colon \Omega \longto & \R^n \\
        (u_1, \dots, u_d) \; \longmapsto & \; x(u_1, \dots, u_d)
    \end{align*}
    die ein Diffeomorphismus $h \colon \Omega \to U \cap M$ ist. Das orthogonale Komplement 
    von $T_xM$ in $\R^n$ hat Dimension $n - d$. Es sei also 
    $(v_1(x), ..., v_{n-d}(x))$ eine Basis von $(T_xM)^{\perp}$. Dann ist 
    \[ (u_1, ..., u_d, t_1, ..., t_{n - d}) \longmapsto 
        \left(x(u_1, ..., u_n), \sum_{k = 1}^{n - d} t_k \cdot v_k(u_1, ..., u_d)\right) \]
    eine lokale Parametrisierung von $NM$ als Untermannigfaltigkeit von $M \times \R^n$.
\end{proof}

\begin{definition}[Brennpunkt]
    \label{def: brennpunkt}
    Es sei $M \subseteq \R^n$ eine Untermannigfaltigkeit von $\R^n$. Es sei $E \colon NM \to R^n$ 
    mit $E (x, v) = x + v$. Ein \textit{Brennpunkt} von $M$ ist ein kriticher Wert von $E$.
\end{definition}

\begin{remark}
    Aus dem Satz von Sard folgt, dass die Menge der Brennpunkte eine Nullmenge ist.
    Intuitiv sind die Brennpunkte einer Untermannigfaltigkeit die Punkte im $\R^n$, an denen sich
    die Normalen von nahe aneinanderliegenden Punkten schneiden.
\end{remark}

\begin{lemma}
    \label{lemma: char. von Brennpunkten}
    Es sei $M \subseteq \R^n$ eine Untermannigfaltigkeit von $\R^n$ $x \in M$ und $M$ in einer 
    Umgebung von $x$ und $NM$ parametriesiert wie im Beweis von 
    Proposition~\ref{prop: NM ist untermannigfaltigkeit}. Dann ist $p = x + v$ genau dann ein
    Brennpunkt von $M$, wenn die Matrix 
    \[
        \left( \left\langle \pderive[x]{u_j}, \pderive[x]{u_i} \right\rangle - 
        \left\langle v , \pdderive[x]{u_i}{u_j} \right\rangle \right)_{ij}
    \]
    nicht invertierbar ist.
\end{lemma}

\begin{proof}
    Wir haben partielle Ableitungen
    \[ \pderive[e]{u_i} = \pderive[x]{u_i} + \sum_{k = 1}^{n - d} t_k \pderive[v_k]{u_i} \]
    und 
    \[ \pderive[E]{t_j} = v_j \]
    Nun ein kleines Ergebnis aus der Linearen Algebra:

    sind $v_1, ..., v_n, u_1, ..., u_n \in \R^n$ und $u_1, ..., u_n$ linear unabhängig, 
    dann ist
    \[ (v_1 \; ... \; v_n)^T \cdot (u_1 \; ... \; u_n) = (\langle v_i, u_j \rangle)_{ij} , \]
    Also 
    \[ \rank (v_1 ... v_n) = \rank (\langle v_i, u_j \rangle)_{ij} . \]

    Die Vektoren $ \pderive[x]{u_1}, ..., \pderive[x]{u_d}, v_1, ..., v_{n - d}$ sind linear
    unabhängig. Außerdem ist $\pderive[x]{u_l}$ orthogonal zu $v_k$, also hat die Matrix mit 
    Einträgen die Skalarprodukte dieser linear unabhängigen Vektoren mit den obigen partiellen
    Ableitungen von $E$ die Form 
    \[
        \begin{pmatrix}
            \left( \left\langle \pderive[x]{u_i}, \pderive[x]{u_j} \right\rangle + 
                \sum_{k = 1}^{n - d} t_k 
                \left\langle \pderive[v_k]{u_i} , \pderive[x]{u_j} \right\rangle \right)_{ij} &
            \left( \sum_{k = 1}^{n - d} 
                \left\langle \pderive[v_k]{u_i}, v_j \right\rangle \right)_{ij} \\
            0 & E_{n - d}
        \end{pmatrix}
    \]
    Diese Matrix hat Rang $< n$ genau dann, wenn 
    \[ \rank \left( \left\langle \pderive[x]{u_i}, \pderive[x]{u_j} \right\rangle + 
        \sum_{k = 1}^{n - d} t_k 
        \left\langle \pderive[v_k]{u_i} , \pderive[x]{u_j} \right\rangle \right)_{ij} < d 
    , \]
    Aber da $v_k$ und $\pderive[x]{u_j}$ orthogonal aufeinander stehen gilt 
    \[ 
        0 = \pderive{u_i} \left\langle v_k, \pderive[x]{u_j} \right\rangle
        = \left\langle \pderive[v_k]{u_j}, \pderive[x]{u_i} \right\rangle 
        + \left\langle v_k, \pdderive[x]{u_i}{u_j} \right\rangle
    \]
    Also 
    \begin{align*}
        \left\langle \pderive[x]{u_i}, \pderive[x]{u_j} \right\rangle + 
                \sum_{k = 1}^{n - d} t_k 
                \left\langle \pderive[v_k]{u_i} , \pderive[x]{u_j} \right\rangle
        = & \left\langle \pderive[x]{u_i}, \pderive[x]{u_j} \right\rangle - 
        \sum_{k = 1}^{n - d} t_k 
        \left\langle v_k , \pdderive[x]{u_i}{u_j} \right\rangle \\
        = & \left\langle \pderive[x]{u_i}, \pderive[x]{u_j} \right\rangle - 
        \left\langle v , \pdderive[x]{u_i}{u_j} \right\rangle
    \end{align*}
    Es folgt die Behauptung.
\end{proof}

\begin{prop}
    \label{prop: existenz morse-funktionen}
    Es sei $M \subseteq \R^n$ eine Untermannigfaltikgeit. Für fast jeden Punkt in $\R^n$ ist
    die Funktion
    \begin{align*}
        f_p \colon M & \longrightarrow \R \\
        x & \longmapsto \| x - p \|^2
    \end{align*}
    eine Morse-Funktion.
\end{prop}

\begin{proof}
    Offensichtlich ist $f_p$ glatt. $x \in M$ ist genau dann ein kritischer Punkt von $f_p$, wenn
    $T_xM \perp (x - p)$, denn das differential von $f_p$ erweitert auf $\R^n$ ist
    \[ \opd f_p (x) = 2 (x - p). \]
    Also gilt
    \[ \opd f_p (x) (v) = \langle 2 (x - p), v \rangle . \]
    $x \in M$ ist folglich genau dann ein kritischer Punkt von $f_p$, wenn $T_xM$ orthogonal 
    zu $(x - p)$ ist.

    Bemerke, dass für eine Abbildung $f \colon \R^n \to \R$ mit $f = \langle \phi_1, \phi_2 \rangle$,
    $\phi_1, \phi_2 \colon \R^n \to \R^n$ und eine Derivation $X_p$ gilt 
    \[ X_p (f) = \langle X_p(\phi_1), \phi_2 \rangle + \langle \phi_1, X_p(\phi_2) \rangle.  \]
    Sei nun $x \in M$. Dann existiert eine Umgebung $U \subseteq \R^n$ von $x$, eine Umgebung 
    $\Omega \subseteq \R^d$ von $0$ und eine Immersion 
    \[ h \colon \Omega \longto \R^n , \]
    die ein Diffeomorphismus $h \colon \Omega \to U \cap M$ ist.
    Schreibe
    \[ h(u_1, ..., u_n) = x(u_1, ..., u_n). \]
    Dann bekommen wir die partiellen Ableitungen
    \[ 
        \pderive[f_p]{u_i} = \sum_{k = 1}^n \pderive[f_p]{x_k} \cdot \pderive[x_k]{u_i} 
        = \langle 2(x - p), \pderive[x]{u_i} \rangle 
    \]
    und 
    \[ 
        \pdderive[f_p]{u_i}{u_j} = 
            2 \left( \left\langle \pderive[x]{u_j}, \pderive[x]{u_i} \right\rangle + 
            \left\langle x - p , \pdderive[x]{u_i}{u_j} \right\rangle \right) 
    . \]
    
    Also hat nach Lemma~\ref{lemma: char. von Brennpunkten} $f_p$ in einer Umgebung von $x$ genau 
    dann nicht-degenerierte kritische Punkte, wenn $f_p$ ein Brennpunkt von $M$ ist. Mit der 
    Bemerkung nach der Definition von Brennpunkten~\ref{def: brennpunkt} folgt dann direkt die 
    Behauptung.
\end{proof}

\begin{remark}
    Mit dem Einbettungssatz von Whitney folgt dann direkt, dass es auf jeder Mannigfaltigkeit 
    $M$ viele Morse-Funktionen gibt. Wir können sogar noch eine stärkere Aussage beweisen:
\end{remark}

\begin{theorem}
    \label{satz: morse-approximation}
    Es sei $M$ eine Mannigfaltigkeit, $f \colon M \to \R$ glatt. Dann kann $f$ in jeder kompakten
    Teilmenge $K$ beliebig gut von einer Morse Funktion approximirt werden, also für jedes 
    $\eps > 0$ existiert eine Morse Funktion $g \colon K \to \R$, sodass 
    \[ \| \, f - g \, \|_{\infty} < \eps . \]
\end{theorem}

\begin{proof}
    Wir wählen eine Einbettung $h' \colon M \to \R^{n - 1}$. Dann ist 
    \[ h \colon M \longto \R^n \; ; \; h(x) = (f(x), h'(x)) \]
    eine Einbettung von $M$ in $\R^n$. Seien $c, \eps_1, \dots, \eps_n > 0$, sodass für \\
    $p = (c - \eps_1, \eps_2, \dots, \eps_n)$ die Funktion $f_p$ eine Morse Funktion ist.
    Setze nun 
    \[ g(x) = \frac{f_p(x) - c^2}{2c} . \]
    $g$ ist offensichtlich eine Morse-Funktion. Wir rechnen:
    \begin{align*}
        g(x) = & \frac{1}{2c} \left( (f(x) + c - \eps_1)^2 + (h_1(x) - \eps_2)^2 
            + \dots + (h_{n-1}(x) - \eps_n)^2 - c^2 \right) \\
        = & f(x) + \frac{f(x)^2 + \sum h_i(x)^2}{2c} - \frac{\eps_1 f(x) 
            + \sum \eps_i h_{i - 1}(x)}{c} + \sum \eps_i^2 - \eps_1
    \end{align*}
    Man kann nun $c$ beliebig groß und $\eps_1, \dots, \eps_n$ beliebig klein wählen,
    sodass $g$ beliebig nah an $f$ ist.
\end{proof}

\begin{remark}
    Die meiste Zeit werden wir uns in dieser Arbeit kompakte Mannigfaltigkeiten untersuchen,
    auf solchen kann jede glatte Funktion sogar global mit einer Morse Funktion approximieren.
\end{remark}

\section{Vektorfelder und Pseudo-Gradienten}

Wir untersuchen erst ein Paar Eigenschaften von Vektorfeldern.

\begin{definition}[Flusslinie]
    \label{def: flussliene}
    Es sei $I \subseteq \R$ ein Intervall, $M$ eine glatte Mannigfaltigkeit und  
    $\gamma \colon I  \to M$ ein glatter Weg. Dann definiere für $t_0 \in \R$
    \[ \derive[\gamma]{t} (t_0) := 
        \opd \gamma (t_0) \left( \pderive{t} \right) \in T_{\gamma(t_0)}M \]
    wobei $\pderive{t}$ das von der Indentität auf $\R$ induziertze Element in $T_t\R$ ist.

    Es sei $X \in \VFs (M)$ ein Vektorfeld auf $M$. $\gamma$ heißt Flusslinie von $X$
    falls für alle $t_0 \in \R$ gilt: 
    \[ \derive[\gamma]{t}(t_0) = X(\gamma(t_0)) . \]
\end{definition}

\begin{definition}[1-Parameter Gruppe aus Diffeomorphismen]
    \label{def: 1-parameter gruppe aus diffeos}
    Es sei $M$ eine glatte Mannigfaltigkeit. Eine 
    \textit{1-Parameter Gruppe aus Diffeomorphismen} ist eine glatte Abbildung
    \begin{align*}
        \phi \colon \R \times M \longto & \; M \\
        (t, p) \longmapsto & \; \phi_t(p)
    \end{align*}
    sodass gelten: 
    \begin{itemize}
        \item Für alle $s, t \in \R$ gilt $\phi_{s + t} = \phi_s \circ \phi_t$ und
        \item $\phi_0 = \id_{M}$.
    \end{itemize}

    Für eine 1-Parameter Gruppe aus Diffeomorphismen $\phi$ schreiben wir 
    \[ \varphi_{\bullet}(p): \R \to M ; t \mapsto \varphi_t(p) . \]

    Es sei $X \in \VFs (M)$. Eine 1-Parameter Gruppe aus Diffeomorphismen $\phi$ heißt 
    \textit{von $X$ erzeugt}, falls für alle $p \in M$ gilt:
    \[ X(p) = \derive[\phi_{\bullet}(p)]{t}(0) \]
\end{definition}

\begin{remark}
    Wie der Name suggestiert, ist für jedes $t \in \R$ $\phi_t$ ein 
    Diffeomorphismus: Das Inverse von $\phi_t$ ist $\phi_{-t}$.

    Ist außerdem $\phi$ eine von einem Vektorfeld $X$ erzeugte 1-Parameter Gruppe aus 
    Diffeomorphismen, dann sind $\phi_{\bullet}(p)$ Flusslinien von $X$:
    \begin{align*}
        X(\varphi_{t_0}(p)) 
        & = \derive[\varphi_{\bullet}(\varphi_{t_0}(p))]{t}(0)
        = \opd \varphi_{\bullet} (0) (\varphi_{t_0}(p)) \left(\derive{t}\right) \\
        & = \opd (\varphi_{t_0 + \bullet}(p)) (0) \left(\derive{t}\right)
        = \opd (\varphi_{\bullet}(p)) (t_0) \cdot \opd (t_0 + \id_{\R}) (0) \left(\derive{t}\right) \\
        & = \opd (\varphi_{\bullet}(p)) (t_0) \left(\derive{t}\right)
        = \opd (\varphi_{\bullet}(p)) (t_0) \left(\derive{t}\right) \\
        & = \derive[\varphi_{\bullet}(p)]{t}(t_0)
    \end{align*}
\end{remark}

\begin{prop}
    \label{prop: kompaktes VF generiert 1-param. grp.}
    Es sei $M$ eine glatte Mannigfaltigkeit, $X \in \VFs (M)$ mit kompaktem Träger. Dann 
    generiert $X$ eine eindeutige 1-Parameter Gruppe aus Diffeomorphismen.
\end{prop}

\begin{proof}
    Für jeden Punkt $p \in M$ existiert eine Karten-Ungebung $(U_p, \phi_p)$. In dieser
    Umgebung hat das Anfangswertproblem
    \[ \derive[\gamma]{t} = X (\gamma) \; , \; \gamma(0) = p \]
    eine eindeutige Lösung in einem Intervall $[-\eps_p, \eps_p]$. Diese Lösung 
    $\gamma$ hängt glatt vom Anfangswert ab. Wir schreiben
    $\phi_{\bullet}(p) := \gamma$. In dieser Umgebung gilt schon \\ 
    $\phi_{t + s} = \phi_t \circ \phi_s$, solange $t, s, t + s \in [-\eps_p, \eps_p]$. 
    Da $\supp \, X$ kompakt ist existiert eine endliche Menge ${p_1, \dots, p_k}$, 
    sodass $\supp \, X \subseteq \bigcup_i U_{p_i}$. Es sei $\eps$ das Minimum der 
    $\eps_{p_i}$. Setze $\phi_t(p) = p$ für alle $p$ nicht im Träger von $X$. Wir haben nun 
    fast einen Kandidaten für die von $X$ generierte 1-Parameter Gruppe aus Diffeomorphismen;
    $\phi_t(p)$ ist definiert für alle $p \in M$ und $t \in [-\eps, \eps]$. Wir müssen also nur 
    noch einen Kandidaten für $\phi_t(p)$ finden, falls $|t| \geq \eps$.

    Wir können jede Zahl $t \in \R$ schreiben als $t = m \cdot \sfrac{\eps}{2} + r$ mit 
    $0 \leq r < \sfrac{\eps}{2}$ und $m \in \Z$. Sei nun zuerst $t \geq 0$, dann ist $m \geq 0$.
    Setze für alle $p \in M$ 
    \[ \phi_t(p) 
        := \phi_{\sfrac{\eps}{2}} \circ \dots \circ \phi_{\sfrac{\eps}{2}} \circ \phi_r , \] 
    Wobei wir $\phi_{\sfrac{\eps}{2}}$ $|m|$ mal anwenden. Falls $t < 0$ ersetze $\sfrac{\eps}{2}$
    mit $- \sfrac{\eps}{2}$. 
\end{proof}

\begin{remark}
    Falls $M$ eine kompakte Mannigfaltigkeit ist, dann generieren alle Vektorfelder eindeutige 
    1-Parametergruppen aus Diffeomorphismen.
\end{remark}

\begin{definition}[Riemannsche Metrik]
    \label{def: riemannsche metrik}
    Es sei $M$ eine Mannigfaltigkeit. Es sei 
    \[ g_p \colon T_pM \times T_pM \longto T_pM \]
    ein Skalarprodukt für jedes $p \in M$, sodass für alle $X, Y \in \VFs (M)$ die Abbildung 
    \[ p \longmapsto g_p(X(p), Y(p)) \]
    glatt ist. Dann heißt $g$ \textit{Riemmannsche Metrik} auf $M$. Wir schreiben für 
    $x, y \in T_pM$ 
    \[ \langle x, y \rangle := g_p(x, y) \text{ und } \| x \| := \sqrt{g_p(x, x)} . \]
\end{definition}

\begin{remark}
    Man kann zeigen, dass alle Mannigfaltigkeiten eine Riemannsche Metrik besitzen.
\end{remark}

\begin{definition}[Gradient]
    \label{def: gradient}
    Es sei $M$ eine glatte Mannigfaltigkeit, $f \colon M \to \R$ eine glatte Abbilding. Dann 
    ist der Gradient von $f$ das eindeutige Vektorfeld $\grad f$, sodass für alle $X \in \VFs (M)$
    gilt 
    \[ \langle X , \grad f \rangle = \opd f X . \]
\end{definition}

\begin{definition}[Pseudo-Gradient]
    \label{def: pseudo-gradient}
    Es sei $M$ eine Mannigfaltigkeit, $f \colon M \to \R$ eine glatte Funktion. $X \in \VFs (M)$
    heißt \textit{Pseudo-Gradient} oder \textit{Pseudo-Gradientenfeld} von $f$, falls gelten:
    \begin{itemize}
        \item $\opd f (p) (X(p)) \leq 0$ für alle $p \in M$, mit Gleichheit genau dann wenn 
            $p$ ein kritischer Punkt von $f$ ist.
        \item Für jeden kritischen Punkt $p$ von $f$ existiert eine Morse-Umgebung 
            $(U_p, \phi_p)$, in der $X (q) = - \opd (\phi_p^{-1}) (q) \cdot \grad (f \circ \phi_p^{-1})$.
    \end{itemize}
\end{definition}

\begin{prop}
    Es sei $M$ eine Mannigfaltigkeit und $f \colon M \to \R$ eine Morse-Funktion. 
    Dann existiert ein Pseudo-Gradientenfeld von $f$.
\end{prop}

\begin{proof}
    Da $M$ zweitabzählbar ist und die kritischen Punkte isoliert, ist die Menge der kritischen
    punkte ${p_i}_{i \in I'}$ abzählbar. Seien dann ${(U_i, \phi_i)}_{i \in I'}$ Karten-Umgebungen
    von den kritischen Punkten, sodass in diesen Umgebungen $f$ die Form hat wie im 
    Morse-Lemma~\ref{satz: morse-lemma}. Ergänze ${(U_i, \phi_i)}_{i \in I'}$ zu einem Atlas 
    ${(U_i, \phi_i)}_{i \in I}$, sodass jeder kritische Punkt $p_i$ nur in $U_i$ enthalten ist.
    definiere nun die Vektorfelder
    \[ X_i (p) := \opd (\phi_i) (p) \circ \grad (f \circ \phi_i^{-1}) (\phi_i(p)) \]
    auf $\phi_i(U_i)$. Setze nun
    \[ \tilde{X_i}(p) = \begin{cases}
        \lambda_i (p) \cdot X_i(p) & \text{ falls } p \in \phi_i(U) \\
        0 & \text{ sonst }
    \end{cases} . \]
    Per Definition gilt schon $\opd f (p) (X_i(p)) \leq 0$ für alle $p \in M$ und $i \in I$.
    Nun wähle eine Partition der 1 $(\lambda_i)_{i \in I}$ über $(U_i)_{i \in I}$. Dann setze
    \[ X := \sum_{i \in I} \tilde{X_i}(p) . \]
    Falls $p_i$ ein kritischer Punkt von $M$ ist, dann ist $\tilde{X_j}(p) = 0$ 
    für alle $j \neq i$. Also ist 
    \[ X(p) = \tilde{X_i}(p) = 0 . \]
\end{proof}

\section{Topologische Eigenschaften anhand kritischer Punkte}

In diesem Abschnitt werden wir das erste Mal das Ausmaß der Möglichkeiten, die Morse Theorie 
bietet erfahren. Es werden die beiden Deformationslemmata bewiesen. Anhand dieser kann man 
die Morse Ungleichungen beweisen und sogar zeigen, dass jede (glatte) Mannigfaltigkeit
den Homotopietypen eines CW-Komplexes besitzt.

\begin{theorem}[Erstes Deformationslemma]
    \label{satz: erstes deformationslemma}
    Es sei $M$ eine glatte Mannigfaltigkeit und $f: M \rightarrow \R$ eine
    glatte Abbildung. Hat $f$ keine kritischen Werte im Intervall $[a, b]$ und 
    ist $f^{-1}[a, b]$ kompakt, so existiert ein Diffeomorphismus 
    $M^a \rightarrow M^b$, und $M^a$ ist ein Deformationsretrakt von $M^b$.
\end{theorem}

Die Idee des Beweises ist es, $M^a$ entlang der Richtung, in die $f$ am stärksten
steigt, also entlang des Gradientenfeldes mit einem Diffeomorphismus $\varphi$ 
"nach oben zu ziehen", bis $\varphi(f^{-1}(a)) = f^{-1}(b)$.

\begin{bigproof}[Beweis erstes Deformationslemma]
    Es existiert eine kompakte Umgebung $K \in M$ von $f^{-1}[a, b]$. Dies folgt
    aus Whitneys Einbettungssatz und dem Satz von Heine-Borel.
    Sei $\rho: M \to \R$ eine glatte, positive Funktion, sodass
    \[ \rho(p) = 1 / \langle \grad f, \grad f \rangle \]
    für alle $p \in f^{-1}[a, b]$ und die außerhalb von $K$ verschwindet und für
    die für alle $p \in K$, die keine kritischen Punkte sind, gilt: 
    \[ 0 \leq \rho(p) \leq 1 / \langle \grad f, \grad f \rangle \]
    Bemerke dass $\rho$ innerhalb von $f^{-1}[a, b]$ wohldefiniert 
    ist, da sich keine kritischen Punkte im Intervall $[a, b]$ befinden. 
    Definiere ein Vektorfeld $X$ durch
    \[ X(p) = \rho(p) \cdot \grad f (p) \]
    Dann hat $X$ kompakten Träger, erfüllt also die Vorraussetzungen von 
    Lemma~\ref{prop: kompaktes VF generiert 1-param. grp.}. Sei also $\varphi$ die
    einzigartige 1-Parameter Gruppe aus Diffeomorphismen, die von $X$ generiert
    wird. 
    Wir bekommen für jedes $p \in M$ eine Abbildung 
    $f \circ \varphi_{\bullet}(p): \R \to \R$.
    
    \begin{claim} 
        Für alle $p \in M$, $t_0 \in \R$ und $q = \varphi_{t_0}(q)$
        ist $\derive{t} f \circ \varphi_{\bullet}(p) (t_0) \in [0, 1]$ und falls $f(\varphi_t(q)) \in [a, b]$
        gilt sogar $\derive{t} f \circ \varphi_{\bullet}(q) (t_0) = 1$.
    \end{claim}

    \begin{smallproof}
        Für $q = \varphi_{t_0}(p)$:
        \begin{align*}
            \derive{t} f \circ \varphi_{t_0}(p)
            & = T_{\varphi_{t_0}(p)} f \cdot T_{t_0}\varphi_{\bullet}(p) \left( \derive{t} \right)
            = \opd f (q) \cdot X(q) \\
            & = \langle X(q), \grad f (q) \rangle 
            = \rho(q) \langle \grad f (q), \grad f (q) \rangle \in [0, 1]
        \end{align*}
        
        $f \circ \varphi_{\bullet}(p)$ ist also monoton wachsend für alle $p \in M$.

        Falls sogar $f(\varphi_p(t_0)) \in [a, b]$, dann gilt
        \[ \frac{d}{dt} f \circ \varphi^p (t_0) = 1 \]
    \end{smallproof}

    \begin{claim} 
        Für $p \in f^{-1}(a)$, $t_0 \in [0, b-a]$ gilt $f(\varphi_{t_0}(p)) \in [a, b]$.
    \end{claim}
    
    \begin{smallproof}
        \[ f(\varphi_{t_0}(p)) \geq f(\varphi_0(p)) = a \]
        und
        \begin{align*}
            f(\varphi_t(p)) 
            & \leq f(\varphi_{b-a}(p)) \\
            & = \int_0^{b-a}\derive{t} f(\varphi_t(p)) \opd t + f(\varphi_0(p)) \\
            & = \int_0^{b-a}\rho(\varphi_t(p)) \langle \grad f (\varphi_t(p)), \grad f (\varphi_t(p)) \rangle \opd t + a \\
            & \leq \int_0^{b-a} 1 \, \opd t + a \\
            & = b
        \end{align*}
    \end{smallproof}

    \begin{claim} 
        Unter $\varphi_{b-a}$ wird die Niveaumenge 
        $f^{-1}(a)$ auf die Niveaumenge $f^{-1}(b)$ abgebildet.
    \end{claim}
     
    \begin{smallproof}
        Für $p \in f^{-1}(a)$ gilt:
        \[ \varphi_{a-a}(p) = \varphi_0(p) = p \]
        und für $t_0 \in [0, b - a]$ gilt wegen Behauptung 1 und 2
        \[ \derive{t}f(\varphi_{\id_{\R} - a}(p)) (t_0) = 1 \]
        also
        \[ f(\varphi_{b - a}(p)) = f(\varphi_{0}(p)) + (b - a) = b \]
        Genauso gilt für $q \in f^{-1}(b)$: $f(\varphi_{a - b}(q)) = a$, also 
        $\varphi_{b - a}(f^{-1}(a)) = f^{-1}(b)$.
    \end{smallproof}

    \begin{claim}
        $\varphi_{b - a} (M^a) = M^b$
    \end{claim}

    \begin{smallproof}
        "$\subseteq$": Sei $p \in M^a$. OBdA. existiert $s \in [0, b-a]$, sodass 
        $f(\varphi_s(p)) = a$, ansonsten gilt für alle 
        $s \in [0, b-a]: f(\varphi_s(p)) \leq a < b$. Dann gilt
        \[ f(\varphi_{b-a}(p)) \leq f(\varphi_{b-a+s}(p)) = f(\varphi_{b-a}(\varphi_s(p))) = b \] 
        "$\supseteq$": Analog.
    \end{smallproof}

    Damit ist $\left. \varphi_{b-a} \right\vert_{M^a}$ ein Diffeomorphismus zwischen
    $M^a$ und $M^b$. 

    Betrachte nun $r: M^b \times \R \to M^b$,
    \[  
        r(p, t) = \begin{cases}
            p & \text{ falls } f(p) \leq a \\
            \varphi_{t(a - f(p))}(p) & \text{ falls } a \leq f(p) \leq b 
        \end{cases}
    \]

    Dann ist $r$ stetig, $r(\cdot, 0)$ ist die Identität auf $M^b$, 
    $r(\cdot, 1)|_{M^a}$ ist die Identität auf $M^a$ und 
    $r(1, M^b) \subseteq M^a$, also ist $M^a$ ein Deformationsretrakt von $M^b$.

\end{bigproof}

\begin{theorem}[Zweites Deformations-Lemma]
    \label{satz: zweites deformationslemma}
    Es sei $M$ eine glatte Mannigfaltigkeit, $f: M \rightarrow \R$ eine glatte
    Abbildung und $p$ ein nicht-degenerierter kritischer Punkt mit Index 
    $k$. Sei $c := f(p)$ und $\varepsilon \geq 0$, sd. 
    $f^{-1}[c - \varepsilon, c + \varepsilon]$ kompakt ist und außer $p$ keine 
    weiteren kritischen Punkte von $f$ beinhaltet. Dann hat $M^{c-\varepsilon}$
    denselben Homotopietypen wie $M^{c - \varepsilon} \cup e^k$.
\end{theorem}

Die Idee für den Beweis ist, sich eine neue Funktion $F: M \to \R$ zu definieren,
die Außerhalb von einer kleinen Umgebung von $p$ $f$ entspricht und in der 
Umgebung etwas kleiner ist. Dann bekommen wir die folgende Situation:

\begin{figure}[H]
    \centering
    \includegraphics[width=0.8\linewidth]{../resources/Me-Diagram5-sublevelsets-of-f-and-F.jpeg}
    \label{fig:me-diagram5}
    \caption{Die Niveaumengen von $f$ (links) und $F$ (rechts)}
\end{figure}

Wir wollen also, dass $M^{c + \varepsilon} = F^{-1}(- \infty, c + \varepsilon]$ 
gilt und $F^{-1}(-\infty, c - \varepsilon]$ fast dasselbe ist wie 
$M^{c - \varepsilon}$, nur dass $F^{-1}(-\infty, c - \varepsilon]$ einen "Henkel"
enthält der den kritischen Punkt $p$ enthält.

\begin{bigproof}[Beweis zweites Deformationslemma]
    Sei $c := f(p)$. Mit dem Morse-Lemma können wir lokale Koordinaten 
    $\varphi = (u_1, ..., u_n)$ in einer Umgebung $U$ von $p$ wählen, sodass
    \[ f = c - u_1^2 - ... - u_k^2 + u_{k+1}^2 + ... + u_n^2 \]
    in dieser Umgebung, und sodass für den kritischen Punkt $p$ gilt:
    \[ u_1(p) = ... = u_n(p) = 0 \]

    Sei oBdA. $\varepsilon > 0$ klein genug, sodass 
    \begin{enumerate}
        \item $f^{-1}[c - \varepsilon, c + \varepsilon]$ kompakt ist und keine
            kritischen Punkte außer $p$ enthält
        \item $\{ x \in \R^n: \lVert x \rVert^2 \leq 2 \varepsilon \} \subseteq \varphi(U) $
    \end{enumerate}

    Wähle nun die $k$-Zelle 
    \[ 
        e^k := \{ p \in M: (u_1(p))^2 + ... + (u_k(p))^2 \leq \varepsilon 
        \text{ und } u_{k+1}(p) = ... = u_n(p) = 0 \} 
    \]

    Wir bekommen die folgende Situation:

    \begin{figure}[H]
        \centering
        \includegraphics[width=0.8\linewidth]{../resources/Me-Diagram6-U-parameterized.png}
        \label{fig:me-diagram6}
        \caption{U parametrisiert}
    \end{figure}

    Nun definiere eine glatte Funktion $\mu: \R \to \R$ mit den Eigenschaften:

    \begin{enumerate}
        \item $ \mu(0) > \varepsilon $
        \item $ \mu(r) = 0 $ falls $ r \geq 2 \varepsilon $
        \item $ -1 < \mu'(r) \leq 0 $ für alle $ r \in \R $
    \end{enumerate}

    Sei nun $F$ außerhalb von $U$ gleich $f$, und sei
    \[ F = f - \mu(u_1^2 + ... + u_k^2 + 2u_{k+1}^2 + ... + 2u_n^2) \]

    $F$ ist wohldefiniert und glatt, da $F$ außerhalb des Kreises mit Radius 
    $\sqrt{2\varepsilon}$ mit $f$ übereinstimmt und der gesamte Kreis in $U$ 
    enthalten ist. Damit haben wir einen guten Kandidaten foür $F$ gefunden.

    Wir definieren nun

    \begin{align*}
        & \eta, \xi: U \to [0, \infty) \\
        & \xi = u_1^2 + ... + u_k^2 \\
        & \eta = u_{k + 1}^2 + ... + e_n^2
    \end{align*}

    Dann gilt innerhalb von $U$:
    \[ f = c - \xi + \eta \]
    und 
    \[ F = f - \mu(\xi + 2 \eta) = c - \xi + \eta - \mu(\xi + 2 \eta) \]

    Jetzt wollen wir überprüfen:
    \begin{enumerate}
        \item $F^{-1}(-\infty, c + \varepsilon] = M^{c + \varepsilon}$.
        \item $F^{-1}(-\infty, c - \varepsilon]$ ist ein Deformationsretrakt von 
            $M^{c + \varepsilon}$.
        \item $M^{c - \varepsilon} \cup e^k$ ist ein Deformationsretrakt von
            $F^{-1}(-\infty, c - \varepsilon]$.
    \end{enumerate}

    Dann folgt schon die Behauptung.

    \begin{claim} 
        $F^{-1}(-\infty, c + \varepsilon] = M^{c + \varepsilon}$
    \end{claim}

    \begin{smallproof}
        Sei $q \in M$. Falls gilt $\xi(q) + 2 \eta(q) > 2 \varepsilon$ gilt 
        $F(q) = f(q) - \mu(\xi(q) + 2\eta(q)) = f(q)$,
        also gelte oBdA.
        \[ \xi(q) + 2 \eta(q) \leq 2 \varepsilon \]
        Dann:
        \[ F(q) \leq f(q) = c - \xi(q) + \eta(q) \leq c + \frac{1}{2}\xi(q) + \eta(q) \leq c + \varepsilon \]
    \end{smallproof}

    \begin{claim} 
        $F^{-1}(-\infty, c - \varepsilon]$ ist ein
        Deformationsretrakt von $M^{c + \varepsilon}$.
    \end{claim}

    \begin{smallproof}
        Bemerke: Die kritischen Punkte von $F$ stimmen mit denen von $f$ überein, 
        denn:

        \[ \pderive[F]{\xi} = -1 - \mu'(\xi + 2\eta)  < 0 \]
        und
        \[ \pderive[F]{\eta} = 1 - 2 \mu'(\xi + 2\eta) \geq 1 \]
        Insbsondere sind diese beiden Ableitungen also niemals $0$. Da 
        \[ \opd F = \pderive[F]{\xi}\opd \xi + \pderive[F]{\eta} \opd \eta \]
        und $\opd \xi$ und $\opd \eta$ nur in $p$ gleichzeitig Null sind, haben $f$ 
        und $F$  dieselben kritischen Punkte.

        Betrachte die Region $F^{-1}[c - \varepsilon, c + \varepsilon]$. Wegen 
        Behauptung 1 und der Tatsache, dass $F \leq f$ gilt:
        \[ F^{-1}[c - \varepsilon, c + \varepsilon] \subseteq f^{-1}[c - \varepsilon, c + \varepsilon] \]
        Da $f^{-1}[c - \varepsilon, c + \varepsilon]$ kompakt ist und 
        $F^{-1}[c - \varepsilon, c + \varepsilon]$ abgeschlossen ist, ist 
        $F^{-1}[c - \varepsilon, c + \varepsilon]$ auch kompakt. Da $f$ und $F$
        dieselben kritischen Punkte haben kann diese Menge maximal den kritischen 
        Punkt $p$ enthalten, aber
        \[ F(p) = c - \xi(p) + \eta(p) + \mu(\xi(p) + 2\eta(p)) = c - \mu(0) < c - \varepsilon \]
        Also gibt es in $F^{-1}[c - \varepsilon, c + \varepsilon]$ keine kritischen
        Punkte. Mit dem ersten Deformationslemma gilt dann:
        $F^{-1}(- \infty, c - \varepsilon]$ ist Def. Retrakt von 
        $F^{-1}(-\infty, c + \varepsilon] = M^{c + \varepsilon}$.
    \end{smallproof}

    \begin{claim}
        $M^{c - \varepsilon} \cup e^{k}$ ist ein 
        Deformationsretrakt von $F^{-1}(-\infty, c - \varepsilon]$.
    \end{claim}

    \begin{smallproof}
        Diese Aussage ergibt nur Sinn, falls 
        $M^{c - \varepsilon} \cup e^{k} \subseteq F^(-\infty, c - \varepsilon]$.
        Wir wissen schon, dass $M^{c - \varepsilon} \subseteq F^{-1}(c - \varepsilon]$.

        Sei $q \in e^k$, dann gilt $\xi(p) = 0 \leq \xi(q) \leq 1$ und 
        $\eta(p) = 0 = \eta(q)$. Da 
        $\pderive[F]{\xi} < 0$ gilt dann
        \[ F(q) \leq F(p) < c - \varepsilon \]

        Also ergibt sich folgende Situation:

        \begin{figure}[H]
            \centering
            \includegraphics[width=0.8\linewidth]{../resources/Me-Diagram7-handle.png}
            \label{fig:me-diagram7}
            \caption{Henkel}
        \end{figure}

        Die hellgrün eingefärbte Fläche ist $M^{c - \varepsilon}$ die hellgelbe
        zusammen mit der hellgrünen Fläch ist $F^{-1}(-\infty, c - \varepsilon]$. 

        Dafür konstruieren wir eine Deformationsretraktion
        $r: F^{-1}(-\infty, c - \varepsilon] \times [0,1] \to F^{-1}(-\infty, c - \varepsilon]$
        für $q \in F^{-1}(-\infty, c - \varepsilon], t \in [0, 1]$, die 
        $F^{-1}(-\infty, c - \varepsilon] - M^{c - \varepsilon}$ auf $e^k$ 
        deformiert, wie folgt.

        \[
            r(q, t) = \begin{cases}
                \varphi^{-1} \circ (u_1, ..., u_k, tu_{k + 1}, ..., tu_n)(q)
                    & \text{ im Fall 1: } \xi(q) \leq \varepsilon \\
                \varphi^{-1} \circ (u_1, ..., u_k, s_tu_{k + 1}, ..., s_tu_n)(q)
                    & \text{ im Fall 2: } \varepsilon \leq \xi(q) \leq \eta(q) + \varepsilon \\
                q & \text{ im Fall 3: } \eta(q) + \varepsilon \leq \xi(q)
            \end{cases}
        \]

        Wobei 

        \[ s_t = t + (1 -t)((\xi - \varepsilon)/\eta)^{1/2} \]

        Die Fälle sind dann wie folgt:

        \begin{figure}[H]
            \centering
            \includegraphics[width=0.8\linewidth]{../resources/Me-Diagram9-handle-cases.png}
            \label{me-diagram9}
            \caption{
                Fall 3 ist $M^{c - \varepsilon}$, also die grün eingefärbte Fläche, die
                orangene Fläche ist Fall 1 und die gelbe ist Fall 2.
            }
        \end{figure}

        Wir müssen überprüfen:
        \begin{enumerate}
            \item $r$ ist wohldefiniert und stetig
            \item $r(F^{-1}(-\infty, c - \varepsilon], 0) \subseteq M^{c - \varepsilon} \cup e^k$
            \item $r(\cdot, 1) = \id_{F^{-1}(-\infty, c - \varepsilon]}$ und 
                $\left. r(\cdot , 0) \right\vert_{M^{c - \varepsilon} \cup e^k} 
                = \id_{M^{c - \varepsilon} \cup e^k}$
        \end{enumerate}

        3. ist einfach nachzurechnen. In Fall 1 und Fall 3 ist 2. offensichtlich
        wahr. Für Fall 2 gilt:
        \begin{align*} 
            f(r(0, q)) & = 
                f\left( \varphi^{-1} \left(u_1(q), ..., u_k(q), 
                \left( \frac{\xi(q) - \varepsilon}{\eta(q)} \right)^{1/2}u_{k + 1}(q), ...,
                \left( \frac{\xi(q) - \varepsilon}{\eta(q)} \right)^{1/2}u_n(q)
                \right)
                \right) \\
            & = c - \xi(q)
                + \left( \left( \frac{\xi(q) - \varepsilon}{\eta(q)} \right)^{1/2}u_{k + 1}(q) \right)^2 + ... 
                + \left( \left( \frac{\xi(q) - \varepsilon}{\eta(q)} \right)^{1/2}u_n(q) \right)^2 \\
            & = c - \left( \frac{\xi(q) - \varepsilon}{\eta(q)} \right) \eta(q) \\
            & = c - \varepsilon
        \end{align*}
        also ist $r(0, q) \in f^{-1}(c - \varepsilon)$. Um 1. zu prüfen müssen wir 
        Stetigkeit in den Grenzfällen überprüfen:
        \begin{align*}
            & \text{For } \xi(q) = \varepsilon \text{ : }
                & s_t(q)  =t + (1 - t)((\varepsilon - \varepsilon)/\eta(q))^{1/2} = t \\
            & \text{For } \eta(q) + \varepsilon = \xi(q) \text{ : }
                & s_t(q) = t + (1 - t)((\xi(q) - \varepsilon)/(\xi(q) - \varepsilon))^{1/2} = 1
        \end{align*}

        Das einzig andere Problem was wir bekommen könnten ist nun in Fall 2 falls
        $\eta \to 0$. In Fall 1 und Fall 3 bekommen wir für $q$ mit $\eta(q) = 0$:
        $r(q, t) = \varphi^{-1} \circ (u_1, ..., u_k, 0, ..., 0)(q)$, also wollen
        wir zeigen dass für $\eta \in $ Fall 2 mit $\eta \to 0$ gilt $s_tu_i \to 0$
        für $i \in \{k+1, ..., n\}$. In Fall 2 gilt
        $0 \leq \xi - \varepsilon \leq \eta$. Dann gilt:

        \begin{align*}
            \lim\limits_{\eta \to 0} | s_t u_i |
            & = \lim\limits_{\eta \to 0} (1 - t)((\xi - \varepsilon)/\eta)^{1/2} | u_i | \\
            & \leq \lim\limits_{\eta \to 0} (1 - t)(\eta/\eta)^{1/2}|u_i| \\
            & = \lim\limits_{\eta \to 0} (1 - t)|u_i| = 0 
        \end{align*}
        
        Also ist $r$ stetig.
    \end{smallproof}

    Mit Behauptung 3 und 4 bekommen wir
    \[ M^{c + \varepsilon} \simeq F^{-1}(c - \varepsilon] \]
    und 
    \[ F^{-1}(-\infty, c - \varepsilon] \simeq M^{c - \varepsilon} \cup e^k \]
    Also folgt die Behauptung:
    \[ M^{c + \varepsilon} \simeq M^{c - \varepsilon} \cup e^k \]

\end{bigproof}
