\chapter{Einführung}

Anschauliche Beispiele, vielleicht die zu den Deformations-Lemmata? Dann müsste
ich aber auch noch die Deformations-Lemmata machen.

\section{Höhenfunktionen}

\section{Definitionen und Lemmata}

Dieser Abschnitt folgt dem gleichnamigen Kapitel in \cite{milnor}.

\begin{definition}[Kritischer Punkt]
    \label{def: kritischer Punkt}
    Sei $M$ eine glatte Mannigfaltigkeit und $f \colon M \to \R$ eine glatte Abbildung. 
    Ein \textit{kritischer Punkt} von $f$ ist ein Punkt $p \in M$, sodass $\opd f (p) = 0$.
\end{definition}

\begin{remark}
    Allgemeiner lassen sich kritische Punkte von glatten Abbildungen $f \colon M \to N$ 
    definieren, siehe im Anhang %~\ref{def: kritischer Punkt allgemein}
\end{remark}

Wir würden gerne eine Hessische Bilinearform für die Tangentialräume der Mannigfaltigkeit
definieren, allerdings ist dies ein nicht ganz einfaches Unterfangen. Wir werden am Ende
einen Begriff erhalten, der mit dem der gewohnten Hessischen Bilinearform im $\R^n$
übereinstimmt, allerdings nur für kritische Werte definiert ist.

\begin{definition}[Lie-Klammer]
    \label{def: lie-klammer}
    Es seien $X$ und $Y$ Vektorfelder auf einer glatten Mannigfaltigkeit $M$. Die 
    \textit{Lie-Klammer} ist die Abbildung 
    \begin{align*} 
        [\cdot, \cdot] \colon \VFs(M) \times \VFs(M) & \to \VFs(M) \\
        (X, Y) & \mapsto [X, Y] := XY - YX
    \end{align*}
    Wobei 
    \[ (XY - YX) (p) (f) = X(p)(Y(\cdot)(f)) - Y(p)(X(\cdot)(f)) \]
\end{definition}

\begin{remark}
    Es ist leicht nachzurechnen, dass die Lie-Klammer tatsächlich eine \\ Lie-Klammer ist,
    also dass sie folgende Eigenschaften erfüllt:
    \begin{itemize}
        \item $[\cdot, \cdot]$ ist bilinear.
        \item $[X, Y] = -[Y, X]$
        \item $[X, [Y, Z]] + [Z, [X, Y]] + [Y, [Z, X]] = 0$
    \end{itemize}
\end{remark}

\begin{prop}
    \label{prop: lie-klammer ist null}
    Es sei $f \colon M \to \R$ glatt, $p$ ein kritischer Punkt von $f$, 
    $X, Y \in \VFs (M)$. Dann gilt:
    \[ [X, Y] (p) f = 0 \]
\end{prop}

\begin{proof}
    Es seien $(x_1, ..., x_n)$ lokale Koordinaten um $p$. Wir können ohne beschränkung
    der Allgemeinheit annehmen, dass $X = g_X \cdot \pderive{x_i}$ und 
    $Y = g_Y \cdot \pderive{x_j}$ für $g_X, g_Y \in C^{\infty} (M)$. Dann gilt:
    \begin{align*}
        \left[g_X \cdot \pderive{x_i}, g_Y \cdot \pderive{x_j}\right] (p) (f) = & 
            g_X (p) \cdot \pderive{x_i} (p) \left(g_Y \cdot \pderive[f]{x_j} \right) -
            g_Y (p) \cdot \pderive{x_j} (p) \left(g_X \cdot \pderive[f]{x_i} \right) \\
        = & g_X (p) \cdot \left( \pderive[g_Y]{x_i}(p) \cdot \pderive[f]{x_j}(p) + 
                g_Y(p) \cdot \pdderive[f]{x_j}{x_i}(p) \right) \\ 
        & - g_Y (p) \cdot \left( \pderive[g_X]{x_j}(p) \cdot \pderive[f]{x_i}(p) + 
            g_X (p) \cdot \pdderive[f]{x_i}{x_j}(p) \right) \\
        = & \; 0
    \end{align*}
    Der letzte Ausdruck ist Null wegen des Satzes von Schwarz und da $p$ ein kritischer
    Punkt von $f$ ist.
\end{proof}

\begin{definition}[Hessische Bilinearform]
    Es sei $f \colon M \to \R$ eine glatte Abbildung, $p$ ein kritischer Punkt von $f$.
    Es seien $x, y \in T_pM$. Wähle $X, Y \in \VFs (M)$, sodass $X(p) = x$ und 
    $Y(p) = y$. Definiere nun
    \[ \opd^2 f (x, y) (p) = X(p)(Y(\cdot)f). \]
    $\opd^2 f (\cdot, \cdot) (p)$ heißt \textit{Hessische Bilinearform}. 
\end{definition}

\begin{prop}
    $\opd^2 f (p)$ hängt nicht von den gewählten Vektorfeldern $X$ und $Y$ ab und ist
    für alle kritischen Punkte eine symmetrische Bilinearform.
\end{prop}

\begin{proof}
    Bilinearität folgt direkt aus der Definition.
    Da $p$ ein kritischer Punkt ist gilt 
    \[ \opd^2 f (x, y) (p) - \opd^2 f (y, x) (p) = [X, Y] (p) (f) = 0, \]
    die Zuordnung ist also symmetrisch. Außerdem gilt
    \[ XY f (p) = X(p) (Y(\cdot) f) = x(Y(\cdot) f), \]
    also hängt die Form nicht von $X$ ab, und wegen der Symmetrie auch nicht von $Y$.
\end{proof}

\begin{definition}[nicht-degeneriertheit, Index]
    Es sei $f \colon M \to \R$ eine glatte Abbildung, $p$ ein kritischer Punkt von
    $f$. Wir nennen $p$ \textit{nicht degeneriert}, falls die Bilinearform 
    $\opd^2 f (\cdot, \cdot) (p)$ nicht ausgeartet ist. Der \textit{Index} eines
    nicht degenerierten kritischen Punktes ist die maximale Dimension eines
    Untervektorraumes, auf dem $\opd^2 f (\cdot, \cdot) (p)$ negativ definit ist.
\end{definition}

\begin{remark}
    Nicht-Degeneriertheit und Index lassen sich auch über lokale Koordinaten definieren,
    aber nachzurechnen, dass diese Begriffe wohldefiniert sind ist recht aufwändig.
    Trotzdem wollen wir diese Sichtweise nicht vorenthaltern:

    Es seien $\phi = (x_1, ..., x_n)$ lokale Koordinaten um den kritischen Punkt $p$. 
    Dann ist $\mathcal{B} = \left(\pderive{x_1}, ..., \pderive{x_n}\right)$ eine Basis des
    Vektorraums $T_pM$. Wir bekommen
    \[ 
        \opd^2 f \left( \pderive{x_i}, \pderive{x_j} \right) (p) 
        = \pderive{x_i}(p) \left( \pderive[f]{x_j} \right) 
        = \pdderive[f]{x_j}{x_i} (p).
    \]
    Dann ist $p$ nicht degeneriert genau dann wenn die Matrix
    \[ H^\phi_p(f) = \left( \pdderive[f]{x_i}{x_j} \right)_{1 \leq i, j \leq n} \]
    invertierbar ist. Der Index von $p$ ist dann die Anzahl der negativen Eigenwerte
    von $H^\phi_p(f)$. Der Index und die nicht-degeneriertheit hängen offensichtlich
    nicht von den gewählten Koordinaten ab, aber die Matrix $H_p^{\phi}(f)$ schon.
\end{remark}

\begin{remark}
    Die beiden Begriffe Index und nicht-Degeneriertheit sind zentral in der Morse-Theorie 
    und werden uns über die gesamte Arbeit begleiten. Auch der nachfolgende Satz wird in 
    fast jedem Beweis genutzt:
\end{remark}

\begin{theorem}[Morse-Lemma]
    \label{satz: morse-lemma}
    Es sei $p$ ein nicht degenerierter kritischer Punkt mit Index $k$ einer glatten 
    Funktion $f \colon M \to \R$. Dann existieren lokale koordinaten 
    $\varphi = (x_1, ..., x_n)$, sodass in einer Umgebung $U$ von $p$ gilt:
    \[ f = f(p) - x_1^2 - ... - x_k^2 + x_{k + 1}^2 + ... + x_n^2 \]
    und 
    \[ \varphi (p) = 0. \]
\end{theorem}

\begin{proof}
    
\end{proof}