\section{Topologische Eigenschaften anhand kritischer Punkte}
\label{sec: topologische eigenschaften anhand kritischer punkte}

Wir wollen die beiden Deformations-Lemmata beweisen, die eine Verbindung zwischen den topologischen
Eingeschaften einer Mannigfaltigkeit und den kritischen Punkten einer Morse Funktion herstellt.
Ist $f \colon M \to \R$ eine glatte Abbildung, dann ist die Subniveaumenge von $f$ bezüglich 
einer reellen Zahl $c$ die Menge $M^c := f^{-1}(- \infty, c]$. Sind $a < b$ reelle Zahlen, dann
stellen die Deformationslemmata die Topologien von $M^a$ und $M^b$ in Relation: Das erste beschreibt, 
was passiert \textit{wenn kein} kritischer Wert überschritten wird, und das zweite beschreibt, was 
passiert \textit{wenn ein} kritischer Wert überschritten wird. Die Beweise folgen Milnors Buch 
\cite{morse}. \\

\begin{theorem}[Erstes Deformationslemma]
    \label{satz: erstes deformationslemma}
    Es sei $M$ eine glatte Mannigfaltigkeit und $f: M \rightarrow \R$ eine
    glatte Abbildung. Hat $f$ keine kritischen Werte im Intervall $[a, b]$ und 
    ist $f^{-1}[a, b]$ kompakt, so existiert ein Diffeomorphismus 
    $M^b \rightarrow M^a$, und $M^a$ ist ein Deformationsretrakt von $M^b$.
\end{theorem}

Die Idee des Beweises ist es, $M^b$ entlang eines Pseudo-Gradientenfeldes von $f$ mit einem 
Diffeomorphismus $\varphi$ "nach unten zu ziehen", bis $\varphi(f^{-1}(b)) = f^{-1}(a)$.

\begin{proof}
    Sei $Y$ ein Pseudo-Gradientenfeld von $f$. Es existiert eine kompakte Umgebung $K \in M$ von 
    $f^{-1}[a, b]$, in der keine kritischen Punkte enthalten sind, das folgt zum Beispiel aus 
    Whitneys Einbettungssatz (siehe \cite{whitney}) und dem Satz von Heine-Borel. 
    Sei $\rho: M \to \R$ eine glatte, 
    positive Funktion, sodass
    \[ \rho(p) = - \frac{1}{\opd f (p) (Y (p))} \]
    für alle $p \in f^{-1}[a, b]$ und die außerhalb von $K$ verschwindet und für
    die für alle $p \in K$, die keine kritischen Punkte sind, gilt: 
    \[ 0 \leq \rho(p) \leq - \frac{1}{\opd f (p) (Y (p))} . \]
    Bemerke dass $\rho$ innerhalb von $K$ wohldefiniert ist, da sich keine kritischen Punkte in $K$ 
    befinden. Definiere ein Vektorfeld $X$ durch
    \[ X(p) = \rho(p) \cdot Y (p) \]
    Dann hat $X$ kompakten Träger, erfüllt also die Vorraussetzungen von 
    Lemma~\ref{prop: kompaktes VF generiert 1-param. grp.}. Sei also $\varphi$ die
    einzigartige 1-Parameter Gruppe aus Diffeomorphismen, die von $X$ generiert
    wird. 
    Wir bekommen für jedes $p \in M$ eine Abbildung 
    $f \circ \varphi_{\bullet}(p): \R \to \R$.
    
    \begin{claim*} 
        Für alle $p \in M$, $t_0 \in \R$ und für $q = \varphi_{t_0}(q)$
        ist $\derive{t} (f \circ \varphi_{\bullet}(p)) (t_0) \in [-1, 0]$ und falls 
        $f(\varphi_t(q)) \in [a, b]$ gilt sogar $\derive{t} f \circ \varphi_{\bullet}(q) (t_0) = -1$.
    \end{claim*}

    \begin{smallproof}
        Für $q = \varphi_{t_0}(p)$:
        \begin{align*}
            \derive{t} f \circ \varphi_{t_0}(p)
            & = \opd f (\varphi_{t_0}(p)) \cdot \opd \varphi_{\bullet}(p) (t_0) 
                \left( \derive{t} \right)
            = \opd f (q) ( X(q) ) \in [-1, 0]
        \end{align*}
        $f \circ \varphi_{\bullet}(p)$ ist also monoton wachsend für alle $p \in M$.
        Falls sogar $f(\varphi_p(t_0)) \in [b, a]$, dann gilt
        \[ \frac{d}{dt} f \circ \varphi^p (t_0) = -1 \]
    \end{smallproof}

    Man zeigt dann leicht, dass für $p \in f^{-1}(b)$, $t_0 \in [0, b - a]$ gilt 
    $f(\varphi_{t_0}(p)) \in [a, b]$.

    Dann ist für $p \in f^{-1}(b)$ die Abbildung $f \circ \phi_{\bullet}(p)$ im Intervall $
    [0, b - a]$ linear mit Steigung $- 1$ und es gilt 
    \[ f(\varphi_{b - a}(p)) = f(\varphi_{0}(p)) - (b - a) = a \]
    Genauso für $q \in f^{-1}(a)$: $f(\varphi_{a - b}(q)) = b$, also 
    $\varphi_{b - a}(f^{-1}(a)) = f^{-1}(b)$.

    Dann haben wir $\varphi_{b - a} (M^b) = M^a$, also ist $\varphi_{b - a}|_{M^b}$ ein 
    Diffeomorphismus zwischen $M^a$ und $M^b$. 

    Betrachte nun $r: M^b \times \R \to M^a$,
    \[  
        r(p, t) = \begin{cases}
            p & \text{ falls }  f(p) \leq a \\
            \varphi_{t(f(p) - a)}(p) & \text{ falls } a \leq f(p) \leq b 
        \end{cases}
    \]

    $r$ ist stetig, $r(\cdot, 0)$ ist die Identität auf $M^b$, $r(\cdot, 1)|_{M^a}$ ist die 
    Identität auf $M^a$ und \\ $r(1, M^b) \subseteq M^a$, also ist $M^a$ ein Deformationsretrakt 
    von 
    $M^b$.
\end{proof}

\begin{theorem}[Zweites Deformations-Lemma]
    \label{satz: zweites deformationslemma}
    Es sei $M$ eine glatte Mannigfaltigkeit, $f: M \rightarrow \R$ eine glatte
    Abbildung und $p$ ein nicht-degenerierter kritischer Punkt mit Index 
    $k$. Sei $c := f(p)$ und $\varepsilon \geq 0$, sd. 
    $f^{-1}[c - \varepsilon, c + \varepsilon]$ kompakt ist und außer $p$ keine 
    weiteren kritischen Punkte von $f$ beinhaltet. Dann hat $M^{c-\varepsilon}$
    denselben Homotopietypen wie $M^{c - \varepsilon} \cup e^k$.
\end{theorem}

\begin{proof}
    Die Idee für den Beweis ist, sich eine neue Funktion $F: M \to \R$ zu definieren,
    die Außerhalb von einer kleinen Umgebung von $p$ $f$ entspricht und in der 
    Umgebung etwas kleiner ist. Dann bekommen wir die folgende Situation:

    \begin{figure}[H]
        \centering
        \subfloat{
            \includegraphics[width=0.5\linewidth]{../resources/Me-Diagram5-sublevelsets-of-f-and-F.jpeg}
        } \\
        \subfloat{
            \begin{tikzpicture}
                % Include the image in a node
                \node [
                    above right,
                    inner sep=0] (image) at (0,0) {\includegraphics[width=0.3\textwidth]{../resources/subniveaumengen-parametrisiert.jpeg}};
                    
                % Create scope with normalized axes
                \begin{scope}[
                x={($0.02*(image.south east)$)},
                y={($0.05*(image.north west)$)}]

                \node at (12.5, 9.5) {\tiny $p$};
                \node[text=white] at (38.5, 9.55) {\tiny $p$};
                
                %\draw[lightgray,step=1] (image.south west) grid (image.north east);
                \end{scope}
            \end{tikzpicture}
        }
        \caption{die Subniveaumengen von $f$ (links) und $F$ (rechts)}
    \end{figure}

    Wir wollen also, dass $M^{c + \varepsilon} = F^{-1}(- \infty, c + \varepsilon]$ 
    gilt und $F^{-1}(-\infty, c - \varepsilon]$ fast dasselbe ist wie 
    $M^{c - \varepsilon}$, nur dass $F^{-1}(-\infty, c - \varepsilon]$ einen \say{Henkel}
    enthält der den kritischen Punkt $p$ enthält.

    Wir benutzen das Morse-Lemma~\ref{satz: morse-lemma}:
    Wir finden lokale Koordinaten $\phi = (u_1, \dots, u_n)$ in einer Umgebung $U$ von $p$, sodass 
    \[ f = c - u_1 - \dots - u_k + u_{k + 1} + \dots + u_n
     \text{ und } u_1 (p) = \dots = u_n(p) = 0 . \]
    Sei $\eps$ ohne Beschränkung der Allgemeinheit klein genug, sodass $f^{-1}[c - \eps, c + \eps]$ 
    kompakt ist und die Kreisscheibe $\{ x \in \R^n : \| x \|^2 \leq 2 \eps \}$ im Bild von $\phi$ 
    enthalten ist. Wähle nun die $k$-Zelle 
    \[ e^k := \{ q \in M : u_1^2 (q) + \dots + u_k^2 (q) \leq \eps \text{ und }
        u_{k + 1} (q) = \dots = u_n (q) = 0 \} . \]
    Wir bekommen folgende Situation:
    
    % \begin{figure}[H]
    %     \centering
    %     % \label{fig:me-diagram6}
    %     \begin{tikzpicture}
    %         % Include the image in a node
    %         \node [
    %             above right,
    %             inner sep=0] (image) at (0,0) {\includegraphics[width=0.4\textwidth]{../resources/subniveaumengen-mit-zelle.jpeg}};
                
    %         % Create scope with normalized axes
    %         \begin{scope}[
    %         x={($0.04*(image.south east)$)},
    %         y={($0.05*(image.north west)$)}]

    %         \node at (2, 5.1) {\tiny $- \sqrt{\eps}$};
    %         \node at (2, 10) {\tiny $0$};
    %         \node at (2, 14.9) {\tiny $\sqrt{\eps}$};

    %         \node at (7.5, 1.2) {\tiny $- \sqrt{\eps}$};
    %         \node at (12.5, 1.2) {\tiny $0$};
    %         \node at (17.1, 1.2) {\tiny $\sqrt{\eps}$};
            
    %         \node[text=white] at (38.5, 9.55) {\tiny $p$};
    %         \node[text=white] at (13, 9.5) {\tiny $p$};
    %         \node[text=white] at (9, 9.5) {\tiny $e^k$};
            
    %         %\draw[lightgray,step=1] (image.south west) grid (image.north east);
    %         \end{scope}
    %     \end{tikzpicture}
    %     \caption{some label}
    % \end{figure}

    \begin{figure}[H]
        \centering
        \includegraphics[width=0.3\textwidth]{../resources/subniveaumengen-parametrisiert2.png}
    \end{figure}

    $F$ wird nun so definiert, dass eine Umgebung der $k$-Zelle $e^k$ ein wenig abgesenkt wird.
    Sei $\mu \colon \R \to \R$ eine glatte Funktion mit den Eigenschaften
    \begin{enumerate}
        \item $\mu(0) > \eps$
        \item $\mu (r) = 0$ falls $r \geq 2 \eps$
        \item $-1 < \mu' (r) \leq 0$ für alle $r \in \R$ .
    \end{enumerate}
    Sei dann $F$ außerhalb von $U$ gleich $f$, und innerhalb von $U$ setze
    \[ F = f - \mu ( u_1^2 + \dots + u_k^2 + 2 u_{k + 1}^2 + \dots + 2 u_n^2) . \]
    Man überprüft dann die drei Behauptungen
    \begin{enumerate}
        \item $F^{-1}(- \infty, c + \eps] = M^{c + \eps}$,
        \item $F^{-1}(- \infty, c - \eps]$ ist Deformationsretrakt von $M^{c + \eps}$,
        \item $M^{c - \eps} \cup e^k$ ist Deformationsretrakt von $F^{-1}(- \infty, c - \eps]$.
    \end{enumerate}
    Die zweite Behauptung folgt aus dem ersten 
    Deformationslemma~\ref{satz: erstes deformationslemma}.
    
    Den detaillierten Beweis findet man zum Beispiel bei Milnor \cite{milnor}.
\end{proof}

Die beiden Deformationslemmata sind genau die beiden Aussagen, die wir am Anfang des Kapitels schon
als Vermutung aufgestellt haben!

Mit den beiden Aussagen kommt man schon sehr weit. Die berühmten Morse-\\Ungleichungen 
(Satz~\ref{satz: morse-ungleichungen}) lassen
sich mit ein bisschen linearer Algebra an exakten Sequenzen leicht daraus folgern. Man kann sogar
zeigen, dass jede Mannigfaltigkeit den Homotopietypen eines \textit{CW-Komplexes}
(siehe Def.~\ref{def: cw-komplex}) besitzt, wie es zum Beispiel Milnor  in \cite{milnor} tut. Die 
folgenden Kapitel geben eine etwas eleganterere Sicht auf die Thematik, die in gewisser Hinsicht 
stärkere Aussagen liefert:

\textit{Kompakte} Mannigfaltigkeiten haben nicht nur den selben Homotopie-Typen eines CW-Komplexes, 
sondern sie sind sogar CW-Komplexe, und wir können uns die zelluläre Homologie, die aus der 
CW-Struktur hervorgeht sogar auf eine einfache Art und Weise erklären 
(siehe~\ref{prop: cw-zerlegung}).