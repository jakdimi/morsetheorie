\section{Vektorfelder und Pseudo-Gradienten}

Wir untersuchen erst ein Paar Eigenschaften von Vektorfeldern.

\begin{definition}[Flusslinie]
    \label{def: flussliene}
    Es sei $I \subseteq \R$ ein Intervall, $M$ eine glatte Mannigfaltigkeit und  
    $\gamma \colon I  \to M$ ein glatter Weg. Dann definiere für $t_0 \in \R$
    \[ \derive[\gamma]{t} (t_0) := 
        \opd \gamma (t_0) \left( \pderive{t} \right) \in T_{\gamma(t_0)}M \]
    wobei $\pderive{t}$ das von der Indentität auf $\R$ induziertze Element in $T_t\R$ ist.

    Es sei $X \in \VFs (M)$ ein Vektorfeld auf $M$. $\gamma$ heißt Flusslinie von $X$
    falls für alle $t_0 \in \R$ gilt: 
    \[ \derive[\gamma]{t}(t_0) = X(\gamma(t_0)) . \]
    Das Bild einer Flusslinie von $X$ heißt \textit{Trajektorie}. Wir benutzen den Begriff 
    Trajektorie recht frei. Manchmal ist damit auch die Flusslinie gemeint.
\end{definition}

\begin{definition}[1-Parameter Gruppe aus Diffeomorphismen]
    \label{def: 1-parameter gruppe aus diffeos}
    Es sei $M$ eine glatte Mannigfaltigkeit. Eine 
    \textit{1-Parameter Gruppe aus Diffeomorphismen} ist eine glatte Abbildung
    \begin{align*}
        \phi \colon \R \times M \longto & \; M \\
        (t, p) \longmapsto & \; \phi_t(p)
    \end{align*}
    sodass gelten: 
    \begin{itemize}
        \item Für alle $s, t \in \R$ gilt $\phi_{s + t} = \phi_s \circ \phi_t$ und
        \item $\phi_0 = \id_{M}$.
    \end{itemize}

    Für eine 1-Parameter Gruppe aus Diffeomorphismen $\phi$ schreiben wir 
    \[ \varphi_{\bullet}(p): \R \to M ; t \mapsto \varphi_t(p) . \]

    Es sei $X \in \VFs (M)$. Eine 1-Parameter Gruppe aus Diffeomorphismen $\phi$ heißt 
    \textit{von $X$ erzeugt}, falls für alle $p \in M$ gilt:
    \[ X(p) = \derive[\phi_{\bullet}(p)]{t}(0) \]
\end{definition}

\begin{remark}
    Wie der Name suggestiert, ist für jedes $t \in \R$ die Abbildung $\phi_t$ ein 
    Diffeomorphismus: Das Inverse von $\phi_t$ ist $\phi_{-t}$.

    Ist außerdem $\phi$ eine von einem Vektorfeld $X$ erzeugte 1-Parameter Gruppe aus 
    Diffeomorphismen, dann rechnet man leicht nach, dass $\phi_{\bullet}(p)$ Flusslinien 
    von $X$ sind.
\end{remark}

\begin{prop}
    \label{prop: kompaktes VF generiert 1-param. grp.}
    Es sei $M$ eine glatte Mannigfaltigkeit, $X \in \VFs (M)$ mit kompaktem Träger. Dann 
    generiert $X$ eine eindeutige 1-Parameter Gruppe aus Diffeomorphismen.
\end{prop}

\begin{proof}
    Einen Beweis für diese Aussage findet man zum Beispiel auch in \cite{milnor}.
\end{proof}

\begin{remark}
    Falls $M$ eine kompakte Mannigfaltigkeit ist, dann generieren alle Vektorfelder eindeutige 
    1-Parametergruppen aus Diffeomorphismen.
\end{remark}

% \begin{definition}[Riemannsche Metrik]
%     \label{def: riemannsche metrik}
%     Es sei $M$ eine Mannigfaltigkeit. Es sei 
%     \[ g_p \colon T_pM \times T_pM \longto T_pM \]
%     ein Skalarprodukt für jedes $p \in M$, sodass für alle $X, Y \in \VFs (M)$ die Abbildung 
%     \[ p \longmapsto g_p(X(p), Y(p)) \]
%     glatt ist. Dann heißt $g$ \textit{Riemmannsche Metrik} auf $M$. Wir schreiben für 
%     $x, y \in T_pM$ 
%     \[ \langle x, y \rangle := g_p(x, y) \text{ und } \| x \| := \sqrt{g_p(x, x)} . \]
% \end{definition}

% \begin{remark}
%     Man kann zeigen, dass alle Mannigfaltigkeiten eine Riemannsche Metrik besitzen.
% \end{remark}

% \begin{definition}[Gradient]
%     \label{def: gradient}
%     Es sei $M$ eine glatte Mannigfaltigkeit, $f \colon M \to \R$ eine glatte Abbilding. Dann 
%     ist der Gradient von $f$ das eindeutige Vektorfeld $\grad f$, sodass für alle $X \in \VFs (M)$
%     gilt 
%     \[ \langle X , \grad f \rangle = \opd f X . \]
% \end{definition}

\begin{definition}[Pseudo-Gradient]
    \label{def: pseudo-gradient}
    Es sei $M$ eine Mannigfaltigkeit, $f \colon M \to \R$ eine glatte Funktion. $X \in \VFs (M)$
    heißt \textit{Pseudo-Gradient} oder \textit{Pseudo-Gradientenfeld} von $f$, falls gelten:
    \begin{itemize}
        \item $\opd f (p) (X(p)) \leq 0$ für alle $p \in M$, mit Gleichheit genau dann wenn 
            $p$ ein kritischer Punkt von $f$ ist.
        \item Für jeden kritischen Punkt $p$ von $f$ existiert eine Morse-Umgebung 
            $(U_p, \phi_p)$, in der $X (q) = 
                - \opd (\phi_p^{-1}) (q) \cdot \grad (f \circ \phi_p^{-1})$.
    \end{itemize}
    $\grad$ ist hier der standard Gradient auf $\R^n$, der vom standard Skalarprodukt auf $\R^n$ 
    induziert wird.
\end{definition}

Die erste Eigenschaft haben auch Gradientenfelder. Tatsächlich ist dies die einzige
Eigenschaft des Gradienten, die uns in diesem Kontext interessiert.

Die zweite Eigenschaft stellt sicher, dass sich Pseudo-Gradientenfelder in der nähe von 
kritischen Punkten, also den Punkten, die wir in dieser Arbeit untersuchen, einfach verhält, 
nämlich genau wie durch das Morse-Lemma~\ref{satz: morse-lemma} bestimmt.

\begin{prop}
    Es sei $M$ eine Mannigfaltigkeit und $f \colon M \to \R$ eine Morse-Funktion. 
    Dann existiert ein Pseudo-Gradientenfeld von $f$.
\end{prop}

\begin{proof}
    Den Beweis findet man bei Audin und Damian \cite{audin}.
\end{proof}