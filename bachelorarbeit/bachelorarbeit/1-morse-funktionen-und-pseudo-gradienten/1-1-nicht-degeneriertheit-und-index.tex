\section{Nicht-Degeneriertheit und Index}

Dieser Abschnitt folgt dem ersten Kapitel aus \cite{milnor}. Wir werden Grundlegende Begriffe der
Morse-Theorie erklären und das \textit{Morse-Lemma}~\ref{satz: morse-lemma} beweisen.

\begin{definition}[Kritischer Punkt]
    \label{def: kritischer Punkt}
    Sei $M$ eine glatte Mannigfaltigkeit und $f \colon M \to \R$ eine glatte Abbildung. 
    Ein \textit{kritischer Punkt} von $f$ ist ein Punkt $p \in M$, sodass $\opd f (p)$ nicht 
    surjektiv ist.
    Die Menge der kritischen Punkte von $f$ heißt $\Crit (f)$.
\end{definition}

\begin{remark}
    Wir werden uns ausschließlich mit glatten Abbildungen $f \colon M \to \R$ beschäftigen.
    bei solchen Abbildungen ist $p \in M$ ein kritischer Punkt genau dann, wenn 
    $\opd f (p) = 0$.
\end{remark}

Wir würden gerne eine Hessesche Bilinearform für die Tangentialräume der Mannigfaltigkeit
definieren, allerdings ist dies ein nicht ganz einfaches Unterfangen. Wir werden am Ende
einen Begriff erhalten, der mit dem der gewohnten Hesseschen Bilinearform im $\R^n$
übereinstimmt, allerdings nur für kritische Werte definiert ist.

\begin{definition}[Hessesche Bilinearform]
    Es sei $f \colon M \to \R$ eine glatte Abbildung, $p$ ein kritischer Punkt von $f$.
    Es seien $x, y \in T_pM$. Wähle $X, Y \in \VFs (M)$, sodass $X(p) = x$ und 
    $Y(p) = y$. Definiere nun
    \[ \opd^2 f (x, y) (p) = X(p)(Y(\cdot)f). \]
    $\opd^2 f (\cdot, \cdot) (p)$ heißt \textit{Hessesche Bilinearform}. 
\end{definition}

\begin{prop}
    \label{prop: hessesche ist sym bilinearform}
    $\opd^2 f (\cdot, \cdot) (p)$ hängt nicht von den gewählten Vektorfeldern $X$ und $Y$ ab 
    und ist für alle kritischen Punkte eine symmetrische Bilinearform.
\end{prop}

\begin{proof}
    Wir betrachten die so genannte \textit{Lie-Klammer} für Vektorfelder $X$ und $Y$:
    \[ [X, Y] = XY - YX , \]
    wobei
    \[ (XY - YX) (p) (f) = X(p)(Y(\cdot)(f)) - Y(p)(X(\cdot)(f)) . \]
    Man rechnet nach, dass die Lie-Klammer symmetrisch ist und das für jeden kritischen Punkt $p$
    von $f$ gilt $[X, Y](p)(f) = 0$.
    Bilinearität folgt dann direkt aus der Definition. Da $p$ ein kritischer Punkt ist gilt 
    \[ \opd^2 f (x, y) (p) - \opd^2 f (y, x) (p) = [X, Y] (p) (f) = 0, \]
    die Zuordnung ist also symmetrisch. Außerdem gilt
    \[ XY f (p) = X(p) (Y(\cdot) f) = x(Y(\cdot) f), \]
    also hängt die Form nicht von $X$ ab, und wegen der Symmetrie auch nicht von $Y$.
\end{proof}

\begin{definition}[nicht-degeneriert, Index]
    \label{def: nicht-degeneriert u index}
    Es sei $f \colon M \to \R$ eine glatte Abbildung, $p$ ein kritischer Punkt von
    $f$. Wir nennen $p$ \textit{nicht degeneriert}, falls die Bilinearform 
    $\opd^2 f (\cdot, \cdot) (p)$ nicht ausgeartet ist. Der \textit{Index} eines
    nicht degenerierten kritischen Punktes ist die maximale Dimension der
    Untervektorräume, auf denen $\opd^2 f (\cdot, \cdot) (p)$ negativ definit ist.
    Die Menge der kritischen Punkte von $f$ mit Index $k$ heißt $\Crit_k (f)$.
\end{definition}

\begin{remark}
    Nicht-Degeneriertheit und Index lassen sich auch über lokale Koordinaten definieren.
    Tatsächlich stimmt die Sichtweise über lokale Koordinaten mit der Vorstellung der Hesseschen-
    Bilinearform, die wir für Abbildungen $\R^n \to \R$ haben überein:

    Es seien $\phi = (x_1, ..., x_n)$ lokale Koordinaten um den kritischen Punkt $p$. 
    Dann ist $\mathcal{B} = \left(\pderive{x_1}, ..., \pderive{x_n}\right)$ eine Basis des
    Vektorraums $T_pM$. Wir bekommen
    \[ 
        \opd^2 f \left( \pderive{x_i}, \pderive{x_j} \right) (p) 
        = \pderive{x_i}(p) \left( \pderive[f]{x_j} \right) 
        = \pdderive[f]{x_j}{x_i} (p).
    \]
    Dann ist $p$ nicht degeneriert genau dann wenn die Matrix
    \[ H^\phi_p(f) = \left( \pdderive[f]{x_i}{x_j} \right)_{1 \leq i, j \leq n} \]
    invertierbar ist. Der Index von $p$ ist dann die Anzahl der negativen Eigenwerte
    von $H^\phi_p(f)$. Der Index und die nicht-degeneriertheit hängen offensichtlich
    nicht von den gewählten Koordinaten ab, aber die Matrix $H_p^{\phi}(f)$ schon.
\end{remark}

Die beiden Begriffe Index und nicht-Degeneriertheit sind zentral in der Morse-Theorie 
und werden uns über die gesamte Arbeit begleiten. Auch der nachfolgende Satz wird in 
fast jedem Beweis genutzt:

\begin{theorem}[Morse-Lemma]
    \label{satz: morse-lemma}
    Es sei $p$ ein nicht degenerierter kritischer Punkt mit Index $k$ einer glatten 
    Funktion $f \colon M \to \R$. Dann existieren lokale koordinaten 
    $\phi = (x_1, ..., x_n)$, sodass in einer Umgebung $U$ von $p$ gilt:
    \[ f = f(p) - x_1^2 - ... - x_k^2 + x_{k + 1}^2 + ... + x_n^2 \]
    und 
    \[ \varphi (p) = 0. \]
    $(U, \phi)$ heißt \textit{Morse-Karte}, und $U$ \textit{Morse-Umgebung}.
\end{theorem}

Der hier geführte Beweis für das Morse-Lemma ist in \cite{hirsch} zu finden. 
Bevor wir das Morse Lemma beweisen, benötigen wir eine Aussage aus der Linearen Algebra:

\begin{lemma}
    \label{lemma: lina lemma}
    Es sei $A = \diag(a_1, ..., a_n)$ eine diagonale $n \times n$ Matrix mit 
    Diagonaleinträgen $\pm 1$. Dann gibt es eine Umgebung $N$ von $A$ im Vektorraum der 
    symmetrischen $n \times n$ Matrizen und eine glatte Abbildung 
    $P \colon N \to GL_n(\R)$, sodass $P(A) = E_n$ und falls $P(B) = Q$, dann gilt 
    $Q^TBQ = A$.
\end{lemma}

\begin{proof}
    Die Aussage kann man Induktiv beweisen, siehe \cite{hirsch}.
\end{proof}

\begin{proof}[Beweis von Satz~\ref{satz: morse-lemma}]
    Es sei $U$ eine Karten Umgebung von $p$. Dann können wir ohne Beschränkung der 
    Allgemeinheit annehmen, dass $f \colon \R^n \to \R$, $p = 0$ und $f(0) = 0$.
    Außerdem können wir mithilfe eines Korrdinatenwechsels annehmen, dass 
    \[ A = H_0(f) \]
    eine Diagonalmatrix mit ausschließlich Diagonaleinträgen $\pm 1$ hat, denn da $p$
    nicht degeneriert ist ist $A$ invertierbar. 
    \begin{claim*}
        Es existiert eine glatte Abbildung $x \mapsto B_x$ von $M$ in die symmetrischen
        $n \times n$ Matrizen, sodass für $B_x = (b_{ij}(x))_{ij}$ gilt 
        \[ f(x) = \sum_{i, j = 1}^n b_{ij}(x) x_i x_j , \]
        und sodass $B_0 = A$. 
    \end{claim*}
    \begin{smallproof}
        Da $f(0) = 0$ bekommen wir mit dem Fundamentalsatz der 
        Differenzial - und Integralrechnung: 
        \begin{align*}
            f(x) = & f(x) - f(0) 
                = \int_0^1 \derive[f (tx)]{t} \opd t \\
            = & \int_0^1 \sum_{i = 1}^n \pderive[f]{x_i}(tx) x_i \opd t 
                = \sum_{i = 1}^n \left( \int_0^1 \pderive[f]{x_i}(tx) \opd t \right) x_i
        \end{align*}
        Da $p = 0$ ein kritischer Punkt ist, gilt $\pderive[f]{x_i}(0) = 0$ für alle 
        $i$. Mit dem selben Argument sehen wir dann, dass
        \[ \pderive[f]{x_i}(tx) = 
        \sum_{j = 1}^n \left( \int_0^1 \pdderive[f]{x_i}{x_j}(stx) \opd s \right) x_j . \]
        Dann gilt 
        \[ f(x) = 
            \sum_{i, j = 1}^n 
            \left( \int_0^1 \int_0^1 \pdderive[f]{x_i}{x_j} \opd s \opd t \right) 
            x_i x_j 
        . \]
        Setze also 
        \[ b_{ij}(x) = \int_0^1 \int_0^1 \pdderive[f]{x_i}{x_j} \opd s \opd t . \]
        Dann gilt schon $B_0  = A$, und die Abbilfungen $b_{ij}$ sind glatt, also 
        auch $x \mapsto B_ x$.
    \end{smallproof}
    Wir dürfen nun das vorherige Lemma~\ref{lemma: lina lemma} anwenden:

    Sei $P \colon N \to GL_n(\R)$ eine Abbildung wie in~\ref{lemma: lina lemma}.
    Setze $P(B_x) := Q_x$ Definiere nun eine glatte Abbildung $\phi \colon U \to \R^n$
    durch $\phi (x) = Q_x^{-1}x$ in einer Umgebung von $0$. Wir rechnen nach, dass 
    $\opd \phi (0) \colon \R^n \to \R^n$ die Identität ist:

    Schreibe $Q_x^{-1} = (q_{ij}(x))_{ij}$. Dann
    \[ \phi(x) = \left( 
        \sum_{k = 1}^n q_{1k}(x) x_k, \cdots , \sum_{k = 1}^n q_{nk}(x) x_k
        \right)
    \]
    Also 
    \begin{align*}
        \pderive[\phi_i]{x_j} (x) 
            = & \pderive{x_j} \left( \sum_{k = 1}^n q_{ik} (x) x_k \right) \\
        = & \sum_{k = 1}^n \left( 
            \pderive[q_{ik}]{x_j}(x) x_k + q_{ik}(x) \delta_{ki}
        \right)
    , \end{align*}
    Wobei $\delta_{ki}$ das Kronecker Delta ist. Setzen wir also $0$  in $\phi$ ein
    bekommen wir
    \[ \pderive[\phi_i]{x_j}(0) = q_{ij}(0). \]
    Das Differential von $\phi$ in $0$ ist also gegeben durch
    \[ Q_0^{-1} = P(B_0)^{-1} = P(A)^{-1} = E_n . \]
    Das differential an der Stelle $0$ ist also invertierbar, und dann können wir mit dem 
    Satz über die Umkehrfunktion annehmen, dass $U$ klein genug ist, sodass $\phi$ 
    eingeschräkt aufs Bild ein Diffeomorphismus ist. 
    Dann ist $\phi$ eine Karte um $0$. Setze $(y_1, ..., y_n) := \phi$, dann gilt 
    \begin{align*}
        f(x) = & x^T B_x x 
            = (Q_x \phi(x))^T B_x (Q_x \phi(x)) \\
        = & \phi(x)^T (Q_x^T B_x Q_x) \phi(x) 
            = \phi(x)^T A \phi(x) \\
        = & \sum_{i = 1}^n a_{ii} y_i(x)^2
    . \end{align*}
    Das entspricht genau der gewünschten Form.
\end{proof}

\begin{corollary}
    Jeder nicht degenerierte kritische Punkt besitzt eine offene Umgebung, in der sich keine weiteren
    kritischen Punkte befinden.
\end{corollary}
