\chapter{Der Morse-Komplex}

In diesem Kapitel wird der Morse Komplex definiert und gezeigt, dass der 
Morse-Komplex ein Kettenkomplex ist.

\section{Die stabile- und instabile Mannigfaltigkeit}

\begin{definition}[Stabile- und instabile Mannigfaltigkeit]
    \label{def: stabile und instabile mannigfaltigkeit}
    Es sei $f \colon M \to \R$ eine Morse-Funktion, $p$ ein kritischer Punkt von $f$ und $X$ ein
    Pseudo-Gradientenfeld von $f$. Die stabile Mannigfaltigkeit von $f$ ist die Menge
    \[ \stab (p) = \left\{ q \in M : \lim_{t \to + \infty} \phi_t(q) = p \right\} \]
    und die instabile Mannigfltigkeit ist
    \[ \unst (p) = \left\{ q \in M : \lim_{t \to - \infty} \phi_t(q) = p \right\} . \]
\end{definition}

Bevor wir die stabile- und instabile Mannigfaltigkeit eines kritischen Punktes weiter untersuchen, 
fixieren wir ein Paar Notationen zu Morse Umgebungen.

\begin{definition}[Notationen zu Morse Umgebungen]
    \label{notation: morse umgebung}
    Zuerst untersuchen wir eine quadratische Form, die die Form hat wie Funktionen in Morse
    Umgebungen, also
    \[ Q \colon \R^n \to \R ; Q(x_1, \dots, x_n) = - x_1 - \dots - x_k + x_{k + 1} + \dots + x_n \]
    für ein $1 \leq k \leq n$.
    Mit $x_- := (x_1, \dots x_k) \colon \R^n \to \R^k$ und $x_+ := (x_{k + 1} \dots x_n)$ gilt dann
    \[ Q = - \| x_- \|^2 + \| x_+ \|^2 . \]
    Der Gradient von $Q$ ist mit dem Standardskalarprodukt auf $\R^n$
    \[ \grad Q (x_-, x_+) = 2(x_-, x_+) . \]
    Seien nun $\eps, \eta > 0$. Dann setzen wir
    \[ U(\eps, \eta) := \left\{ x \in \R^n : - \eps < Q(x) < \eps 
    \text{ und } \| x_- \|^2 \| x_+ \|^2 \leq \eta(\eps + \eta) \right\} := U \]
    Wir definieren außerdem
    \begin{align*}
        \del_{\pm} U := & \left\{ x \in U: Q(x) = \pm \eps \text{ und } \|x_{\mp} \|^2 \leq \eta \right\} 
            \text{ und} \\
        \del_0 U := & \left\{ x \in \del U: \| x_- \|^2 \| x_+ \|^2 = \eta(\eps + \eta) \right\} .
    \end{align*}
    Dann gilt 
    \[ \del U = \del_+ \cup \del_- \cup \del_0 . \]
    Wir setzen nun $V_- = \langle e_1, \dots, e_k \rangle$ und 
    $V_+ = \langle e_1, \dots, e_n \rangle \subseteq \R^n$. $V_+$ ist der größte Vektorraum, 
    auf dem $Q$ positiv definit ist und $V_-$ der größte Vektorraum, auf dem $Q$ negativ definit ist. 
    Es gilt 
    \[ \del U \cap V_{\pm} \subseteq \del_{\pm} U . \]

    Ist nun $f \colon M \to \R$ eine Morse Funktion, $p$ ein kritischer Punkt von $f$ und $(U, \phi)$
    eine Morse Umgebung von $p$, dann hat $\phi \circ (f - f(p))$ auf $U$ genau die Form von $Q$, wenn
    $k$ der Index von $p$ ist. Dann gilt 
    \[ 
        \phi(\stab(p) \cap U) = V_- \cap \phi(U) \text{ und } 
        \phi(\unst(p) \cap U) = V_+ \cap \phi(U) .
    \]
\end{definition}

Wir sind nun bereit eine grundlegende aber wichtige Aussage zu beweisen.

\begin{prop}
    $\stab (p)$ und $\unst (p)$ sind Mannigfaltigkeiten mit 
    \[ \dim \unst (p) = n - \dim \stab (p) = \Index (p) \]
\end{prop}

\begin{proof}
    Es sei $(V, \psi)$ eine Morse Karte um $p$ in einer Form wie in ~\ref{notation: morse umgebung},
    also sodass $\eps, \eta > 0$ exitstieren, sodass $\psi(V) = U(\eps, \eta) := U$. Es sei außerdem
    $\phi$ der Fluss eines Pseudo-Gradientenfeldes von $f$. Dann ist 
    \[ \Phi \colon \psi^{-1}(\del_+ U \cap V_+) \times \R \to M ; \psi(q, t) = \phi_t(q) \]
    eine Einbettung und es gilt 
    \[ \stab (p) = \Ima \Phi \cup \psi^{-1}(U \cap V_+) . \]
    Tatsächlich ist 
    \[ \stab (p) - \Ima \Phi = \{ p \} . \]
    Außerdem ist $\del_+ U \cap V_+ = \{ x \in \R^n: \| x_+ \|^2 = \eps \} \isom S^{n - k - 1}$, 
    denn für alle $x \in V_+$ gilt sowieso schon $x_- = 0$. Also ist $ \stab(p)$ diffeomorph zum Raum 
    $S^{n - k - 1} \times (-\infty, \infty]/\sim$, in dem alle Punkte in $\infty$ zusammengeklebt 
    werden. Dieser Quotient ist wiederum diffeomorph zur offenen Kreisscheibe mit Dimension $n - k$.
    Genauso zeigt man, dass $W^u (p)$ diffeomorph zur offenen Kreisscheibe mit Dimension $k$ ist.
\end{proof}

\begin{prop}
    Es sei $f \colon M \to \R$ eine Morse-Funktion und $X$ ein Pseudo-Gradientenfeld von $f$. 
    Sei außerdem $M$ kompakt. Ist dann $\phi$ der Fluss von $X$, dann existieren für jeden Punkt 
    $p \in M$ kritische Punkte $q$ und $r$ von $f$, sodass
    \[ \lim_{t \to + \infty} \phi_t(p) = q \;\;\; 
    \text{ und } \;\;\; \lim_{t \to -\infty} \phi_t(p) = r \]
\end{prop}

\begin{proof}
    Wir zeigen die erste Aussage. Seien für kritische Punkte $q$ $(U_q, \psi_q)$ die Karten, 
    auf denen der Pseudo Gradient mit dem negativen Gradienten auf $\R^n$ übereinstimmt. 
    Es ist $\lim_{t \to + \infty} \phi_t(p) = q$, genau dann wenn
    der Fluss $\phi_{\bullet}(p)$ den Punkt $p$ irgendwann in die Umgebung 
    $\del_+\psi_q(U_q) \cap \stab(q)$ transportiert. Angenommen $\phi_{\bullet}(p)$ transportiert
    $p$ nie zu einem kritischen Punkt. Jedes mal wenn $\phi_{\bullet}(p)$ also ins Innere einer
    Morse-Umgebung $U_q$ gerät, muss diese Umgebung auch wieder verlassen werden. Da 
    $f \circ \phi_{bullet}(p)$ monoton ist, kann nachdem $\phi_{bullet}(p)$ die Morse-Umgebung $U_q$
    verlassen hat, nie wieder zu dieser zurückgekehrt werden.
    Sei also 
    \[ U = \bigcup_{q \in \Crit(f)} U_q \]
    und $t_0$ der Zeitpunkt an dem $\phi_{\bullet} (p)$ die Umgebung $U$ das letzte mal verlässt.
    Da $M - U$ keine kritischen Punkte von $f$ enthält existiert ein $\eps_0 > 0$, sodass für alle 
    $x \in M - U$ gilt 
    \[ \opd f (x) ((X(x))) \leq - \eps_ . \]
    Wir rechnen also: Für jedes $t \geq t_0$ gilt
    \begin{align*}
        f(\varphi_t(p) - f(\varphi_{t_0}(p))) = & 
            \int_{t_0}^t \derive[f \circ \phi_{\bullet}(p)]{s} (s) \opd s \\
        = & \int_{t_0}^t \opd f (\phi_s(p)) (X(\phi_s(p))) \opd s \\
        \leq & - \eps_0 (t - t_0) . 
    \end{align*}
    Also für $t \to + \infty$ gilt $f(\phi_t(p)) \to - \infty$. Das kann aber nicht sein, denn da 
    $M$ kompakt ist muss auch $\Ima f$ kompakt sein. Also kann $\phi_{\bullet}(p)$ nicht alle 
    $U_q$ verlassen. aber dann ist 
    \[ \lim_{t \to + \infty} \phi_t(p) = q \]
    für einen kritischen Punkt $q$.
    Genauso zeigt man, dass $\lim_{t \to - \infty} \phi_t(p) = r$ für einen kritischen Punkt $r$.
\end{proof}

\section{Der Raum der gebrochenen Trajektorien}

Das ist der Raum der gebrochenen Trajektorien.

\begin{theorem}[Klassifizierung kompakter 1-Mannigfaltigkeiten]
    Es sei $M$ eine kompakte zusammenhängende Mannigfaltigkeit mit Rand. Dann ist
    \begin{itemize}
        \item $M$ diffeomorph zu $S^1$, falls $\del M = \varnothing$
        \item $M$ diffeomorph zu $[0, 1]$, falls $\del M \neq \varnothing$
    \end{itemize}
\end{theorem}

\begin{proof}
    Betrachte zuerst den ersten Fall, also dass $M$ keinen Rand hat. Es sei 
    $\{ (U_1, \phi_1), \dots, (U_n, \phi_n) \}$ ein Atlas von $M$. Ohne Beschränkung der 
    Allgemeinheit können wir annehmen, dass alle $U_i$ zusammenhängend sind. 
    Da $M$ zusammenhängend ist, können wir außerdem annehmen, dass 
    $\bigcup_{i = 1}^k U_i$ für alle $k$ zusammenhängend ist. Dann existiert ein $1 \leq k < n$, 
    sodass $\bigcup_{i = 1}^k U_i$ diffeomorph zu $\R$ ist, aber $\bigcup_{i = 1}^{k + 1} U_i$ 
    nicht. Setze $U := \bigcup_{i = 1}^k U_i$ und $V = U_{k + 1}$. Seien $\phi \colon U \to \R$
    und $\psi \colon V \to \R$ Diffeomorphismen. 
    \begin{claim*}
        Ist $M = U \cup V$ eine 1-dimensionale Mannigfaltigkeit, $U$ und $V$ diffeomorph zu 
        $\R$, dann ist $M$ diffeomorph zu $S^1$ oder $\R$.
    \end{claim*}

    \begin{smallproof}
        Ohne Beschränkung der Allgemeinheit ist $U \not\subseteq V$ und \\ 
        $V \not\subseteq U$. Dann gibt es zwei Möglichkeiten:
        \begin{itemize}
            \item $\phi(U \cap V)$ und $\psi(U \cap V)$ sind offene Halbintervalle
            \item $\phi(U \cap V)$ und $\psi(U \cap V)$ sind jeweils die disjunkte Vereinigung
                zweier offener Halbintervalle
        \end{itemize}
        Im ersten Fall ist $M$ homeomorph zu $\R$:

        Wir dürfen annehmen, dass $\phi(U \cap V) = (- \infty, a)$ und $\psi(U \cap V) = (b, \infty)$,
        ansonsten ersetze $\phi$ mit $\-phi$ bzw. $\psi$ mit $-\psi$. Die Verkettung
        \shorthandoff{"}
        \[ \begin{tikzcd}
            (- \infty, a) = \phi(U \cap V) \arrow[r, "\phi^{-1}"] & 
                U \cap V \arrow[r, "\psi"] & 
                \psi(U \cap V) = (b, \infty)
        \end{tikzcd} \] 
        \shorthandon{"}
        
        Ist insbesondere injektiv, also auch monoton wachsend. 
        \todo{}
    \end{smallproof}
\end{proof}

\section{Der Morse-Komplex}
