\documentclass[a4paper,12pt,twoside]{scrreport}

\usepackage[lang=de,lib=true]{../general/preamble}

\begin{document}

\makeheaderempty

\begin{titlepage}
    \centering
    Bachelorarbeit Mathematik \\
    \color{DarkBlue}\rule{\linewidth}{1pt}
    \color{Black}\Huge Der Morse-Komplex und Morse-Homologie \\[14pt]
    \color{DarkBlue}\rule{\linewidth}{2pt}
    \color{Black}

    \vspace{3cm}
    \includegraphics[width=\textwidth]{../resources/Me-Titlepage-Color.jpeg}

    \vfill
    \small

    \textit{eingereicht von}
    \hfill
    \textit{beaufsichtigt von} \\
    Jakob Dimigen
    \hfill
    Prof. Ursula Ludwig

    \vspace{2cm}

    Universität Münster \\
    \vspace{0.02cm}
    \includegraphics[width=0.2\textwidth]{../resources/WWU_Logo.png}
\end{titlepage}

\tableofcontents

\begin{abstract}
    In der Morse Theorie werden glatte Abbildungen $f \colon M \to \R$, deren kritische Punkte 
    alle nicht degeneriert sind untersucht. Anhand einer solcher Abbildungen lassen sich
    Rückschlüsse auf topologische Eigenschaften der Mannigfaltigkeit $M$ ziehen. In dieser Arbeit
    wird der \textit{Morse-Komplex} definiert, und gezeigt, dass dieser isomorph zu einem zellulären 
    Kettenkomplex ist. Dafür wird anfangs eine kurze Einführung in die Morse-Theorie gegeben und 
    grundlegende Begriffe definiert. Im zweiten Kapitel werden Morse Funktionen und Pseudo-Gradienten 
    untersucht. Im dritten Kapitel wird bewiesen, dass der Morse Komplex ein Kettenkomplex ist und 
    im letzten Kapitel wird anhand der erarbeiteten Theorie eine zelluläre Struktur auf kompakten 
    Mannigfaltigkeiten konstruiert, deren zellulärer Kettenkomplex isomorph zum Morse-Komplex ist.
    Zu guter letzt werden einige bekannte Eigenschaften der zellulären Homologie anhand der 
    Morse Homologie bewiesen. Something something
\end{abstract}

\makeheaderfancy
\setcounter{page}{1}

\chapter{Traditionelle Morse-Theorie}

Anschauliche Beispiele, vielleicht die zu den Deformations-Lemmata? Dann müsste
ich aber auch noch die Deformations-Lemmata machen.

\input{1-einfuehrung/1-einfuehrung}
\input{1-einfuehrung/2-nicht-degeneriertheit-und-index}

\chapter{Morse-Funktionen und Pseudo-Gradienten}

Das Ziel dieses Kapitels ist es, Morse-Funktionen und Pseudo-Gradienten zu
definieren und ihre \todo{allgegen- wertigkeit ist nicht so ein schönes Wort}
\textit{allgegenwertigkeit} zu zeigen. Ein weiteres wichtiges Ergebnis ist
das \textit{Morse-Lemma}.

\input{2-morse-funktionen-und-pseudo-gradienten/1-morse-funktionen}
\input{2-morse-funktionen-und-pseudo-gradienten/2-vektorfelder-und-pseudo-gradienten}
\section{Topologische Eigenschaften anhand kritischer Punkte}

In diesem Abschnitt werden wir das erste Mal das Ausmaß der Möglichkeiten, die Morse Theorie 
bietet erfahren. Es werden die beiden Deformationslemmata bewiesen. Anhand dieser kann man 
die Morse Ungleichungen beweisen und sogar zeigen, dass jede (glatte) Mannigfaltigkeit
den Homotopietypen eines CW-Komplexes besitzt.

\begin{theorem}[Erstes Deformationslemma]
    \label{satz: erstes deformationslemma}
    Es sei $M$ eine glatte Mannigfaltigkeit und $f: M \rightarrow \R$ eine
    glatte Abbildung. Hat $f$ keine kritischen Werte im Intervall $[a, b]$ und 
    ist $f^{-1}[a, b]$ kompakt, so existiert ein Diffeomorphismus 
    $M^a \rightarrow M^b$, und $M^a$ ist ein Deformationsretrakt von $M^b$.
\end{theorem}

Die Idee des Beweises ist es, $M^a$ entlang der Richtung, in die $f$ am stärksten
steigt, also entlang des Gradientenfeldes mit einem Diffeomorphismus $\varphi$ 
"nach oben zu ziehen", bis $\varphi(f^{-1}(a)) = f^{-1}(b)$.

\begin{bigproof}[Beweis erstes Deformationslemma]
    Es existiert eine kompakte Umgebung $K \in M$ von $f^{-1}[a, b]$. Dies folgt
    aus Whitneys Einbettungssatz und dem Satz von Heine-Borel.
    Sei $\rho: M \to \R$ eine glatte, positive Funktion, sodass
    \[ \rho(p) = 1 / \langle \grad f, \grad f \rangle \]
    für alle $p \in f^{-1}[a, b]$ und die außerhalb von $K$ verschwindet und für
    die für alle $p \in K$, die keine kritischen Punkte sind, gilt: 
    \[ 0 \leq \rho(p) \leq 1 / \langle \grad f, \grad f \rangle \]
    Bemerke dass $\rho$ innerhalb von $f^{-1}[a, b]$ wohldefiniert 
    ist, da sich keine kritischen Punkte im Intervall $[a, b]$ befinden. 
    Definiere ein Vektorfeld $X$ durch
    \[ X(p) = \rho(p) \cdot \grad f (p) \]
    Dann hat $X$ kompakten Träger, erfüllt also die Vorraussetzungen von 
    Lemma~\ref{prop: kompaktes VF generiert 1-param. grp.}. Sei also $\varphi$ die
    einzigartige 1-Parameter Gruppe aus Diffeomorphismen, die von $X$ generiert
    wird. 
    Wir bekommen für jedes $p \in M$ eine Abbildung 
    $f \circ \varphi_{\bullet}(p): \R \to \R$.
    
    \begin{claim} 
        Für alle $p \in M$, $t_0 \in \R$ und $q = \varphi_{t_0}(q)$
        ist $\derive{t} f \circ \varphi_{\bullet}(p) (t_0) \in [0, 1]$ und falls $f(\varphi_t(q)) \in [a, b]$
        gilt sogar $\derive{t} f \circ \varphi_{\bullet}(q) (t_0) = 1$.
    \end{claim}

    \begin{smallproof}
        Für $q = \varphi_{t_0}(p)$:
        \begin{align*}
            \derive{t} f \circ \varphi_{t_0}(p)
            & = T_{\varphi_{t_0}(p)} f \cdot T_{t_0}\varphi_{\bullet}(p) \left( \derive{t} \right)
            = \opd f (q) \cdot X(q) \\
            & = \langle X(q), \grad f (q) \rangle 
            = \rho(q) \langle \grad f (q), \grad f (q) \rangle \in [0, 1]
        \end{align*}
        
        $f \circ \varphi_{\bullet}(p)$ ist also monoton wachsend für alle $p \in M$.

        Falls sogar $f(\varphi_p(t_0)) \in [a, b]$, dann gilt
        \[ \frac{d}{dt} f \circ \varphi^p (t_0) = 1 \]
    \end{smallproof}

    \begin{claim} 
        Für $p \in f^{-1}(a)$, $t_0 \in [0, b-a]$ gilt $f(\varphi_{t_0}(p)) \in [a, b]$.
    \end{claim}
    
    \begin{smallproof}
        \[ f(\varphi_{t_0}(p)) \geq f(\varphi_0(p)) = a \]
        und
        \begin{align*}
            f(\varphi_t(p)) 
            & \leq f(\varphi_{b-a}(p)) \\
            & = \int_0^{b-a}\derive{t} f(\varphi_t(p)) \opd t + f(\varphi_0(p)) \\
            & = \int_0^{b-a}\rho(\varphi_t(p)) \langle \grad f (\varphi_t(p)), \grad f (\varphi_t(p)) \rangle \opd t + a \\
            & \leq \int_0^{b-a} 1 \, \opd t + a \\
            & = b
        \end{align*}
    \end{smallproof}

    \begin{claim} 
        Unter $\varphi_{b-a}$ wird die Niveaumenge 
        $f^{-1}(a)$ auf die Niveaumenge $f^{-1}(b)$ abgebildet.
    \end{claim}
     
    \begin{smallproof}
        Für $p \in f^{-1}(a)$ gilt:
        \[ \varphi_{a-a}(p) = \varphi_0(p) = p \]
        und für $t_0 \in [0, b - a]$ gilt wegen Behauptung 1 und 2
        \[ \derive{t}f(\varphi_{\id_{\R} - a}(p)) (t_0) = 1 \]
        also
        \[ f(\varphi_{b - a}(p)) = f(\varphi_{0}(p)) + (b - a) = b \]
        Genauso gilt für $q \in f^{-1}(b)$: $f(\varphi_{a - b}(q)) = a$, also 
        $\varphi_{b - a}(f^{-1}(a)) = f^{-1}(b)$.
    \end{smallproof}

    \begin{claim}
        $\varphi_{b - a} (M^a) = M^b$
    \end{claim}

    \begin{smallproof}
        "$\subseteq$": Sei $p \in M^a$. OBdA. existiert $s \in [0, b-a]$, sodass 
        $f(\varphi_s(p)) = a$, ansonsten gilt für alle 
        $s \in [0, b-a]: f(\varphi_s(p)) \leq a < b$. Dann gilt
        \[ f(\varphi_{b-a}(p)) \leq f(\varphi_{b-a+s}(p)) = f(\varphi_{b-a}(\varphi_s(p))) = b \] 
        "$\supseteq$": Analog.
    \end{smallproof}

    Damit ist $\left. \varphi_{b-a} \right\vert_{M^a}$ ein Diffeomorphismus zwischen
    $M^a$ und $M^b$. 

    Betrachte nun $r: M^b \times \R \to M^b$,
    \[  
        r(p, t) = \begin{cases}
            p & \text{ falls } f(p) \leq a \\
            \varphi_{t(a - f(p))}(p) & \text{ falls } a \leq f(p) \leq b 
        \end{cases}
    \]

    Dann ist $r$ stetig, $r(\cdot, 0)$ ist die Identität auf $M^b$, 
    $r(\cdot, 1)|_{M^a}$ ist die Identität auf $M^a$ und 
    $r(1, M^b) \subseteq M^a$, also ist $M^a$ ein Deformationsretrakt von $M^b$.

\end{bigproof}

\begin{theorem}[Zweites Deformations-Lemma]
    \label{satz: zweites deformationslemma}
    Es sei $M$ eine glatte Mannigfaltigkeit, $f: M \rightarrow \R$ eine glatte
    Abbildung und $p$ ein nicht-degenerierter kritischer Punkt mit Index 
    $k$. Sei $c := f(p)$ und $\varepsilon \geq 0$, sd. 
    $f^{-1}[c - \varepsilon, c + \varepsilon]$ kompakt ist und außer $p$ keine 
    weiteren kritischen Punkte von $f$ beinhaltet. Dann hat $M^{c-\varepsilon}$
    denselben Homotopietypen wie $M^{c - \varepsilon} \cup e^k$.
\end{theorem}

Die Idee für den Beweis ist, sich eine neue Funktion $F: M \to \R$ zu definieren,
die Außerhalb von einer kleinen Umgebung von $p$ $f$ entspricht und in der 
Umgebung etwas kleiner ist. Dann bekommen wir die folgende Situation:

\begin{figure}[H]
    \centering
    \includegraphics[width=0.8\linewidth]{../resources/Me-Diagram5-sublevelsets-of-f-and-F.jpeg}
    \label{fig:me-diagram5}
    \caption{Die Niveaumengen von $f$ (links) und $F$ (rechts)}
\end{figure}

Wir wollen also, dass $M^{c + \varepsilon} = F^{-1}(- \infty, c + \varepsilon]$ 
gilt und $F^{-1}(-\infty, c - \varepsilon]$ fast dasselbe ist wie 
$M^{c - \varepsilon}$, nur dass $F^{-1}(-\infty, c - \varepsilon]$ einen "Henkel"
enthält der den kritischen Punkt $p$ enthält.

\begin{bigproof}[Beweis zweites Deformationslemma]
    Sei $c := f(p)$. Mit dem Morse-Lemma können wir lokale Koordinaten 
    $\varphi = (u_1, ..., u_n)$ in einer Umgebung $U$ von $p$ wählen, sodass
    \[ f = c - u_1^2 - ... - u_k^2 + u_{k+1}^2 + ... + u_n^2 \]
    in dieser Umgebung, und sodass für den kritischen Punkt $p$ gilt:
    \[ u_1(p) = ... = u_n(p) = 0 \]

    Sei oBdA. $\varepsilon > 0$ klein genug, sodass 
    \begin{enumerate}
        \item $f^{-1}[c - \varepsilon, c + \varepsilon]$ kompakt ist und keine
            kritischen Punkte außer $p$ enthält
        \item $\{ x \in \R^n: \lVert x \rVert^2 \leq 2 \varepsilon \} \subseteq \varphi(U) $
    \end{enumerate}

    Wähle nun die $k$-Zelle 
    \[ 
        e^k := \{ p \in M: (u_1(p))^2 + ... + (u_k(p))^2 \leq \varepsilon 
        \text{ und } u_{k+1}(p) = ... = u_n(p) = 0 \} 
    \]

    Wir bekommen die folgende Situation:

    \begin{figure}[H]
        \centering
        \includegraphics[width=0.8\linewidth]{../resources/Me-Diagram6-U-parameterized.png}
        \label{fig:me-diagram6}
        \caption{U parametrisiert}
    \end{figure}

    Nun definiere eine glatte Funktion $\mu: \R \to \R$ mit den Eigenschaften:

    \begin{enumerate}
        \item $ \mu(0) > \varepsilon $
        \item $ \mu(r) = 0 $ falls $ r \geq 2 \varepsilon $
        \item $ -1 < \mu'(r) \leq 0 $ für alle $ r \in \R $
    \end{enumerate}

    Sei nun $F$ außerhalb von $U$ gleich $f$, und sei
    \[ F = f - \mu(u_1^2 + ... + u_k^2 + 2u_{k+1}^2 + ... + 2u_n^2) \]

    $F$ ist wohldefiniert und glatt, da $F$ außerhalb des Kreises mit Radius 
    $\sqrt{2\varepsilon}$ mit $f$ übereinstimmt und der gesamte Kreis in $U$ 
    enthalten ist. Damit haben wir einen guten Kandidaten foür $F$ gefunden.

    Wir definieren nun

    \begin{align*}
        & \eta, \xi: U \to [0, \infty) \\
        & \xi = u_1^2 + ... + u_k^2 \\
        & \eta = u_{k + 1}^2 + ... + e_n^2
    \end{align*}

    Dann gilt innerhalb von $U$:
    \[ f = c - \xi + \eta \]
    und 
    \[ F = f - \mu(\xi + 2 \eta) = c - \xi + \eta - \mu(\xi + 2 \eta) \]

    Jetzt wollen wir überprüfen:
    \begin{enumerate}
        \item $F^{-1}(-\infty, c + \varepsilon] = M^{c + \varepsilon}$.
        \item $F^{-1}(-\infty, c - \varepsilon]$ ist ein Deformationsretrakt von 
            $M^{c + \varepsilon}$.
        \item $M^{c - \varepsilon} \cup e^k$ ist ein Deformationsretrakt von
            $F^{-1}(-\infty, c - \varepsilon]$.
    \end{enumerate}

    Dann folgt schon die Behauptung.

    \begin{claim} 
        $F^{-1}(-\infty, c + \varepsilon] = M^{c + \varepsilon}$
    \end{claim}

    \begin{smallproof}
        Sei $q \in M$. Falls gilt $\xi(q) + 2 \eta(q) > 2 \varepsilon$ gilt 
        $F(q) = f(q) - \mu(\xi(q) + 2\eta(q)) = f(q)$,
        also gelte oBdA.
        \[ \xi(q) + 2 \eta(q) \leq 2 \varepsilon \]
        Dann:
        \[ F(q) \leq f(q) = c - \xi(q) + \eta(q) \leq c + \frac{1}{2}\xi(q) + \eta(q) \leq c + \varepsilon \]
    \end{smallproof}

    \begin{claim} 
        $F^{-1}(-\infty, c - \varepsilon]$ ist ein
        Deformationsretrakt von $M^{c + \varepsilon}$.
    \end{claim}

    \begin{smallproof}
        Bemerke: Die kritischen Punkte von $F$ stimmen mit denen von $f$ überein, 
        denn:

        \[ \pderive[F]{\xi} = -1 - \mu'(\xi + 2\eta)  < 0 \]
        und
        \[ \pderive[F]{\eta} = 1 - 2 \mu'(\xi + 2\eta) \geq 1 \]
        Insbsondere sind diese beiden Ableitungen also niemals $0$. Da 
        \[ \opd F = \pderive[F]{\xi}\opd \xi + \pderive[F]{\eta} \opd \eta \]
        und $\opd \xi$ und $\opd \eta$ nur in $p$ gleichzeitig Null sind, haben $f$ 
        und $F$  dieselben kritischen Punkte.

        Betrachte die Region $F^{-1}[c - \varepsilon, c + \varepsilon]$. Wegen 
        Behauptung 1 und der Tatsache, dass $F \leq f$ gilt:
        \[ F^{-1}[c - \varepsilon, c + \varepsilon] \subseteq f^{-1}[c - \varepsilon, c + \varepsilon] \]
        Da $f^{-1}[c - \varepsilon, c + \varepsilon]$ kompakt ist und 
        $F^{-1}[c - \varepsilon, c + \varepsilon]$ abgeschlossen ist, ist 
        $F^{-1}[c - \varepsilon, c + \varepsilon]$ auch kompakt. Da $f$ und $F$
        dieselben kritischen Punkte haben kann diese Menge maximal den kritischen 
        Punkt $p$ enthalten, aber
        \[ F(p) = c - \xi(p) + \eta(p) + \mu(\xi(p) + 2\eta(p)) = c - \mu(0) < c - \varepsilon \]
        Also gibt es in $F^{-1}[c - \varepsilon, c + \varepsilon]$ keine kritischen
        Punkte. Mit dem ersten Deformationslemma gilt dann:
        $F^{-1}(- \infty, c - \varepsilon]$ ist Def. Retrakt von 
        $F^{-1}(-\infty, c + \varepsilon] = M^{c + \varepsilon}$.
    \end{smallproof}

    \begin{claim}
        $M^{c - \varepsilon} \cup e^{k}$ ist ein 
        Deformationsretrakt von $F^{-1}(-\infty, c - \varepsilon]$.
    \end{claim}

    \begin{smallproof}
        Diese Aussage ergibt nur Sinn, falls 
        $M^{c - \varepsilon} \cup e^{k} \subseteq F^(-\infty, c - \varepsilon]$.
        Wir wissen schon, dass $M^{c - \varepsilon} \subseteq F^{-1}(c - \varepsilon]$.

        Sei $q \in e^k$, dann gilt $\xi(p) = 0 \leq \xi(q) \leq 1$ und 
        $\eta(p) = 0 = \eta(q)$. Da 
        $\pderive[F]{\xi} < 0$ gilt dann
        \[ F(q) \leq F(p) < c - \varepsilon \]

        Also ergibt sich folgende Situation:

        \begin{figure}[H]
            \centering
            \includegraphics[width=0.8\linewidth]{../resources/Me-Diagram7-handle.png}
            \label{fig:me-diagram7}
            \caption{Henkel}
        \end{figure}

        Die hellgrün eingefärbte Fläche ist $M^{c - \varepsilon}$ die hellgelbe
        zusammen mit der hellgrünen Fläch ist $F^{-1}(-\infty, c - \varepsilon]$. 

        Dafür konstruieren wir eine Deformationsretraktion
        $r: F^{-1}(-\infty, c - \varepsilon] \times [0,1] \to F^{-1}(-\infty, c - \varepsilon]$
        für $q \in F^{-1}(-\infty, c - \varepsilon], t \in [0, 1]$, die 
        $F^{-1}(-\infty, c - \varepsilon] - M^{c - \varepsilon}$ auf $e^k$ 
        deformiert, wie folgt.

        \[
            r(q, t) = \begin{cases}
                \varphi^{-1} \circ (u_1, ..., u_k, tu_{k + 1}, ..., tu_n)(q)
                    & \text{ im Fall 1: } \xi(q) \leq \varepsilon \\
                \varphi^{-1} \circ (u_1, ..., u_k, s_tu_{k + 1}, ..., s_tu_n)(q)
                    & \text{ im Fall 2: } \varepsilon \leq \xi(q) \leq \eta(q) + \varepsilon \\
                q & \text{ im Fall 3: } \eta(q) + \varepsilon \leq \xi(q)
            \end{cases}
        \]

        Wobei 

        \[ s_t = t + (1 -t)((\xi - \varepsilon)/\eta)^{1/2} \]

        Die Fälle sind dann wie folgt:

        \begin{figure}[H]
            \centering
            \includegraphics[width=0.8\linewidth]{../resources/Me-Diagram9-handle-cases.png}
            \label{me-diagram9}
            \caption{
                Fall 3 ist $M^{c - \varepsilon}$, also die grün eingefärbte Fläche, die
                orangene Fläche ist Fall 1 und die gelbe ist Fall 2.
            }
        \end{figure}

        Wir müssen überprüfen:
        \begin{enumerate}
            \item $r$ ist wohldefiniert und stetig
            \item $r(F^{-1}(-\infty, c - \varepsilon], 0) \subseteq M^{c - \varepsilon} \cup e^k$
            \item $r(\cdot, 1) = \id_{F^{-1}(-\infty, c - \varepsilon]}$ und 
                $\left. r(\cdot , 0) \right\vert_{M^{c - \varepsilon} \cup e^k} 
                = \id_{M^{c - \varepsilon} \cup e^k}$
        \end{enumerate}

        3. ist einfach nachzurechnen. In Fall 1 und Fall 3 ist 2. offensichtlich
        wahr. Für Fall 2 gilt:
        \begin{align*} 
            f(r(0, q)) & = 
                f\left( \varphi^{-1} \left(u_1(q), ..., u_k(q), 
                \left( \frac{\xi(q) - \varepsilon}{\eta(q)} \right)^{1/2}u_{k + 1}(q), ...,
                \left( \frac{\xi(q) - \varepsilon}{\eta(q)} \right)^{1/2}u_n(q)
                \right)
                \right) \\
            & = c - \xi(q)
                + \left( \left( \frac{\xi(q) - \varepsilon}{\eta(q)} \right)^{1/2}u_{k + 1}(q) \right)^2 + ... 
                + \left( \left( \frac{\xi(q) - \varepsilon}{\eta(q)} \right)^{1/2}u_n(q) \right)^2 \\
            & = c - \left( \frac{\xi(q) - \varepsilon}{\eta(q)} \right) \eta(q) \\
            & = c - \varepsilon
        \end{align*}
        also ist $r(0, q) \in f^{-1}(c - \varepsilon)$. Um 1. zu prüfen müssen wir 
        Stetigkeit in den Grenzfällen überprüfen:
        \begin{align*}
            & \text{For } \xi(q) = \varepsilon \text{ : }
                & s_t(q)  =t + (1 - t)((\varepsilon - \varepsilon)/\eta(q))^{1/2} = t \\
            & \text{For } \eta(q) + \varepsilon = \xi(q) \text{ : }
                & s_t(q) = t + (1 - t)((\xi(q) - \varepsilon)/(\xi(q) - \varepsilon))^{1/2} = 1
        \end{align*}

        Das einzig andere Problem was wir bekommen könnten ist nun in Fall 2 falls
        $\eta \to 0$. In Fall 1 und Fall 3 bekommen wir für $q$ mit $\eta(q) = 0$:
        $r(q, t) = \varphi^{-1} \circ (u_1, ..., u_k, 0, ..., 0)(q)$, also wollen
        wir zeigen dass für $\eta \in $ Fall 2 mit $\eta \to 0$ gilt $s_tu_i \to 0$
        für $i \in \{k+1, ..., n\}$. In Fall 2 gilt
        $0 \leq \xi - \varepsilon \leq \eta$. Dann gilt:

        \begin{align*}
            \lim\limits_{\eta \to 0} | s_t u_i |
            & = \lim\limits_{\eta \to 0} (1 - t)((\xi - \varepsilon)/\eta)^{1/2} | u_i | \\
            & \leq \lim\limits_{\eta \to 0} (1 - t)(\eta/\eta)^{1/2}|u_i| \\
            & = \lim\limits_{\eta \to 0} (1 - t)|u_i| = 0 
        \end{align*}
        
        Also ist $r$ stetig.
    \end{smallproof}

    Mit Behauptung 3 und 4 bekommen wir
    \[ M^{c + \varepsilon} \simeq F^{-1}(c - \varepsilon] \]
    und 
    \[ F^{-1}(-\infty, c - \varepsilon] \simeq M^{c - \varepsilon} \cup e^k \]
    Also folgt die Behauptung:
    \[ M^{c + \varepsilon} \simeq M^{c - \varepsilon} \cup e^k \]

\end{bigproof}

\chapter{Der Morse-Komplex}
In diesem Kapitel wird der Morse Komplex definiert und gezeigt, dass der 
Morse-Komplex ein Kettenkomplex ist.

\input{3-der-morse-komplex/1-die-stabile-und-instabile-mannigfaltigkeit}
\section{Der Morse-Komplex und der Raum der gebrochenen Trajektorien}

Wir sind nun bereit, den Morse-Komplex mit Koeffizienten in $\F_2$ (wenigstens) hinzuschreiben.
Wir fixieren für den gesamten Abschnitt eine glatte kompakte Mannigfaltigkeit $M$ und ein
Morse-Smale Paar $(f, X)$, und für jeden kritischen Punkt $p$ von $f$ eine Morse Umgebung 
$(\psi_p, \Omega(p))$, sodass $\psi(\Omega(p)) = U(\eps_p, \eta_p)$ wie in der Notation zu Morse 
Umgebungen~\ref{def: notation morse umgebung}. Dann definiere $C_k (M, (f, X))$ als das $\F_2$ Modul, 
das von den kritischen Punkten von $f$ mit Index $k$ erzeugt wird. Außerdem sei 
$n_X(p, q) = \# \Lt (p, q) \mod 2$. Dann definiere für einen kritischen Wert $p$:
\[ \del_X (p) := \sum_{\substack{ q \in \Crit(f) \\ \Index (p) + 1 = \Index (q) }} n_X(p, q)p . \]

Das Ziel dieses Abschnittes ist es zu zeigen, dass der Komplex $C_{\ast}(M, (f, X))$ wohldefiniert
ist, also dass gilt $n_X (p, q) < \infty$, und dass es ein Kettenkomple ist, also dass 
$\del_X \circ \del_X = 0$. Sobald das gezeigt wurde ist es ein Leichtes, den Komplex auch über die 
ganzen Zahlen zu definieren.

\subsection*{Wohldefiniertheit}

\begin{definition}[Der Raum der gebrochenen Trajektorien]
    \label{def: raum der gebrochenen trajektorien}
    Es seien $p$ und $q$ kritische Punkte von $f$. Der \textit{Raum der gebrochenen Trajektorien} ist
    \[ \Lb (p, q) = 
        \bigcup_{k \in \N} \left( \bigcup_{\substack{c_1, \dots, c_{k - 1} \\ \in \Crit(f)}} 
            \Lt (p, c_1) \times \Lt (c_1, c_2) \times \dots 
                \times \Lt (c_{k - 2}, c_{k - 1}) \times \Lt (c_{k - 1}, q) \right) . \]
\end{definition}

Obwohl die Formulierung recht sperrig wirkt ist sie doch intuitiv: 
$\ell \in \Lt (p, q)$ ist eine \glqq Verbindung\grqq{} zwischen den kritischen Punkten $p$ und $q$ 
entlang des Pseudo-Gradientenfeldes $X$. Ein Element 
$(\ell_1, ..., \ell_k) \in \Lt (p, c_1) \times \dots \times \Lt (c_{k - 1}, q) \subseteq \Lb (p, q)$
ist eine \glqq Verbindung\grqq{} zwischen $p$ und $q$ entlang des Pseudo-Gradientenfeldes $X$, die noch 
bei den kritischen Punkten $c_1, \dots, c_{k - 1}$ \glqq Halt\grqq{} macht. 

Offensichtlich gilt:
\begin{itemize}
    \item Ist $\Index (p) + 1 = \Index (q)$, so ist $\Lb (p, q) = \Lt (p, q)$.
    \item Ist $\Index (p) + 2 = \Index (q)$, so ist 
        $\Lb (p, q) = \Lt (p, q) \cup \bigcup_{c \in \Crit (f)} \Lt(p, c) \times \Lt(c, q)$
\end{itemize}

Wir werden sehen, dass man wie mit der Notation angedeutet $\Lb (p, q)$ mit einer Topologie ausstatten 
kann, sodass es die Kompaktifizierung von $\Lt (p, q)$ ist. (In der Tat ist ja 
$\Lt (p, q) \subseteq \Lb (p, q)$).

\begin{definition}[Topologie von $\Lb (p, q)$]
    Ese seien $p$ und $q$ kritische Punkte von $f$. Wir erinnern uns an unsere Vorstellung von Morse-
    Umgebungen $U = U(\eps, \eta)$ wie in~\ref{def: notation morse umgebung}:
    \begin{itemize}
        \item $\del_+ U$ sind alle Punkte auf dem Rand von $U$, auf denen Trajektorien von $X$ in 
            die Umgebung $U$ eintreten.
        \item $\del_- U$ sind alle Punkte auf dem Rand von $U$, auf denen Trajektorien von $X$ die
            Umgebung $U$ verlassen.
        \item Die Trajektorien von $X$ verlaufen tangential zu $\del_0 U$.
    \end{itemize}
    Es sei nun 
    \[ \ell = (\lambda_1, \dots, \lambda_k) 
        \in \Lt (p, c_1) \times \dots \times \Lt(c_{k - 1}, q) \subseteq \Lb (p, q) . \]
    Seien $U_i = U_i (\eps_i, \eta_i)$ Morse Umgebungen von $c_i$ und $U_0$ und $U_k$ Morse 
    Umgebungen von $p$ und $q$. $\lambda_i \cap \del_+ U_i$ ist der Punkt, an dem $\lambda_i$
    in $U_i$ eintritt, und $\lambda_{i + 1} \cap \del_- U_i$ ist der Punkt, an dem $\lambda_{i + 1}$
    die Umgebung $U_i$ verlässt. Es sei $U_i^-$ eine Umgebung von $\lambda_i \cap \del_+ U_i$ in 
    $\del_+ U$ und $U_i^-$ eine Umgebung von $\lambda_{i + 1} \cap \del_- U_i$ in $\del_- U_i$. 
    Seien dann $U^- = \bigcup U_i^-$ und $U^+ = \bigcup U_i^+$. Dann definiere die Menge 
    $\mathcal{U} (\ell, U^-, U^+)$ wie folgt:

    Wir sagen 
    $\ell' = (\mu_1, ..., \mu_{k'}) \in \Lt (p, c_{i_1}) \times \dots \times \Lt (c_{i_{k'-1}}, q)$
    ist in $\mathcal{U}(\ell, U^-, U^+)$ enthalten, falls $\mu_j \cap U_j^+ \neq \varnothing$
    und $\mu_j \cap U_{j + 1}^- \neq \varnothing$. 
    Dann ist $\mathcal{W} \subseteq \Lb (p, q)$ offen genau dann, wenn es für jedes 
    $\ell \in \mathcal{W}$ Umgebungen $U^+$ und $U^-$ wie oben gibt, sodass 
    $\mathcal{U}(\ell, U^+, U^-) \subseteq \mathcal{W}$.
\end{definition}

\begin{remark}
    Die Topologie von $\Lt(p, q)$ als Quotient stimmt mit der von $\Lb(p, q)$ überein.
    \todo{}
\end{remark}

\begin{prop}
    \label{prop: Lb ist kompakt}
    Es seien $p$ und $q$ kritische Punkte von $f$. Dann ist $\Lb (p, q)$ kompakt.
\end{prop}

Um diese Proposition zu beweisen benötigen wir noch ein Lemma:

\begin{lemma}
    \label{lemma: konvergenz einer folge}
    Es sei $x \in M$ \emph{kein} kritischer Punkt von $f$. Sei außerdem $(x_n)_n$ eine Folge in
    $M$ die gegen $x$ kovnvergiert und seien $y_n$ und $y$ Punkte, die auf den selben Trajektorien
    wie $x_n$ und $x$ liegen. Es gelte außerdem $f(y_n) = f(y)$ für alle $n \in \N$. Dann gilt
    \[ \lim_{n \to + \infty} y_n = y . \]
\end{lemma}

\begin{proof}
    Es sei $U$ eine Umgebung von $\Crit (f)$. Dann ist $\opd f (\cdot) (X)$ nie Null, und ähnlich wie 
    Im Beweis vom ersten Deformationslemma~\ref{satz: erstes deformationslemma} betrachten wir das
    Vektorfeld 
    \[ Y = - \frac{1}{\opd f (\cdot) (X)} \cdot X \]
    Auf $M - U$. Sei $\phi$ die von $Y$ erzeugte 1-Parameter Gruppe aus Diffeomorphismen. 
    Da $Y$ in die selbe Richtung zeigt wie $X$, stimmen die Trajektorien von $Y$ mit denen von $X$ 
    überein und es gilt 
    \[ f(\phi_t(z)) = f(z) - t . \]
    Dann gilt
    \[ \lim_{n \to \infty} y_n 
    = \lim_{n \to \infty} \phi_{- f(y_n) + f(x_n)}(x_n) 
    = \lim_{n \to \infty} \phi_{- f(y) + f(x_n)}(x_n) 
    = \phi_{- f(y) + f(x)}(x) = y \]
\end{proof}

\begin{proof}[Beweis von Proposition~\ref{prop: Lb ist kompakt}]
    Es sei $(\ell_n)_n$ eine Folge in $\Lb (p, q)$. Um zu zeigen, dass $Lb (p, q)$ kompakt ist
    müssen wir zeigen, dass $(l_n)_n$ eine konvergente Teilfolge besitzt.

    Wir nehmen zuerst an, dass für alle $n \in \N$ die Trajektorie $\ell_n$ in $\Lt (p, q)$.
    Seien $U$ und $V$ Morse Umgebungen von $p$ und $q$ in der Form wie bei der eingeführten
    Notation für Morse Umgebungen~\ref{def: notation morse umgebung}.
    Es außerdem sei $\ell_n^- \in M$ der Punkt, an dem $\ell_n$ die Morse Umgebung $U$ verlässt 
    und $\ell_n^- \in M$ der Punkt, an dem $\ell_n$ in die Morse Umgebung $V$ eintritt.
    $\ell_n^-$ und $\ell_n^+$ sind im Schnitt von $\del U$ bzw. $\del V$ und der stabilen bzw. 
    instabilen Mannigfaltigkeit. Diese Schnitte sind Kugeloberflächen, also kompakt. Die Folgen 
    $(l_n^-)_n$ und $(\ell_n^+)_n$ haben also konvergente Teilfolgen, wir können demnach ohne 
    Beschränkung der Allgemeinheit annehmen, dass sie konvergent sind. Setze
    \[ \lim_{n \to \infty} \ell_n^- = p^- \text{ und } \lim_{n \to \infty} \ell_n^+ = q^+ . \]
    Sei $\phi$ die von $X$ erzeugte 1-Parameter Gruppe aus Diffeomorphismen, dann ist 
    $\gamma = \phi_{\bullet}(p^-)$ die Trajektorie von $p^-$. Sei 
    $c = \lim_{n \to \infty} \phi_t(p^-)$. $c$ ist nach 
    Proposition~\ref{prop: trajektorien enden in kritischen punkten} ein kritischer Punkt,
    also ist $\gamma \in \Lt (p, c)$. Es sei nun $W$ eine Morse-Umgebung von $c$, die auch
    die Form hat wie in~\ref{def: notation morse umgebung}. Da $\phi$ glatt ist, muss für $n$ 
    groß genug auch $\ell_n$ die Morse Umgebung $W$ von $c$ kreuzen. Sei $d_n^+ \in M$ der Punkt, an
    dem $\ell_n$ in $W$ eintritt. Dann gilt $d_n^+, d^+ \in \del_+ W$, also gilt $f(d_n^+) = f(d^+)$
    für alle $n$. Da $d_n^+$ auf der selben trajektorie wie $p_n^-$ liegt, und $d^+$ auf der selben 
    Trajekorie wie $p^-$,folgt da $\lim p_n^- = p^-$ mit dem letzten 
    Lemma~\ref{lemma: konvergenz einer folge}: 
    \[ \lim_{n \to \infty} d_n^+ = d^+ . \]
    Falls $c = q$, dann ist $\lim \ell_n = \gamma \in \Lt (p, q) \subseteq \Lb (p, q)$, also hat 
    dann die Folge $(\ell_n)_n$ eine konvergente Teilfolge. Es sei also $c \neq q$. Dann muss 
    $\ell_n$ die Morse Umgebung $W$ wieder durch einen Punkt $d_n^-$ verlassen. Wie oben können wir 
    ohne Beschränkung der Allgemeinheit annehmen, dass die Folge $(d_n^-)_n$ konvergent ist, da
    sie zumindest eine konvergente Teilfolge besitzt. Wir definieren dann $d^- = \lim d_n^-$. 
    $d^-$ liegt in der instabielen Mannigfaltigkeit von $c$, denn wäre dies nicht der Fall, dann 
    führt das zu einem Widerspruch:
    
    Angenommen $d^- \notin \unst (c)$. Dann wäre $d^-$ auf der Trajektorie von einem Punkt
    $d^+_{\ast} \in \del_+ W$, der nicht in $\stab (c)$ enthalten ist. Wieder wegen des vorherigen
    Lemmas~\ref{lemma: konvergenz einer folge} ist dann $\lim d_n^+ = d^+_{\ast}$, also gilt dann
    $d^+ = d^+_{\ast}$, aber es gilt $d^+ \in \stab{c}$.
    
    Wir können nun wieder mit dem selben Argument zeigen, dass dann die Trajektorie von $d^-$ im
    kritischen Punkt $q$ endet, also liegt dann $\lim \ell_n$ in $\Lt (p, c) \times \Lt (c, q)$.

    Jetzt fehlt uns noch der allgemeine Fall. Wir müssen also für eine Folge $(\ell_n)_n$ in 
    $\Lb (p, q)$ zeine konvergente Teilfolge finden. Wegen der Glattheit von $\phi$ können wir annehmen,
    dass für $n$ groß genug alle $\ell_n$ die Form 
    \[ \ell_n = (\ell^1_n, \dots, \ell^k_n) \in \Lt (p, c_1) \times \dots \times \Lt (c_{k - 1}, q) \]
    haben. Wir finden mit der vorheringen Überlegung komponentenweise eine Teilfolge, sodass wir 
    für den grenzwert maximal noch $k - 1$ kritische Punkte als \glqq Zwischenstopp\grqq{} einfügen
    müssen. 
\end{proof}

\begin{remark}
    Sind nun $p$ und $q$ kritische Punkte von $f$ mit $\Index (p) = \Index (q) + 1$, dann ist 
    $\Lt(p, q)$ $0$-dimensionale Mannigfaltigkeit. Außerdem ist $\Lt (p, q)$ eine Abgeschlossene
    Teilmenge von $\Lb (p, q)$, und wie wir in der letzten Proposition~\ref{prop: Lb ist kompakt}
    gezeigt haben, ist $\Lb(p, q)$ kompakt, also auch $\Lt(p, q)$, also ist $\Lt (p, q)$ endlich.
    Damit ist schon mal $n_X (p, q)$ wohldefiniert.
\end{remark}

\subsection*{Der Morse Komplex ist ein Kettenkomplex}

Wir wollen zeigen, dass der Morse-Komplex tatsächlich ein Kettenkomplex ist, also dass $\del^2 = 0$.
Dafür genügt es zu zeigen, dass für einen kritischen Punkt $p$ mit Index $k + 1$ gilt $\del^2 (0) = 0$, 
also dass für jeden weiteren kritischen Punkt mit Index $k + 1$ gilt, dass die Zahl
$\# (\Lb (p, q) - \Lt (p, q))$ gerade ist. Wir benutzen die folgende Aussage, ohne sie zu beweisen:

\begin{theorem}[Klassifizierung kompakter 1-Mannigfaltigkeiten]
    \label{satz: klassifizierung kompakter 1-mannigfaltigkeiten}
    Es sei $M$ eine kompakte zusammenhängende Mannigfaltigkeit mit Rand. Dann ist
    \begin{itemize}
        \item $M$ diffeomorph zu $S^1$, falls $\del M = \varnothing$
        \item $M$ diffeomorph zu $[0, 1]$, falls $\del M \neq \varnothing$
    \end{itemize}
\end{theorem}

% \begin{proof}
%     Betrachte zuerst den ersten Fall, also dass $M$ keinen Rand hat. Es sei 
%     $\{ (U_1, \phi_1), \dots, (U_n, \phi_n) \}$ ein Atlas von $M$. Ohne Beschränkung der 
%     Allgemeinheit können wir annehmen, dass alle $U_i$ zusammenhängend sind. 
%     Da $M$ zusammenhängend ist, können wir außerdem annehmen, dass 
%     $\bigcup_{i = 1}^k U_i$ für alle $k$ zusammenhängend ist. Dann existiert ein $1 \leq k < n$, 
%     sodass $\bigcup_{i = 1}^k U_i$ diffeomorph zu $\R$ ist, aber $\bigcup_{i = 1}^{k + 1} U_i$ 
%     nicht. Setze $U := \bigcup_{i = 1}^k U_i$ und $V = U_{k + 1}$. Seien $\phi \colon U \to \R$
%     und $\psi \colon V \to \R$ Diffeomorphismen. 
%     \begin{claim*}
%         Ist $M = U \cup V$ eine 1-dimensionale Mannigfaltigkeit, $U$ und $V$ diffeomorph zu 
%         $\R$, dann ist $M$ diffeomorph zu $S^1$ oder $\R$.
%     \end{claim*}

%     \begin{smallproof}
%         Ohne Beschränkung der Allgemeinheit ist $U \not\subseteq V$ und \\ 
%         $V \not\subseteq U$. Dann gibt es zwei Möglichkeiten:
%         \begin{itemize}
%             \item $\phi(U \cap V)$ und $\psi(U \cap V)$ sind offene Halbintervalle
%             \item $\phi(U \cap V)$ und $\psi(U \cap V)$ sind jeweils die disjunkte Vereinigung
%                 zweier offener Halbintervalle
%         \end{itemize}
%         Im ersten Fall ist $M$ homeomorph zu $\R$:

%         Wir dürfen annehmen, dass $\phi(U \cap V) = (- \infty, a)$ und $\psi(U \cap V) = (b, \infty)$,
%         ansonsten ersetze $\phi$ mit $\-phi$ bzw. $\psi$ mit $-\psi$. Die Verkettung
%         \[ \begin{tikzcd}
%             (- \infty, a) = \phi(U \cap V) \arrow[r, "\phi^{-1}"] & 
%                 U \cap V \arrow[r, "\psi"] & 
%                 \psi(U \cap V) = (b, \infty)
%         \end{tikzcd} \] 
        
%         Ist insbesondere injektiv, also auch monoton wachsend. 
%         \todo{}
%     \end{smallproof}
% \end{proof}

\begin{prop}
    \label{prop: gebrochene trajektorien sind 1-dim mannigfaltigkeit}
    Es seiein $p$ und $q$ kritische Punkte von $f$ mit $\Index (p) = k + 1$ und $\Index (q) = k - 1$
    für ein $k \in \N_0$. Dann ist $\Lb (p, q)$ eine 1-dimensionale Mannigfaltigkeit mit Rand, und das
    Innere von $\Lb (p, q)$ ist $\Lt (p, q)$.
\end{prop}

Mit dieser Proposition folgt dann mit der Kalssifizierung von $1$-Mannigfaltigkeiten mit 
Rand~\ref{satz: klassifizierung kompakter 1-mannigfaltigkeiten} schon, dass der Morse Komplex ein
Kettenkomplex ist.

\begin{bigproof}
    Wir wissen schon, dass $\Lt (p, q) \subseteq \Lb(p, q)$ eine $1$-dimensionale Mannigfaltigkeit ist. 
    Um sagen zu können, dass $\Lb (p, q)$ eine $1$-dimensionale Mannigfaltigkeit mit Rand ist, und 
    insbsondere, dass $\Lb (p, q)$ das Innere von $\Lt(p, q)$ ist, reicht die folgende Aussage über 
    $\Lb(p, q)$:
    Es sei $c$ ein weiterer kritischer Punkt mit Index $k$.
    Sei $\lambda_1 \in \Lt (p, c)$ und $\lambda_2 \in \Lt (c, q)$. Dann existiert eine offene Umgebung
    $U \subseteq \Lb (p, q)$ von $(\lambda_1, \lambda_2)$, ein $\delta > 0$ und ein Homeomorphismus
    $\psi \colon [0, \delta) \to U$, sodass gelten: 
    \begin{enumerate}
        \item $\psi|_{(0, \delta)}$ ist glatt.
        \item $\psi(0) = (\lambda_1, \lambda_2)$.
        \item $\psi((0, \delta)) \subseteq \Lt (p, q)$.
        \item Für jede Folge $(\ell_n)_n$ in $\Lt (p, q)$ die gegen $(\lambda_1, \lambda_2)$ konvergiert 
            gilt $\ell_n \in \Ima \psi$ für $n$ groß genug.
    \end{enumerate} 
    Die letzten beiden Bedingungen stellen sicher, dass $\Lt (p, q)$ tatsächlich das Innere von 
    $\Lb (p, q)$ ist.
    Wir begeben uns also auf die (recht lange) Suche nach einer solchen Abbildung $\psi$. 

    Wir machen ein Paar Konstruktionen. Sei $\alpha := f(c)$ und $(V, \psi)$ eine Morse Umgebung von
    $c$, $\eps$, $\eta$ und $\Omega(c)$ wie in der Notation zu Morse 
    Umgebungen~\ref{def: notation morse umgebung}. Dann sind 
    $f(\del_+ \Omega) = \alpha + \eps$ und $f(\del_- \Omega) = \alpha - \eps$ für ein $\eps > 0$.
    Außerdem gilt, wie schon vorher, dass 
    \begin{align*}
        S_+ (c) := & \stab (c) \cap f^{-1}(\alpha + \eps) \isom S^{n - k - 1} \\
        S_- (c) := & \unst (c) \cap f^{-1}(\alpha - \eps) \isom S^{k - 1} .
    \end{align*}
    Es sei $a_1 \in M$ der Punkt, an dem $\lambda_1$ auf $\Omega(c)$ trifft, also 
    $a_1 = S_+ (c) \cap \lambda_1$, und $a_2$ der Punkt, an dem $\lambda_2$ die Umgebung $\Omega (c)$ 
    wieder verlässt, also $a_2 = S_- (c) \cap \lambda_2$. $\alpha + \eps$ ist kein kritischer Wert 
    von $f$ und es gilt $f^{-1}(\alpha + \eps) \pitchfork \unst (p)$, also ist mit 
    Proposition~\ref{prop: schnitt von transversalen untermannigfaltigkeiten} 
    $P = f^{-1}(\alpha) \cap \unst (p)$ eine Mannigfaltigkeit mit Dimension $(n - 1) + (k + 1) - n = k$.
    Da $X$ die Smale-Eigenschaft erfüllt gilt 
    $\unst (p) \supseteq P \pitchfork S_+ (c) \subseteq \stab(c)$, also ist $P \cap S_+ (c)$
    mit Proposition~\ref{prop: schnitt von transversalen untermannigfaltigkeiten} eine 
    Untermannigfaltigkeit der Dimenion $(k) + (n - k) - n = 0$. Offensichtlich gilt 
    $a_1 \in P \cap S_+ (c)$. Es sei $D^k_{\eps} = \{ x \in \R^k : \| x \| < \eps \}$. Dann existiert 
    eine Umgebung D von $a_1$ und ein Diffeomorphismus $\Psi : D \longto D^k_{\delta}$ mit 
    $\Psi(a_1) = 0$, sodass $P \supseteq D \cap S_+ (c) = a_1$ und $D \subseteq \del_+ \Omega (c)$.
    Wir versuchen die Kernidee des Beweises zu verstehen:

    \begin{figure}
        \centering
        \begin{minipage}{.5\textwidth}
          \centering
          \includegraphics[width=.85\linewidth]{../resources/bew-gebrochene-trajektorien-sind-1-dim-mannigfaltigkeit-1.JPG}
          \captionof{figure}{A figure}
          \label{fig: test1}
        \end{minipage}%
        \begin{minipage}{.5\textwidth}
          \centering
          \includegraphics[width=.85\linewidth]{../resources/bew-gebrochene-trajektorien-sind-1-dim-mannigfaltigkeit-2.JPG}
          \captionof{figure}{Another figure}
          \label{fig: test2}
        \end{minipage}
    \end{figure}

    Man betrachte die Abbildungen~\ref{fig: test1} und~\ref{fig: test2}.

    Wir versuchen, Menge $D - a_1$ entlang der Trajektorien von $X$ auf den Teil des Randes der Morse 
    Umgebung, an denen die Trajektorien austreten, via einer Abbildung $\Phi$ zu projizieren. Wir werden
    sehen, dass $Q = \Phi \cup S_+(c)$ eine Mannigfaltigkeit mit Rand ist, und dass $\stab (q)$ eine 
    $1-dimensionale Mannigfaltigkeit$ ist. Fügen wir dieser Mannigfaltigkeit den Punkt $a_2$ hinzu, dann
    können wir eine Umgebung von $(\lambda_1, \lambda_2)$ auf die gewünschte Art über die entstandene
    1-dimensionale Mannigfaltigkeit mit Rand parametrisieren. Also:

    \begin{claim}
        Es sei $\phi$ die von $X$ erzeugfte 1-Parameter Gruppe aus Diffeomorphismen. Für jedes 
        $x \in D - a_1$ existiert ein $t_x \in \R$, sodass $\phi_{t_x} (x) \in \del_- \Omega (c)$
        und $x \mapsto t_x$ glatt ist.
    \end{claim}

    \begin{smallproof}
        Via unserer anfangs gewählten Morse Karte $(V, \psi)$, und da wir ohne Einschränkungen $D$ 
        klein genug wählen können, sodass $\psi(D) \subseteq V$, können wir annehmen, dass sich alles im 
        $\R^n$ abspielt ; Sei also ohne Beschränkung $f(x_-, x_+) = - \|x_-\| + \|x_+\|$. Dann ist 
        $\phi$ gegeben durch 
        \[ \phi_t(x_-, x_+) = (e^{2t}x_-, e^{-2t}x_+) . \]
        Falls $(x_-, x_+) \in \del_+ U$ und $x_- \neq 0$, dann gilt auch $x_+ \neq 0$. Setze
        \[ t_{(x_-, x_+)} = \frac{1}{2} \ln \left( \frac{\| x_+ \|}{\| x_- \|} \right) . \]
        Dann gilt
        \[ \phi_{t_{(x_-, x_+)}} (x) = 
            \left( \frac{\| x_+ \|}{\| x_- \|} x_-, \frac{\| x_- \|}{\| x_+ \|} x_+ \right) . \]
        Die Zuordnung $(x_-, x_+) \mapsto t_{(x_-, x_+)}$ ist glatt und 
        \begin{align*}
            f(\phi_{t_{(x_-, x_+)}})(x_-, x_+) = & - \| \frac{\| x_+ \|}{\| x_- \|} x_- \|
                +  \| \frac{\| x_- \|}{\| x_+ \|} x_+ \| \\
                = & - \| x_+ \| + \| x_- \| \\
                = & - \eps .
        \end{align*}
        Es folgt $\phi_{t_{(x_-, x_+)}} (x) \in \del_- U$.
    \end{smallproof}

    Wir haben nun also eine Einbettung $\Phi$ von $D - a_1$ entlang der Trajektorien von $X$ gefunden.
    Wie am Anfang besprochen wollen wir jetzt zeigen:

    \begin{claim}
        Ist $\delta$ klein genug, dann ist $Q = \Phi (D - a_1) \cup S_(c)$ eine $k$ dimensionale 
        Mannigfaltigkeit mit Rand, und es gilt $\del Q = S_- (c)$.
    \end{claim}

    \begin{smallproof}
        Wider spielt sich alles via $\psi$ im $\R^n$ ab.
        Man betrachte die Projektion
        \[ \pi \colon \R^k \times \R^{n - k} ; \pi (x_-, x_+) = x_- . \]
        und ihre Einschränkung $\del_+U \to D_{\delta}^k$
        \todo{weiß net ob das stimmt, naja} Da $S_+ := \psi(\S_+(c)) = (\pi|_{\del_+U})^{-1}(0)$
        und $D \pitchfork S_+$, ist $0$ ein regulärer Wert von $\pi|_{\del_+U}$.
        Also ist $\opd \pi|_{\del_+U} (0)$ surjektiv, und da dim $\del_+U = k = \dim D^k_{\delta}$ ist
        das Differential auch invertierbar. Jetzt können wir den Satz über die Umkehrfunktion anwenden 
        bekommen lokal ein Inverses der Abbildung $\pi|_{\del_+U}$. Es existiert also ein 
        $\delta' \leq \delta$, sodass das inverse von $\pi|_{\del_+U}$ auf $D^k_{\delta'}$ definiert ist.
        Dann ist 
        \begin{align*}
            (\pi|_{del_+U})^{-1} \colon D^k_{\delta'} \longto & D \\
            x_- \longmapsto & (x_-, x_+) =: (x_-m h(x_-))
        \end{align*}
        ein Diffeomorphismus. Da $D \subseteq \del_+U \subseteq f^{-1}(\eps)$, gilt dann 
        $\| h(x_-) \|^2 = \| x_- \|^2 + \eps $. Ist dann 
        $g = \cfrac{h}{\| h \|} \colon D^k_{\delta'} \to S^{n - k - 1}$, dann gilt
        \[ D = \{ (x_-, h(x_-)): x \in D^k_{\delta'} \} 
            = \{ (x_-, \sqrt{\| x_- \|^2 + \eps} \cdot g(x_-)): x \in D^k_{\delta'} \} . \]
        Dann bekommen wir mit der Einbettung aus Behauptung 1 und da $\| g(x_-) \| = 1$:
        \[ \Phi(D - a_1) = 
            \left\{ \left( \frac{\sqrt{\| x_- \|^2 + \eps}}{\| x_- \|} \, x_-, \; 
                    \| x_- \| \, g(x_-) \right) : 
                x_- \in D^k_{\delta'} - 0 \right\} \]
        Wir können nun auf $D^k_{\delta'}$ Polarkoordinaten anwenden. wir erhalten einen Diffeomorphismus
        \begin{align*}
            H = \Phi \circ \rho \colon (0, \delta') \times S^{k-1} \longto & D \subseteq \del_-U \\
            (r, v) \longmapsto & ( \sqrt{r^2 + \eps} \cdot v, r \cdot g(\rho(r, v)))
        \end{align*}
        $g$ ist auf ganz $D^k_{\delta'}$ definiert, und wenigstens in einer Umgebung von $0$ beschränkt.
        Also können wir $H$ stetig in $0$ durch
        \[ H(0, v) = (\sqrt{\eps} \cdot v, \, 0) \]
        fortsetzen. Dann ist $H$ auch weiterhin eine (topologische) Einbettung 
        \[ H \colon [0, \delta') \times S^{k - 1} \longto \Phi(D - a_1) \cup S_- , \]
        und es gilt 
        \[ H(0, S^{k - 1}) = S_- . \]
    \end{smallproof}
\end{bigproof}

\subsection*{Der Morse Komplex über \texorpdfstring{$\Z$}{TEXT}}

\begin{definition}[Orientierung und Co-Orientierung von Mannigfaltigkeiten]
    Es sei $V$ ein (endlich dimensionaler) Vektorraum. Seien dann $\B_1$ und $\B_2$ zwei Basen von $V$.
    Wir sagen $\B_1$ und $\B_2$ induzieren dieselbe Orientierung, wenn 
    \[ \det \, _{\B_2}[\id_V]_{B_1} > 0 . \]
    \textit{Dieselbe Orientierung induzieren} ist eine Äquivalenzrelation. Eine Orientierung eines 
    Vektorraums ist eine Wahl einer Äquuivalenzklasse.

    Ein \textit{orientierter Atlas} einer $n$-dimensionalen Mannigfaltigkeit $M$ ist ein Atlas 
    $\mathcal{A}$ von $M$, sodass für alle Karten $(U, \phi)$ und $(V, \psi)$ in $\mathcal{A}$ 
    und alle Punkte $p \in M$ gilt
    \[ \det d \psi \circ \phi^{-1} (p) > 0 \]
    Eine \textit{Orientierung} einer Mannigfaltigkeit ist eine Auswahl eines maximalen orientierten Atlas.
    Eine Mannigfaltigkeit heißt orientierbar, falls eine Orientierung für die Mannigfaltigkeit existiert.
\end{definition}

\begin{remark}
    Man kann zeigen, dass es für jeden Vektorraum und jede Mannigfaltigkeit genau zwei Orientierungen
    gibt. Man sagt die ausgewählte Orientierung ist \textit{positiv} und die andere \textit{negativ}.
\end{remark}

Wir haben nun den Morse Komplex über $\F_2$ definiert. Wir wollen noch allgemeiner einen Komplex
über $\Z$ definieren. Die meiste Arbeit dafür ist nun schon gemacht. 
Wir können die Morse Umgebungen jedes kritischen Punktes orientieren. Sind $p$ und $q$ 
kritische Punkte mit $\Index (p) = \Index(q) + 1$, dann ist 
$\mathcal{W} (p, q) = \unst (p) \cap \unst (q)$ eine $1$-dimensionale kreisscheibe, also 
orientierbar. 


\chapter{Morse-Homologie und zelluläre Homologie}

In diesem Kapitel wird aus einem Morse-Smale Paar auf einer Mannigfaltigkeit
eine zelluläre Struktur dieser Mannigfaltigkeit konstruiert. Dann werden wir 
sehen, dass der Kettenkomplex, der von dieser Struktur induziert wird schon mit
dem Morse-Komplex übereinstimmt. Somit stimmt die Morse-Homologie mit der 
zellulären Homologie überein, also auch mit der singulären Homologie.

\input{4-morse-homologie-und-zellulaere-homologie/1-cw-komplexe}
\input{4-morse-homologie-und-zellulaere-homologie/2-cw-struktur-von-mannigfaltigkeiten}
\section{Morse-Homologie ist zelluläre Homologie}
\input{4-morse-homologie-und-zellulaere-homologie/4-anwendungen}

\appendix

\input{0-anhang}

\printbibliography

\eject
\end{document}