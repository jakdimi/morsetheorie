\section{Topologische Eigenschaften anhand kritischer Punkte}

In diesem Abschnitt werden wir das erste Mal das Ausmaß der Möglichkeiten, die Morse Theorie 
bietet erfahren. Es werden die beiden Deformationslemmata bewiesen. Anhand dieser kann man 
die Morse Ungleichungen beweisen und sogar zeigen, dass jede (glatte) Mannigfaltigkeit
den Homotopietypen eines CW-Komplexes besitzt.

\begin{theorem}[Erstes Deformationslemma]
    \label{satz: erstes deformationslemma}
    Es sei $M$ eine glatte Mannigfaltigkeit und $f: M \rightarrow \R$ eine
    glatte Abbildung. Hat $f$ keine kritischen Werte im Intervall $[a, b]$ und 
    ist $f^{-1}[a, b]$ kompakt, so existiert ein Diffeomorphismus 
    $M^a \rightarrow M^b$, und $M^a$ ist ein Deformationsretrakt von $M^b$.
\end{theorem}

Die Idee des Beweises ist es, $M^a$ entlang der Richtung, in die $f$ am stärksten
steigt, also entlang des Gradientenfeldes mit einem Diffeomorphismus $\varphi$ 
"nach oben zu ziehen", bis $\varphi(f^{-1}(a)) = f^{-1}(b)$.

\begin{bigproof}[Beweis erstes Deformationslemma]
    Es existiert eine kompakte Umgebung $K \in M$ von $f^{-1}[a, b]$. Dies folgt
    aus Whitneys Einbettungssatz und dem Satz von Heine-Borel.
    Sei $\rho: M \to \R$ eine glatte, positive Funktion, sodass
    \[ \rho(p) = 1 / \langle \grad f, \grad f \rangle \]
    für alle $p \in f^{-1}[a, b]$ und die außerhalb von $K$ verschwindet und für
    die für alle $p \in K$, die keine kritischen Punkte sind, gilt: 
    \[ 0 \leq \rho(p) \leq 1 / \langle \grad f, \grad f \rangle \]
    Bemerke dass $\rho$ innerhalb von $f^{-1}[a, b]$ wohldefiniert 
    ist, da sich keine kritischen Punkte im Intervall $[a, b]$ befinden. 
    Definiere ein Vektorfeld $X$ durch
    \[ X(p) = \rho(p) \cdot \grad f (p) \]
    Dann hat $X$ kompakten Träger, erfüllt also die Vorraussetzungen von 
    Lemma~\ref{prop: kompaktes VF generiert 1-param. grp.}. Sei also $\varphi$ die
    einzigartige 1-Parameter Gruppe aus Diffeomorphismen, die von $X$ generiert
    wird. 
    Wir bekommen für jedes $p \in M$ eine Abbildung 
    $f \circ \varphi_{\bullet}(p): \R \to \R$.
    
    \begin{claim} 
        Für alle $p \in M$, $t_0 \in \R$ und $q = \varphi_{t_0}(q)$
        ist $\derive{t} f \circ \varphi_{\bullet}(p) (t_0) \in [0, 1]$ und falls $f(\varphi_t(q)) \in [a, b]$
        gilt sogar $\derive{t} f \circ \varphi_{\bullet}(q) (t_0) = 1$.
    \end{claim}

    \begin{smallproof}
        Für $q = \varphi_{t_0}(p)$:
        \begin{align*}
            \derive{t} f \circ \varphi_{t_0}(p)
            & = T_{\varphi_{t_0}(p)} f \cdot T_{t_0}\varphi_{\bullet}(p) \left( \derive{t} \right)
            = \opd f (q) \cdot X(q) \\
            & = \langle X(q), \grad f (q) \rangle 
            = \rho(q) \langle \grad f (q), \grad f (q) \rangle \in [0, 1]
        \end{align*}
        
        $f \circ \varphi_{\bullet}(p)$ ist also monoton wachsend für alle $p \in M$.

        Falls sogar $f(\varphi_p(t_0)) \in [a, b]$, dann gilt
        \[ \frac{d}{dt} f \circ \varphi^p (t_0) = 1 \]
    \end{smallproof}

    \begin{claim} 
        Für $p \in f^{-1}(a)$, $t_0 \in [0, b-a]$ gilt $f(\varphi_{t_0}(p)) \in [a, b]$.
    \end{claim}
    
    \begin{smallproof}
        \[ f(\varphi_{t_0}(p)) \geq f(\varphi_0(p)) = a \]
        und
        \begin{align*}
            f(\varphi_t(p)) 
            & \leq f(\varphi_{b-a}(p)) \\
            & = \int_0^{b-a}\derive{t} f(\varphi_t(p)) \opd t + f(\varphi_0(p)) \\
            & = \int_0^{b-a}\rho(\varphi_t(p)) \langle \grad f (\varphi_t(p)), \grad f (\varphi_t(p)) \rangle \opd t + a \\
            & \leq \int_0^{b-a} 1 \, \opd t + a \\
            & = b
        \end{align*}
    \end{smallproof}

    \begin{claim} 
        Unter $\varphi_{b-a}$ wird die Niveaumenge 
        $f^{-1}(a)$ auf die Niveaumenge $f^{-1}(b)$ abgebildet.
    \end{claim}
     
    \begin{smallproof}
        Für $p \in f^{-1}(a)$ gilt:
        \[ \varphi_{a-a}(p) = \varphi_0(p) = p \]
        und für $t_0 \in [0, b - a]$ gilt wegen Behauptung 1 und 2
        \[ \derive{t}f(\varphi_{\id_{\R} - a}(p)) (t_0) = 1 \]
        also
        \[ f(\varphi_{b - a}(p)) = f(\varphi_{0}(p)) + (b - a) = b \]
        Genauso gilt für $q \in f^{-1}(b)$: $f(\varphi_{a - b}(q)) = a$, also 
        $\varphi_{b - a}(f^{-1}(a)) = f^{-1}(b)$.
    \end{smallproof}

    \begin{claim}
        $\varphi_{b - a} (M^a) = M^b$
    \end{claim}

    \begin{smallproof}
        "$\subseteq$": Sei $p \in M^a$. OBdA. existiert $s \in [0, b-a]$, sodass 
        $f(\varphi_s(p)) = a$, ansonsten gilt für alle 
        $s \in [0, b-a]: f(\varphi_s(p)) \leq a < b$. Dann gilt
        \[ f(\varphi_{b-a}(p)) \leq f(\varphi_{b-a+s}(p)) = f(\varphi_{b-a}(\varphi_s(p))) = b \] 
        "$\supseteq$": Analog.
    \end{smallproof}

    Damit ist $\left. \varphi_{b-a} \right\vert_{M^a}$ ein Diffeomorphismus zwischen
    $M^a$ und $M^b$. 

    Betrachte nun $r: M^b \times \R \to M^b$,
    \[  
        r(p, t) = \begin{cases}
            p & \text{ falls } f(p) \leq a \\
            \varphi_{t(a - f(p))}(p) & \text{ falls } a \leq f(p) \leq b 
        \end{cases}
    \]

    Dann ist $r$ stetig, $r(\cdot, 0)$ ist die Identität auf $M^b$, 
    $r(\cdot, 1)|_{M^a}$ ist die Identität auf $M^a$ und 
    $r(1, M^b) \subseteq M^a$, also ist $M^a$ ein Deformationsretrakt von $M^b$.

\end{bigproof}

\begin{theorem}[Zweites Deformations-Lemma]
    \label{satz: zweites deformationslemma}
    Es sei $M$ eine glatte Mannigfaltigkeit, $f: M \rightarrow \R$ eine glatte
    Abbildung und $p$ ein nicht-degenerierter kritischer Punkt mit Index 
    $k$. Sei $c := f(p)$ und $\varepsilon \geq 0$, sd. 
    $f^{-1}[c - \varepsilon, c + \varepsilon]$ kompakt ist und außer $p$ keine 
    weiteren kritischen Punkte von $f$ beinhaltet. Dann hat $M^{c-\varepsilon}$
    denselben Homotopietypen wie $M^{c - \varepsilon} \cup e^k$.
\end{theorem}

Die Idee für den Beweis ist, sich eine neue Funktion $F: M \to \R$ zu definieren,
die Außerhalb von einer kleinen Umgebung von $p$ $f$ entspricht und in der 
Umgebung etwas kleiner ist. Dann bekommen wir die folgende Situation:

\begin{figure}[H]
    \centering
    \includegraphics[width=0.8\linewidth]{../resources/Me-Diagram5-sublevelsets-of-f-and-F.jpeg}
    \label{fig:me-diagram5}
    \caption{Die Niveaumengen von $f$ (links) und $F$ (rechts)}
\end{figure}

Wir wollen also, dass $M^{c + \varepsilon} = F^{-1}(- \infty, c + \varepsilon]$ 
gilt und $F^{-1}(-\infty, c - \varepsilon]$ fast dasselbe ist wie 
$M^{c - \varepsilon}$, nur dass $F^{-1}(-\infty, c - \varepsilon]$ einen "Henkel"
enthält der den kritischen Punkt $p$ enthält.

\begin{bigproof}[Beweis zweites Deformationslemma]
    Sei $c := f(p)$. Mit dem Morse-Lemma können wir lokale Koordinaten 
    $\varphi = (u_1, ..., u_n)$ in einer Umgebung $U$ von $p$ wählen, sodass
    \[ f = c - u_1^2 - ... - u_k^2 + u_{k+1}^2 + ... + u_n^2 \]
    in dieser Umgebung, und sodass für den kritischen Punkt $p$ gilt:
    \[ u_1(p) = ... = u_n(p) = 0 \]

    Sei oBdA. $\varepsilon > 0$ klein genug, sodass 
    \begin{enumerate}
        \item $f^{-1}[c - \varepsilon, c + \varepsilon]$ kompakt ist und keine
            kritischen Punkte außer $p$ enthält
        \item $\{ x \in \R^n: \lVert x \rVert^2 \leq 2 \varepsilon \} \subseteq \varphi(U) $
    \end{enumerate}

    Wähle nun die $k$-Zelle 
    \[ 
        e^k := \{ p \in M: (u_1(p))^2 + ... + (u_k(p))^2 \leq \varepsilon 
        \text{ und } u_{k+1}(p) = ... = u_n(p) = 0 \} 
    \]

    Wir bekommen die folgende Situation:

    \begin{figure}[H]
        \centering
        \includegraphics[width=0.8\linewidth]{../resources/Me-Diagram6-U-parameterized.png}
        \label{fig:me-diagram6}
        \caption{U parametrisiert}
    \end{figure}

    Nun definiere eine glatte Funktion $\mu: \R \to \R$ mit den Eigenschaften:

    \begin{enumerate}
        \item $ \mu(0) > \varepsilon $
        \item $ \mu(r) = 0 $ falls $ r \geq 2 \varepsilon $
        \item $ -1 < \mu'(r) \leq 0 $ für alle $ r \in \R $
    \end{enumerate}

    Sei nun $F$ außerhalb von $U$ gleich $f$, und sei
    \[ F = f - \mu(u_1^2 + ... + u_k^2 + 2u_{k+1}^2 + ... + 2u_n^2) \]

    $F$ ist wohldefiniert und glatt, da $F$ außerhalb des Kreises mit Radius 
    $\sqrt{2\varepsilon}$ mit $f$ übereinstimmt und der gesamte Kreis in $U$ 
    enthalten ist. Damit haben wir einen guten Kandidaten foür $F$ gefunden.

    Wir definieren nun

    \begin{align*}
        & \eta, \xi: U \to [0, \infty) \\
        & \xi = u_1^2 + ... + u_k^2 \\
        & \eta = u_{k + 1}^2 + ... + e_n^2
    \end{align*}

    Dann gilt innerhalb von $U$:
    \[ f = c - \xi + \eta \]
    und 
    \[ F = f - \mu(\xi + 2 \eta) = c - \xi + \eta - \mu(\xi + 2 \eta) \]

    Jetzt wollen wir überprüfen:
    \begin{enumerate}
        \item $F^{-1}(-\infty, c + \varepsilon] = M^{c + \varepsilon}$.
        \item $F^{-1}(-\infty, c - \varepsilon]$ ist ein Deformationsretrakt von 
            $M^{c + \varepsilon}$.
        \item $M^{c - \varepsilon} \cup e^k$ ist ein Deformationsretrakt von
            $F^{-1}(-\infty, c - \varepsilon]$.
    \end{enumerate}

    Dann folgt schon die Behauptung.

    \begin{claim} 
        $F^{-1}(-\infty, c + \varepsilon] = M^{c + \varepsilon}$
    \end{claim}

    \begin{smallproof}
        Sei $q \in M$. Falls gilt $\xi(q) + 2 \eta(q) > 2 \varepsilon$ gilt 
        $F(q) = f(q) - \mu(\xi(q) + 2\eta(q)) = f(q)$,
        also gelte oBdA.
        \[ \xi(q) + 2 \eta(q) \leq 2 \varepsilon \]
        Dann:
        \[ F(q) \leq f(q) = c - \xi(q) + \eta(q) \leq c + \frac{1}{2}\xi(q) + \eta(q) \leq c + \varepsilon \]
    \end{smallproof}

    \begin{claim} 
        $F^{-1}(-\infty, c - \varepsilon]$ ist ein
        Deformationsretrakt von $M^{c + \varepsilon}$.
    \end{claim}

    \begin{smallproof}
        Bemerke: Die kritischen Punkte von $F$ stimmen mit denen von $f$ überein, 
        denn:

        \[ \pderive[F]{\xi} = -1 - \mu'(\xi + 2\eta)  < 0 \]
        und
        \[ \pderive[F]{\eta} = 1 - 2 \mu'(\xi + 2\eta) \geq 1 \]
        Insbsondere sind diese beiden Ableitungen also niemals $0$. Da 
        \[ \opd F = \pderive[F]{\xi}\opd \xi + \pderive[F]{\eta} \opd \eta \]
        und $\opd \xi$ und $\opd \eta$ nur in $p$ gleichzeitig Null sind, haben $f$ 
        und $F$  dieselben kritischen Punkte.

        Betrachte die Region $F^{-1}[c - \varepsilon, c + \varepsilon]$. Wegen 
        Behauptung 1 und der Tatsache, dass $F \leq f$ gilt:
        \[ F^{-1}[c - \varepsilon, c + \varepsilon] \subseteq f^{-1}[c - \varepsilon, c + \varepsilon] \]
        Da $f^{-1}[c - \varepsilon, c + \varepsilon]$ kompakt ist und 
        $F^{-1}[c - \varepsilon, c + \varepsilon]$ abgeschlossen ist, ist 
        $F^{-1}[c - \varepsilon, c + \varepsilon]$ auch kompakt. Da $f$ und $F$
        dieselben kritischen Punkte haben kann diese Menge maximal den kritischen 
        Punkt $p$ enthalten, aber
        \[ F(p) = c - \xi(p) + \eta(p) + \mu(\xi(p) + 2\eta(p)) = c - \mu(0) < c - \varepsilon \]
        Also gibt es in $F^{-1}[c - \varepsilon, c + \varepsilon]$ keine kritischen
        Punkte. Mit dem ersten Deformationslemma gilt dann:
        $F^{-1}(- \infty, c - \varepsilon]$ ist Def. Retrakt von 
        $F^{-1}(-\infty, c + \varepsilon] = M^{c + \varepsilon}$.
    \end{smallproof}

    \begin{claim}
        $M^{c - \varepsilon} \cup e^{k}$ ist ein 
        Deformationsretrakt von $F^{-1}(-\infty, c - \varepsilon]$.
    \end{claim}

    \begin{smallproof}
        Diese Aussage ergibt nur Sinn, falls 
        $M^{c - \varepsilon} \cup e^{k} \subseteq F^(-\infty, c - \varepsilon]$.
        Wir wissen schon, dass $M^{c - \varepsilon} \subseteq F^{-1}(c - \varepsilon]$.

        Sei $q \in e^k$, dann gilt $\xi(p) = 0 \leq \xi(q) \leq 1$ und 
        $\eta(p) = 0 = \eta(q)$. Da 
        $\pderive[F]{\xi} < 0$ gilt dann
        \[ F(q) \leq F(p) < c - \varepsilon \]

        Also ergibt sich folgende Situation:

        \begin{figure}[H]
            \centering
            \includegraphics[width=0.8\linewidth]{../resources/Me-Diagram7-handle.png}
            \label{fig:me-diagram7}
            \caption{Henkel}
        \end{figure}

        Die hellgrün eingefärbte Fläche ist $M^{c - \varepsilon}$ die hellgelbe
        zusammen mit der hellgrünen Fläch ist $F^{-1}(-\infty, c - \varepsilon]$. 

        Dafür konstruieren wir eine Deformationsretraktion
        $r: F^{-1}(-\infty, c - \varepsilon] \times [0,1] \to F^{-1}(-\infty, c - \varepsilon]$
        für $q \in F^{-1}(-\infty, c - \varepsilon], t \in [0, 1]$, die 
        $F^{-1}(-\infty, c - \varepsilon] - M^{c - \varepsilon}$ auf $e^k$ 
        deformiert, wie folgt.

        \[
            r(q, t) = \begin{cases}
                \varphi^{-1} \circ (u_1, ..., u_k, tu_{k + 1}, ..., tu_n)(q)
                    & \text{ im Fall 1: } \xi(q) \leq \varepsilon \\
                \varphi^{-1} \circ (u_1, ..., u_k, s_tu_{k + 1}, ..., s_tu_n)(q)
                    & \text{ im Fall 2: } \varepsilon \leq \xi(q) \leq \eta(q) + \varepsilon \\
                q & \text{ im Fall 3: } \eta(q) + \varepsilon \leq \xi(q)
            \end{cases}
        \]

        Wobei 

        \[ s_t = t + (1 -t)((\xi - \varepsilon)/\eta)^{1/2} \]

        Die Fälle sind dann wie folgt:

        \begin{figure}[H]
            \centering
            \includegraphics[width=0.8\linewidth]{../resources/Me-Diagram9-handle-cases.png}
            \label{me-diagram9}
            \caption{
                Fall 3 ist $M^{c - \varepsilon}$, also die grün eingefärbte Fläche, die
                orangene Fläche ist Fall 1 und die gelbe ist Fall 2.
            }
        \end{figure}

        Wir müssen überprüfen:
        \begin{enumerate}
            \item $r$ ist wohldefiniert und stetig
            \item $r(F^{-1}(-\infty, c - \varepsilon], 0) \subseteq M^{c - \varepsilon} \cup e^k$
            \item $r(\cdot, 1) = \id_{F^{-1}(-\infty, c - \varepsilon]}$ und 
                $\left. r(\cdot , 0) \right\vert_{M^{c - \varepsilon} \cup e^k} 
                = \id_{M^{c - \varepsilon} \cup e^k}$
        \end{enumerate}

        3. ist einfach nachzurechnen. In Fall 1 und Fall 3 ist 2. offensichtlich
        wahr. Für Fall 2 gilt:
        \begin{align*} 
            f(r(0, q)) & = 
                f\left( \varphi^{-1} \left(u_1(q), ..., u_k(q), 
                \left( \frac{\xi(q) - \varepsilon}{\eta(q)} \right)^{1/2}u_{k + 1}(q), ...,
                \left( \frac{\xi(q) - \varepsilon}{\eta(q)} \right)^{1/2}u_n(q)
                \right)
                \right) \\
            & = c - \xi(q)
                + \left( \left( \frac{\xi(q) - \varepsilon}{\eta(q)} \right)^{1/2}u_{k + 1}(q) \right)^2 + ... 
                + \left( \left( \frac{\xi(q) - \varepsilon}{\eta(q)} \right)^{1/2}u_n(q) \right)^2 \\
            & = c - \left( \frac{\xi(q) - \varepsilon}{\eta(q)} \right) \eta(q) \\
            & = c - \varepsilon
        \end{align*}
        also ist $r(0, q) \in f^{-1}(c - \varepsilon)$. Um 1. zu prüfen müssen wir 
        Stetigkeit in den Grenzfällen überprüfen:
        \begin{align*}
            & \text{For } \xi(q) = \varepsilon \text{ : }
                & s_t(q)  =t + (1 - t)((\varepsilon - \varepsilon)/\eta(q))^{1/2} = t \\
            & \text{For } \eta(q) + \varepsilon = \xi(q) \text{ : }
                & s_t(q) = t + (1 - t)((\xi(q) - \varepsilon)/(\xi(q) - \varepsilon))^{1/2} = 1
        \end{align*}

        Das einzig andere Problem was wir bekommen könnten ist nun in Fall 2 falls
        $\eta \to 0$. In Fall 1 und Fall 3 bekommen wir für $q$ mit $\eta(q) = 0$:
        $r(q, t) = \varphi^{-1} \circ (u_1, ..., u_k, 0, ..., 0)(q)$, also wollen
        wir zeigen dass für $\eta \in $ Fall 2 mit $\eta \to 0$ gilt $s_tu_i \to 0$
        für $i \in \{k+1, ..., n\}$. In Fall 2 gilt
        $0 \leq \xi - \varepsilon \leq \eta$. Dann gilt:

        \begin{align*}
            \lim\limits_{\eta \to 0} | s_t u_i |
            & = \lim\limits_{\eta \to 0} (1 - t)((\xi - \varepsilon)/\eta)^{1/2} | u_i | \\
            & \leq \lim\limits_{\eta \to 0} (1 - t)(\eta/\eta)^{1/2}|u_i| \\
            & = \lim\limits_{\eta \to 0} (1 - t)|u_i| = 0 
        \end{align*}
        
        Also ist $r$ stetig.
    \end{smallproof}

    Mit Behauptung 3 und 4 bekommen wir
    \[ M^{c + \varepsilon} \simeq F^{-1}(c - \varepsilon] \]
    und 
    \[ F^{-1}(-\infty, c - \varepsilon] \simeq M^{c - \varepsilon} \cup e^k \]
    Also folgt die Behauptung:
    \[ M^{c + \varepsilon} \simeq M^{c - \varepsilon} \cup e^k \]

\end{bigproof}