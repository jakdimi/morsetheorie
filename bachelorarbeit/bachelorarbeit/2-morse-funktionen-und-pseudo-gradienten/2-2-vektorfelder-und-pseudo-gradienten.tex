section{Vektorfelder und Pseudo-Gradienten}

Wir untersuchen erst ein Paar Eigenschaften von Vektorfeldern.

\begin{definition}[Flusslinie]
    \label{def: flussliene}
    Es sei $I \subseteq \R$ ein Intervall, $M$ eine glatte Mannigfaltigkeit und  
    $\gamma \colon I  \to M$ ein glatter Weg. Dann definiere für $t_0 \in \R$
    \[ \derive[\gamma]{t} (t_0) := 
        \opd \gamma (t_0) \left( \pderive{t} \right) \in T_{\gamma(t_0)}M \]
    wobei $\pderive{t}$ das von der Indentität auf $\R$ induziertze Element in $T_t\R$ ist.

    Es sei $X \in \VFs (M)$ ein Vektorfeld auf $M$. $\gamma$ heißt Flusslinie von $X$
    falls für alle $t_0 \in \R$ gilt: 
    \[ \derive[\gamma]{t}(t_0) = X(\gamma(t_0)) . \]
\end{definition}

\begin{definition}[1-Parameter Gruppe aus Diffeomorphismen]
    \label{def: 1-parameter gruppe aus diffeos}
    Es sei $M$ eine glatte Mannigfaltigkeit. Eine 
    \textit{1-Parameter Gruppe aus Diffeomorphismen} ist eine glatte Abbildung
    \begin{align*}
        \phi \colon \R \times M \longto & \; M \\
        (t, p) \longmapsto & \; \phi_t(p)
    \end{align*}
    sodass gelten: 
    \begin{itemize}
        \item Für alle $s, t \in \R$ gilt $\phi_{s + t} = \phi_s \circ \phi_t$ und
        \item $\phi_0 = \id_{M}$.
    \end{itemize}

    Für eine 1-Parameter Gruppe aus Diffeomorphismen $\phi$ schreiben wir 
    \[ \varphi_{\bullet}(p): \R \to M ; t \mapsto \varphi_t(p) . \]

    Es sei $X \in \VFs (M)$. Eine 1-Parameter Gruppe aus Diffeomorphismen $\phi$ heißt 
    \textit{von $X$ erzeugt}, falls für alle $p \in M$ gilt:
    \[ X(p) = \derive[\phi_{\bullet}(p)]{t}(0) \]
\end{definition}

\begin{remark}
    Wie der Name suggestiert, ist für jedes $t \in \R$ $\phi_t$ ein 
    Diffeomorphismus: Das Inverse von $\phi_t$ ist $\phi_{-t}$.

    Ist außerdem $\phi$ eine von einem Vektorfeld $X$ erzeugte 1-Parameter Gruppe aus 
    Diffeomorphismen, dann sind $\phi_{\bullet}(p)$ Flusslinien von $X$:
    \begin{align*}
        X(\varphi_{t_0}(p)) 
        & = \derive[\varphi_{\bullet}(\varphi_{t_0}(p))]{t}(0)
        = \opd \varphi_{\bullet} (0) (\varphi_{t_0}(p)) \left(\derive{t}\right) \\
        & = \opd (\varphi_{t_0 + \bullet}(p)) (0) \left(\derive{t}\right)
        = \opd (\varphi_{\bullet}(p)) (t_0) \cdot \opd (t_0 + \id_{\R}) (0) \left(\derive{t}\right) \\
        & = \opd (\varphi_{\bullet}(p)) (t_0) \left(\derive{t}\right)
        = \opd (\varphi_{\bullet}(p)) (t_0) \left(\derive{t}\right) \\
        & = \derive[\varphi_{\bullet}(p)]{t}(t_0)
    \end{align*}
\end{remark}

\begin{prop}
    \label{prop: kompaktes VF generiert 1-param. grp.}
    Es sei $M$ eine glatte Mannigfaltigkeit, $X \in \VFs (M)$ mit kompaktem Träger. Dann 
    generiert $X$ eine eindeutige 1-Parameter Gruppe aus Diffeomorphismen.
\end{prop}

\begin{proof}
    Für jeden Punkt $p \in M$ existiert eine Karten-Ungebung $(U_p, \phi_p)$. In dieser
    Umgebung hat das Anfangswertproblem
    \[ \derive[\gamma]{t} = X (\gamma) \; , \; \gamma(0) = p \]
    eine eindeutige Lösung in einem Intervall $[-\eps_p, \eps_p]$. Diese Lösung 
    $\gamma$ hängt glatt vom Anfangswert ab. Wir schreiben
    $\phi_{\bullet}(p) := \gamma$. In dieser Umgebung gilt schon \\ 
    $\phi_{t + s} = \phi_t \circ \phi_s$, solange $t, s, t + s \in [-\eps_p, \eps_p]$. 
    Da $\supp \, X$ kompakt ist existiert eine endliche Menge ${p_1, \dots, p_k}$, 
    sodass $\supp \, X \subseteq \bigcup_i U_{p_i}$. Es sei $\eps$ das Minimum der 
    $\eps_{p_i}$. Setze $\phi_t(p) = p$ für alle $p$ nicht im Träger von $X$. Wir haben nun 
    fast einen Kandidaten für die von $X$ generierte 1-Parameter Gruppe aus Diffeomorphismen;
    $\phi_t(p)$ ist definiert für alle $p \in M$ und $t \in [-\eps, \eps]$. Wir müssen also nur 
    noch einen Kandidaten für $\phi_t(p)$ finden, falls $|t| \geq \eps$.

    Wir können jede Zahl $t \in \R$ schreiben als $t = m \cdot \sfrac{\eps}{2} + r$ mit 
    $0 \leq r < \sfrac{\eps}{2}$ und $m \in \Z$. Sei nun zuerst $t \geq 0$, dann ist $m \geq 0$.
    Setze für alle $p \in M$ 
    \[ \phi_t(p) 
        := \phi_{\sfrac{\eps}{2}} \circ \dots \circ \phi_{\sfrac{\eps}{2}} \circ \phi_r , \] 
    Wobei wir $\phi_{\sfrac{\eps}{2}}$ $|m|$ mal anwenden. Falls $t < 0$ ersetze $\sfrac{\eps}{2}$
    mit $- \sfrac{\eps}{2}$. 
\end{proof}

\begin{remark}
    Falls $M$ eine kompakte Mannigfaltigkeit ist, dann generieren alle Vektorfelder eindeutige 
    1-Parametergruppen aus Diffeomorphismen.
\end{remark}

\begin{definition}[Riemannsche Metrik]
    \label{def: riemannsche metrik}
    Es sei $M$ eine Mannigfaltigkeit. Es sei 
    \[ g_p \colon T_pM \times T_pM \longto T_pM \]
    ein Skalarprodukt für jedes $p \in M$, sodass für alle $X, Y \in \VFs (M)$ die Abbildung 
    \[ p \longmapsto g_p(X(p), Y(p)) \]
    glatt ist. Dann heißt $g$ \textit{Riemmannsche Metrik} auf $M$. Wir schreiben für 
    $x, y \in T_pM$ 
    \[ \langle x, y \rangle := g_p(x, y) \text{ und } \| x \| := \sqrt{g_p(x, x)} . \]
\end{definition}

\begin{remark}
    Man kann zeigen, dass alle Mannigfaltigkeiten eine Riemannsche Metrik besitzen.
\end{remark}

\begin{definition}[Gradient]
    \label{def: gradient}
    Es sei $M$ eine glatte Mannigfaltigkeit, $f \colon M \to \R$ eine glatte Abbilding. Dann 
    ist der Gradient von $f$ das eindeutige Vektorfeld $\grad f$, sodass für alle $X \in \VFs (M)$
    gilt 
    \[ \langle X , \grad f \rangle = \opd f X . \]
\end{definition}

\begin{definition}[Pseudo-Gradient]
    \label{def: pseudo-gradient}
    Es sei $M$ eine Mannigfaltigkeit, $f \colon M \to \R$ eine glatte Funktion. $X \in \VFs (M)$
    heißt \textit{Pseudo-Gradient} oder \textit{Pseudo-Gradientenfeld} von $f$, falls gelten:
    \begin{itemize}
        \item $\opd f (p) (X(p)) \leq 0$ für alle $p \in M$, mit Gleichheit genau dann wenn 
            $p$ ein kritischer Punkt von $f$ ist.
        \item Für jeden kritischen Punkt $p$ von $f$ existiert eine Morse-Umgebung 
            $(U_p, \phi_p)$, in der $X (q) = - \opd (\phi_p^{-1}) (q) \cdot \grad (f \circ \phi_p^{-1})$.
    \end{itemize}
\end{definition}

\begin{prop}
    Es sei $M$ eine Mannigfaltigkeit und $f \colon M \to \R$ eine Morse-Funktion. 
    Dann existiert ein Pseudo-Gradientenfeld von $f$.
\end{prop}

\begin{proof}
    Da $M$ zweitabzählbar ist und die kritischen Punkte isoliert, ist die Menge der kritischen
    punkte ${p_i}_{i \in I'}$ abzählbar. Seien dann ${(U_i, \phi_i)}_{i \in I'}$ Karten-Umgebungen
    von den kritischen Punkten, sodass in diesen Umgebungen $f$ die Form hat wie im 
    Morse-Lemma~\ref{satz: morse-lemma}. Ergänze ${(U_i, \phi_i)}_{i \in I'}$ zu einem Atlas 
    ${(U_i, \phi_i)}_{i \in I}$, sodass jeder kritische Punkt $p_i$ nur in $U_i$ enthalten ist.
    definiere nun die Vektorfelder
    \[ X_i (p) := \opd (\phi_i) (p) \circ \grad (f \circ \phi_i^{-1}) (\phi_i(p)) \]
    auf $\phi_i(U_i)$. Setze nun
    \[ \tilde{X_i}(p) = \begin{cases}
        \lambda_i (p) \cdot X_i(p) & \text{ falls } p \in \phi_i(U) \\
        0 & \text{ sonst }
    \end{cases} . \]
    Per Definition gilt schon $\opd f (p) (X_i(p)) \leq 0$ für alle $p \in M$ und $i \in I$.
    Nun wähle eine Partition der 1 $(\lambda_i)_{i \in I}$ über $(U_i)_{i \in I}$. Dann setze
    \[ X := \sum_{i \in I} \tilde{X_i}(p) . \]
    Falls $p_i$ ein kritischer Punkt von $M$ ist, dann ist $\tilde{X_j}(p) = 0$ 
    für alle $j \neq i$. Also ist 
    \[ X(p) = \tilde{X_i}(p) = 0 . \]
\end{proof}