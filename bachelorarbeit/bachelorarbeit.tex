\documentclass[a4paper,11pt]{article}

\usepackage{../general/preamble}

\begin{document}

\begin{titlepage}
    \begin{center}
        \vspace*{1cm}
 
        \Large{\textbf{Der Morse-Komplex und die Morse-Homologie}}
 
        \vspace{0.5cm}
        Eine Bachelorarbeit \\ 
        Betreuerin Prof. Ursula Ludwig
             
        \vspace{1.5cm}
 
        \textbf{Jakob Dimigen}
             
    \end{center}
\end{titlepage}

\begin{abstract}
    Abstrct Text
\end{abstract}

\section{Zellulärer Kettenkomplex}

\begin{definition}[CW-Komplex]

\end{definition}

\begin{definition}[Zellulärer Kettenkomplex, Zelluläre Homologie]

\end{definition}

\section{Morse-Funktionen}

\begin{definition}[kritischer Punkt]

\end{definition}

\begin{definition}[Index, nicht degeneriertheit]

\end{definition}

\begin{lemma}[Unabhängigkeit von Karte]

\end{lemma}

\begin{theorem}[Morse Lemma]

\end{theorem}

\begin{theorem}[Existenz von Morse-Funktionen]
    Sei $M \subseteq \R^n$ eine Untermannigfaltigkeit. Dann ist für fast alle
    $p \in \R^n$ die Funktion
    \begin{align*}
        f: M &\to \R \\
        x &\mapsto \lVert x - p \rVert^2
    \end{align*}
    eine Morse-Funktion.
\end{theorem}

\section{Der Morse Komplex}

\begin{definition}[Pseudo-Gradient]

\end{definition}

\begin{definition}[Aufsteigende und Absteigende Mannigfaltigkeit]

\end{definition}

\begin{definition}[Morse-Komplex und Morse-Homologie]

\end{definition}

\begin{lemma}[Unabhängigkeit von $g$ und $f$]
    Die Morse Homologie einer Mannigfaltigkeit $M$ hängt nicht von dem gewählten
    Pseudo-Gradienten $g$ und der Morse-Funktion $f$ ab.
\end{lemma}

\begin{theorem}
    Die Morse-Homologie einer Mannigfaltigkeit entspricht der Zellulären 
    Homologie.
\end{theorem}

\end{document}